\documentclass{article}

% Packages

\usepackage{fullpage}
\usepackage{amsmath, amsthm, amsfonts, amssymb, mathtools, calrsfs, tensor, physics, MnSymbol,tikz-cd}
\usepackage[mathscr]{euscript}
\usepackage{graphicx}
\graphicspath{ {images/} }
\usepackage{enumitem}
\setlist[description]{font=\normalfont}

% Custom Commands

\newcommand*\diff{\mathop{}\!\mathrm{d}}
\newcommand*\Diff[1]{\mathop{}\!\mathrm{d^#1}}
\newcommand*\nrml{\vartriangleleft}
\newcommand*\scr[1]{\mathscr{#1}}
\newcommand*\bb[1]{\mathbb{#1}}
\newcommand*\la{\langle}
\newcommand*\ra{\rangle}
\newcommand*\gen[1]{\langle #1 \rangle}
\newcommand*\x{\times}
\newcommand*\st{\text{ s.t. }}
\newcommand*\ord[1]{\left\vert#1\right\vert}
\newcommand*\aut{\text{Aut}}
\newcommand*\lcm{\text{lcm}}
\newcommand*\mcal{\mathcal}
\newcommand*\es{\emptyset}
\newcommand*\im{\text{ Im }}
\newcommand*\N{\mathbb N}
\newcommand*\Z{\mathbb Z}
\newcommand*\R{\mathbb R}
\newcommand*\Q{\mathbb Q}
\newcommand*\C{\mathbb C}
\newcommand*\te[1]{\text{#1}}
\newcommand*\en[1]{\begin{enumerate}#1\end{enumerate}}
\newcommand*\e{\varepsilon}
\newcommand*\p[1]{\left(#1\right)}
\newcommand*\ps[1]{\left[#1\right]}
\newcommand*\pc[1]{\left\{#1\right\}}
\newcommand*\f[2]{\frac{#1}{#2}}
\newcommand*\mat[2]{\left(\begin{array}{#1}#2\end{array}\right)}
\newcommand*\ocross{\otimes}
\newcommand*\I{\te{i}}
\newcommand*\pd[3]{\frac{\partial^{#3} #1}{\partial {#2}^{#3}}}
\newcommand*\td[3]{\frac{d^{#3}#1}{d #2^{#3}}}
\newcommand*\m{\te{Mat}}
\newcommand*\End{\te{End}}
\newcommand*\irr{\te{Irr}}
\newcommand*\sgn{\te{sgn}}
\newcommand*\pn[2]{\left\|#1\right\|_{#2}}
\newcommand*\esssup{\te{ess sup}}
\newcommand*\essinf{\te{ess inf}}

% Miscellaneous

\newtheorem{theorem}{Theorem}
\usetikzlibrary{matrix,arrows,decorations.pathmorphing}

% Title
\title{P423 Homework 1}
\author{David Cyncynates \\ dcc57@case.edu}
\date{\today}

\begin{document}
\maketitle
\paragraph{1.3} The following are identities computed through the Levi-Civita symbol.
\begin{description}
\item{(a)} $(\vec a\times\vec b)\cdot(\vec c\times\vec d) = (\vec a\cdot\vec c)(\vec b\cdot\vec d)-(\vec a\cdot\vec d)(\vec b\cdot\vec c)$.
\begin{align}
(\vec a\times\vec b)\cdot(\vec c\times\vec d)&=(\epsilon_{ijk}a_j b_k)(\epsilon_{ilm}c_l d_m)\,,\\
&=\epsilon_{ijk}\epsilon_{ilm}(a_j b_k c_l d_m)\,,\\
&=\p{\delta_{jl}\delta_{km}-\delta_{jm}\delta_{kl}}(a_j b_k c_l d_m)\,,\\
&=a_j b_k c_j d_k-a_j b_k c_k d_j\,,\\
&=(a_j c_j) (b_k d_k)-(a_j d_j) (b_k c_k)\,,\\
&=(\vec a\cdot\vec c)(\vec b\cdot\vec d)-(\vec a\cdot\vec d)(\vec b\cdot\vec c)\,.
\end{align}
\item{(b)} $\nabla\cdot(\vec f\times\vec g) = \vec g\cdot(\nabla\times\vec f)-\vec f\cdot(\nabla\times\vec g)$. Consider the components of the LHS
\begin{align}
\ps{\nabla\cdot(\vec f\times\vec g)}_i &=\partial_i(\epsilon_{ijk} f_j g_k)\,,\\
&= \epsilon_{ijk}\partial_i(f_j g_k)\,,\\
&= \epsilon_{ijk}(g_k\partial_if_j +f_j \partial_i g_k)\,,\\
&=g_i (\epsilon_{ijk}\partial_j f_k) - f_i (\epsilon_{ijk}\partial_j g_k)\,,\\
&=g_i \ps{\nabla\times \vec f}_i- f_i \ps{\nabla\times\vec g}_i\,,\\
&=\vec g\cdot \p{\nabla\times \vec f}- \vec f\cdot \p{\nabla\times\vec g}\,.
\end{align}
The negative sign on the second term in equation 10 is from exchanging indices by an odd permutation, while the first term is acted on by an even permutation.
\item{(c)} $(\vec a\times\vec b)\times(\vec c\times\vec d) = (\vec a\cdot\vec c\times\vec d)\vec b-(\vec b\cdot\vec c\times\vec d)\vec a$. Consider the components of the LHS
\begin{align}
\ps{(\vec a\times\vec b)\times(\vec c\times\vec d)}_i&=\epsilon_{ijk}\ps{\vec a\times\vec b}_j\ps{\vec c\times\vec d}_k\,,\\
&=\epsilon_{ijk}\epsilon_{jmn}\epsilon_{kpq}a_mb_nc_p d_q\,,\\
&=\p{\delta_{km}\delta_{in}-\delta_{kn}\delta_{im}}\epsilon_{kpq}a_mb_nc_p d_q\,,\\
&=\delta_{km}\delta_{in}\epsilon_{kpq}a_mb_nc_p d_q-\delta_{kn}\delta_{im}\epsilon_{kpq}a_mb_nc_p d_q\,,\\
&=\p{a_k\epsilon_{kpq}c_p d_q}b_i-\p{b_k\epsilon_{kpq}c_p d_q}a_i\,,\\
&=\p{a_k\ps{\vec c\times\vec d}_k}b_i-\p{b_k\ps{\vec c\times\vec d}_k}a_i\,,\\
&=\p{\vec a\cdot \vec c\times\vec d}b_i-\p{\vec b\cdot \vec c\times\vec d}a_i\,,\\
&=\ps{\p{\vec a\cdot \vec c\times\vec d}\vec b-\p{\vec b\cdot \vec c\times\vec d}\vec a}_i\,.
\end{align}
\item{(d)} $(\vec\sigma\cdot\vec a)(\vec \sigma\cdot\vec b) = \vec a\cdot\vec b + \I\vec\sigma\cdot(\vec a\times\vec b)$. This follows immediately since the Pauli matrices are a 2 dimensional representation of the Quaternion group over $\C$. Or rather:
\begin{align}
(\vec\sigma\cdot\vec a)(\vec \sigma\cdot\vec b)&=\sigma_ia_i\sigma_jb_j\,,\\
&=\sigma_i\sigma_ja_ib_j\,,\\
&=\p{\delta_{ij}+\I\epsilon_{ijk}\sigma_k}a_ib_j\,,\\
&=a_ib_i+\I \sigma_i \epsilon_{ijk}a_jb_k\,,\\
&=a_ib_i+\I\vec\sigma\cdot (\vec a\times \vec b)\,.
\end{align}
Note the even permutation from 23 to 24.
\end{description}
\paragraph{1.21} Observe
\begin{align}
\f12\epsilon_{ijk}(\vec a\times\vec b)_k &= \f12\epsilon_{ijk}\epsilon_{klm}a_l b_m\,,\\
&=\f12\epsilon_{kij}\epsilon_{klm}a_l b_m\,,\\
&=\f12\ps{\delta_{il}\delta_{jm}-\delta_{im}\delta_{jl}}a_l b_m\,,\\
&=\f12\ps{a_i b_j-a_j b_i}\,,\\
&=-\f12\ps{a_i b_j+a_j b_i}+a_ib_j\,.
\end{align}
Hence
\begin{align}
a_ib_j=\f12\epsilon_{ijk}(\vec a\times\vec b)_k+\f12\ps{a_i b_j+a_j b_i}\,.
\end{align}
wwtbd.
\paragraph{3.11} A two-dimensional disk of radius $R$ carries a uniform charge per unit area $\sigma>0$.
\begin{description}
\item{(a)} Calculate the potential at any point on the symmetry axis of the disk.
\\\\
The Maxwell equations imply that the electric field on a charge distribution is given by (3.8). Following the example in the text, we have
\begin{align}
\vec E(z) &= \f{\sigma}{4\pi\epsilon_0}\hat z\int_0^Rr\diff r\int_0^{2\pi}\diff\theta\f{z}{(z^2+r^2)^{3/2}}\,,\\
&= \f{\sigma}{4\pi\epsilon_0}\hat z\int_0^R\diff r\f{2\pi r z}{(z^2+r^2)^{3/2}}\,,\\
&= \f{\sigma}{2\epsilon_0}\hat z\ps{1-\f{z}{\sqrt{z^2+R^2}}}\,.
\end{align}
We can check this answer against what we expect for an infinite sheet of charge. Picking a box of size length $z$ centered on the distribution, we have $E = \f{\sigma z^2}{\epsilon_02z^2} = \f\sigma{2\epsilon_0}$, which is the limit in our case as $R\to\infty$.
\item{(b)} Calculate the potential at any point on the rim of the disk. Hint: Use a point on the rim as the origin.
\\\\
Observe that the potential is given by
\begin{align}
\phi(\te{rim}) &= \f{\sigma}{4\pi\epsilon_0}\int_0^{\pi}\diff\theta \int_0^{2R\sin\theta} r\diff r\f{1}{r}\,,\\
&= \f{R\sigma}{2\pi\epsilon_0}\int_0^{\pi}\diff\theta \sin\theta\,,\\
&= \f{R\sigma}{\pi\epsilon_0}\,.
\end{align}
\item{(c)} Sketch the electric field pattern everywhere in the plane of the disk.
\\\\
In the plane of the disk, we have rotational symmetry and reflection symmetry, so the only possible configuration of electric field lines is lines through the origin.
\item{(d)} Calculate the total energy $U_E$ of the disk.
\end{description}
\end{document}
