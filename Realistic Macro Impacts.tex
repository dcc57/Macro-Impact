\documentclass{article}

% Packages

\usepackage{fullpage}
\usepackage{amsmath, amsthm, amsfonts, amssymb, mathtools, calrsfs, tensor, physics, MnSymbol,tikz-cd}
\usepackage[mathscr]{euscript}
\usepackage{graphicx}
\graphicspath{ {images/} }
\usepackage{enumitem}
\setlist[description]{font=\normalfont}

% Custom Commands

\newcommand*\diff{\mathop{}\!\mathrm{d}}
\newcommand*\Diff[1]{\mathop{}\!\mathrm{d^#1}}
\newcommand*\nrml{\vartriangleleft}
\newcommand*\scr[1]{\mathscr{#1}}
\newcommand*\bb[1]{\mathbb{#1}}
\newcommand*\la{\langle}
\newcommand*\ra{\rangle}
\newcommand*\gen[1]{\langle #1 \rangle}
\newcommand*\x{\times}
\newcommand*\st{\text{ s.t. }}
\newcommand*\ord[1]{\left\vert#1\right\vert}
\newcommand*\aut{\text{Aut}}
\newcommand*\lcm{\text{lcm}}
\newcommand*\mcal{\mathcal}
\newcommand*\es{\emptyset}
\newcommand*\im{\text{ Im }}
\newcommand*\N{\mathbb N}
\newcommand*\Z{\mathbb Z}
\newcommand*\R{\mathbb R}
\newcommand*\Q{\mathbb Q}
\newcommand*\C{\mathbb C}
\newcommand*\te[1]{\text{#1}}
\newcommand*\en[1]{\begin{enumerate}#1\end{enumerate}}
\newcommand*\e{\varepsilon}
\newcommand*\p[1]{\left(#1\right)}
\newcommand*\ps[1]{\left[#1\right]}
\newcommand*\pc[1]{\left\{#1\right\}}
\newcommand*\f[2]{\frac{#1}{#2}}
\newcommand*\mat[2]{\left(\begin{array}{#1}#2\end{array}\right)}
\newcommand*\ocross{\otimes}
\newcommand*\I{\te{i}}
\newcommand*\pd[3]{\frac{\partial^{#3} #1}{\partial {#2}^{#3}}}
\newcommand*\td[3]{\frac{d^{#3}#1}{d #2^{#3}}}
\newcommand*\m{\te{Mat}}
\newcommand*\End{\te{End}}
\newcommand*\irr{\te{Irr}}
\newcommand*\sgn{\te{sgn}}
\newcommand*\pn[2]{\left\|#1\right\|_{#2}}
\newcommand*\esssup{\te{ess sup}}
\newcommand*\essinf{\te{ess inf}}

% Miscellaneous

\newtheorem{theorem}{Theorem}
\usetikzlibrary{matrix,arrows,decorations.pathmorphing}

% Title
\title{Realistic Macro Impacts}
\author{David Cyncynates \\ dcc57@case.edu}
\date{\today}

\begin{document}
\maketitle
\section{The Green's Function}
This should be correct. We have the differential equation
\begin{align}
v(r)^2\nabla^2\phi-\partial_t^2\phi=-\f{f}{\rho}
\end{align}
We want to write $\phi$ as eigenfunctions of the linear differential operator on the LHS. Fourier transforming in time $\phi\to\tilde\phi$, we may write an eigenvalue problem
\begin{align}
v(r)^2\nabla^2\tilde\phi+\lambda\tilde\phi=0\,.
\end{align}
Multiplying by $\f{r^2}{v(r)^2\tilde\phi}$ and supposing $\tilde\phi=R(r)Y(\Omega)$, we have
\begin{align}
\f{\partial_r(r^2\partial_r R)}{R}+\lambda\f{r^2}{v(r)^2}+\f{r^2\nabla_\Omega Y}{Y}=0
\end{align}
It is clear that $\f{r^2\nabla_\Omega Y}{Y}$ depends only on $\Omega$ and that the remaining terms only depend on $r$, hence the equation is separable, and our ansatz is justified. The solutions $Y$ are just the spherical harmonics with eigenvalues $r^2\nabla_{\Omega}Y=-l(l+1)Y$, $l\in\N\cup\{0\}$. $R$ obeys a more complicated equation
\begin{align}
\partial_r(r^2\partial_r R)+\lambda\f{r^2}{v(r)^2}R-l(l+1)R=0
\end{align}
which is just a Sturm--Liouville problem for the appropriate boundary conditions at $r=0$ and $r=\f ab$. The weight function is $\rho=\f{r^2}{v(r)^2}$. Since this equation is in terms of a Sturm--Liouville operator, its eigenfunctions form an orthogonal set with respect to $\rho$, i.e.
\begin{align}
\delta(\lambda-\lambda')=\int\diff r\rho(r) \bar R_\lambda(r)R_{\lambda'}(r)\,.
\end{align}
Assuming completeness of the eigenfunctions, we may write
\begin{align}
\delta(r-r')=\int\diff\lambda\rho(r)\bar R_\lambda(r)R_\lambda(r')\,.
\end{align}
We may decompose any function in terms of $R$, $Y$, and $e^{\I\omega t}$ as
\begin{align}
f(r,\Omega,t)=\sum_{l=0}^\infty\sum_{m=-l}^l\int\diff\lambda\diff\omega F_{lm}(\lambda,\omega)R_{\lambda l}(r)Y_{lm}(\Omega)e^{\I\omega t}\,.
\end{align}
Notice
\begin{align}
\ps{v(r)^2\nabla^2-\partial_t^2}\ps{R_\lambda(r)Y_{lm}(\Omega)e^{\I\omega t}}&=R_\lambda(r)Y_{lm}(\Omega)e^{\I\omega t}\ps{\f{v(r)^2\nabla_r^2R}{R}+\f{v(r)^2\nabla_\Omega^2Y}{Y}+\omega^2}\,,\\
&=R_\lambda(r)Y_{lm}(\Omega)e^{\I\omega t}\ps{-\lambda+\f{v(r)^2}{r^2}l(l+1)-\f{v(r)^2}{r^2}l(l+1)+\omega^2}\,,\\
&=R_\lambda(r)Y_{lm}(\Omega)e^{\I\omega t}\ps{-\lambda+\omega^2}\,.
\end{align}
Thus, we see that
\begin{align}
G(\vec r,t;\vec r',t')=\rho(r)\sum_{l=0}^\infty\sum_{m=-l}^l\int_{a^2b^2}^{\infty}\diff\lambda\int_{-\infty}^\infty\diff\omega\f{\bar R_{\lambda l}(r)R_{\lambda l}(r')\bar Y_{lm}(\Omega)Y_{lm}(\Omega') e^{\I\omega(t-t')}}{\omega^2-\lambda-\I\epsilon}\,.
\end{align}
This may be incorrect because of the dependence of $R$ on $l$... However, we can probably formulate (4) as a S-L problem with eigenvalue $-l(l+1)$ instead, and we'll obtain a second orthogonality relation for $R$. This should mean that (11) is correct up to a weight function.




\section{The Source}
We must find an appropriate body force potential $f$. As a practice problem, let's consider $f=\nabla g=-4\pi G\ps{M\delta(\vec r-\vec vt)+\delta\rho}$ as in the detection of a small black hole problem, where $\delta\rho=-\rho\nabla u$, where $u$ is the displacement field. Then the source term of the wave equation is
\begin{align}
-\f f\rho=\f{4\pi M G}{\rho}\delta(\vec r-\vec vt)-4\pi G\phi
\end{align}
Since $f$ has a dependent variable in it, we need to slightly modify our Green's function. The eigenvectors $\tilde u$ of $\tilde D$ will now satisfy
\end{document}
The velocity field in the Earth has the form $v(r)=a^2-b^2r^2$. The change in velocity occurs over several wavelengths of the oscillations we are concerned with, so a WKB approximation is appropriate if necessary. Since we are not concerned with the behavior of individual waves, it is not of analytic interest to pursue a semiclassical solution. We are most concerned with determining the Green's function of the wave equation so that we can consider a realistic source.

Our wave equation is of the form
\begin{align}
-\f f\rho=v(r)^2\nabla^2\phi-\partial_t^2\phi=D\phi\,.
\end{align}
Our differential operator is the sum of two Hermitian operators, and is therefore Hermitian itself. Its eigenvalues are the sum of the eigenvalues of its components. The eigenvalue problem we are interested in
\begin{align}
0=D u+\lambda u
\end{align}
can be reduced to a problem in a single variable since any function can be written in terms of a radial part and a sum over spherical harmonics. Moreover, we may Fourier transform in time, yielding
\begin{align}
u(\vec x,t)&=\sum_{l=0}^{\infty}\sum_{m=-l}^l\int\f{\diff\omega}{2\pi}\tilde u_{lm}(r,\omega)Y_{lm}(\theta,\phi)e^{\I\omega t}\,.
\end{align}
Converting (2) into spherical harmonics, we have
\begin{align}
\tilde D&=v(r)^2\ps{\nabla^2_r-\f{l(l+1)}{r^2}}+\omega^2\,,\\
&=\f{v(r)^2}{r^2}\ps{\partial_r(r^2\partial_r)-l(l+1)}+\omega^2\,,
\end{align}
and hence
\begin{align}
0&=\f{v(r)^2}{r^2}\ps{\partial_r(r^2\partial_r)-l(l+1)}\tilde u+\omega^2\tilde u+\lambda\tilde u.
\end{align}
An equation of the Sturm--Liouville type has the form
\begin{align}
0&=(s(r)\phi'(r))'+\p{\lambda\rho(r)-q(r)}\phi(r)
\end{align}
where $\rho$ is known as the weight function, $s,\rho,q$ must be continuous on the interval $[0,\f ab]$, and $s,\rho$ positive-valued on $(0,\f ab)$. Rewriting $\tilde D$ in clearer form
\begin{align}
0&=(r^2\tilde u')'+\pc{\lambda\f{r^2}{v(r)^2}-\ps{l(l+1)-\f{r^2}{v(r)^2}\omega^2}}\tilde u\,,
\end{align}
where we see that $s(r)=r^2$, $\rho(r)=\f{r^2}{v(r)^2}$, and $q(r)=l(l+1)-\rho(r)\omega^2$. Hence $\tilde D$ is a Sturm-Liouville operator. For the eigenfunctions of $D$ to be orthogonal, we must satisfy the homogeneous boundary conditions
\begin{align}
s(r)(\phi_1(r)\phi_2'(r)-\phi_1'(r)\phi_2(r))\vert_0^{\f ab}=0\,.
\end{align}
Assuming we have chosen suitable eigenfunctions,
\begin{align}
\delta(\lambda-\lambda')=\int\diff r\rho(r)\tilde u(r,\lambda)\tilde u(r,\lambda')\,.
\end{align}
Note, the spectrum of $\tilde D$ is continuous since the boundary conditions above are automatically satisfied when chosen at the endpoints $0$ and $\f ab$.
\\\\
Now, assuming completeness, we may write
\begin{align}
\delta(r-r')&=\int\diff \lambda f(\lambda)\tilde u(r,\lambda)\,,\\
\rho(r')\bar{\tilde u}(r',\lambda')&=\int\diff r\rho(r) \bar{\tilde u}(r',\lambda')\int\diff \lambda f(\lambda)\tilde u(r,\lambda)\,,\\
&=f(\lambda')\,.
\end{align}
Hence
\begin{align}
\delta(r-r')&=\int\diff \lambda \rho(r')\bar {\tilde u}(r',\lambda)\tilde u(r,\lambda)\,.
\end{align}
Now that we know how to write $\delta$ in terms of eigenfunctions of $\tilde D$, we write the full $\delta$-function
\begin{align}
\delta(r-r')\delta(\Omega-\Omega')\delta(t-t')&=\sum_{\abs{m}\leq l\in\Z}\int\f{\diff\omega}{2\pi}\diff \lambda \rho(r')\tilde {\bar u}_{lm}(r',\lambda)\tilde u_{lm}(r,\lambda)\bar Y_{lm}(\Omega')Y_{lm}(\Omega)e^{-\I\omega (t'-t')}\,,\\
&=\sum\Delta\,.
\end{align}
Applying $D$ to the right hand side yields
\begin{align}
D\sum\Delta=-\sum(\lambda+l(l+1)+\omega^2)\Delta\,.
\end{align}
Hence the Green's function of the problem is
\begin{align}
G(x,x')=-\sum_{\abs{m}\leq l\in\Z}\int\f{\diff\omega}{2\pi}\diff \lambda \f{\rho(r')\tilde {\bar u}_{lm}(r',\lambda)\tilde u_{lm}(r,\lambda)\bar Y_{lm}(\Omega')Y_{lm}(\Omega)e^{-\I\omega (t-t')}}{\lambda-\I\epsilon}\,.
\end{align}
We now employ the WKB ansatz (where $\delta\ll \f{a^2}{b^2R_E^2}$ *this is a guess - think about it!)
\begin{align}
\phi_\lambda(r)=\exp\p{\f1\delta\sum_{n=0}^{\infty}\delta^nS_{\lambda n}(r)}\,,
\end{align}
\section{Corrections!}
\begin{align}
0&=(r^2\tilde u')'+\pc{\lambda\f{r^2}{v(r)^2}-\ps{l(l+1)-\f{r^2}{v(r)^2}(\omega^2+4\pi G)}}\tilde u\,.
\end{align}
The solution is
\begin{align}
\phi&=4\pi M G\int\diff^4 x G(x,x')\f{\delta(\vec r-\vec vt)}{\rho}\,,\\
&=4\pi M G\int\diff t G(\vec v t,t;\vec r',t')\f1{\rho}\,.
\end{align}
$\rho$ is a function of radius, but for the purposes of checking my work, I'll treat it as constant. We now calculate $G$ in the case of $b=0$. I will do this two ways: first to calculate it with $b\neq 0$, perhaps in WKB, and then take the limit $b\to 0$, and to calculate $G$ with $b=0$ from the outset.
