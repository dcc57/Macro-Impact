\documentclass{article}

% Packages

\usepackage{fullpage}
\usepackage{amsmath, amsthm, amsfonts, amssymb, mathtools, calrsfs, tensor, physics,tikz-cd}
\usepackage[mathscr]{euscript}
\usepackage{graphicx}
\graphicspath{ {images/} }
\usepackage{enumitem}
\setlist[description]{font=\normalfont}

% Custom Commands

\newcommand*\diff{\mathop{}\!\mathrm{d}}
\newcommand*\Diff[1]{\mathop{}\!\mathrm{d^#1}}
\newcommand*\nrml{\vartriangleleft}
\newcommand*\scr[1]{\mathscr{#1}}
\newcommand*\bb[1]{\mathbb{#1}}
\newcommand*\la{\langle}
\newcommand*\ra{\rangle}
\newcommand*\gen[1]{\langle #1 \rangle}
\newcommand*\x{\times}
\newcommand*\st{\text{ s.t. }}
\newcommand*\ord[1]{\left\vert#1\right\vert}
\newcommand*\aut{\text{Aut}}
\newcommand*\lcm{\text{lcm}}
\newcommand*\mcal{\mathcal}
\newcommand*\es{\emptyset}
\newcommand*\im{\text{ Im }}
\newcommand*\N{\mathbb N}
\newcommand*\Z{\mathbb Z}
\newcommand*\R{\mathbb R}
\newcommand*\Q{\mathbb Q}
\newcommand*\C{\mathbb C}
\newcommand*\te[1]{\text{#1}}
\newcommand*\en[1]{\begin{enumerate}#1\end{enumerate}}
\newcommand*\e{\varepsilon}
\newcommand*\p[1]{\left(#1\right)}
\newcommand*\ps[1]{\left[#1\right]}
\newcommand*\pc[1]{\left\{#1\right\}}
\newcommand*\f[2]{\frac{#1}{#2}}
\newcommand*\mat[2]{\left(\begin{array}{#1}#2\end{array}\right)}
\newcommand*\ocross{\otimes}
\newcommand*\I{\te{i}}
\newcommand*\pd[3]{\frac{\partial^{#3} #1}{\partial {#2}^{#3}}}
\newcommand*\td[3]{\frac{d^{#3}#1}{d #2^{#3}}}
\newcommand*\m{\te{Mat}}
\newcommand*\End{\te{End}}
\newcommand*\irr{\te{Irr}}
\newcommand*\sgn{\te{sgn}}
\newcommand*\pn[2]{\left\|#1\right\|_{#2}}
\newcommand*\esssup{\te{ess sup}}
\newcommand*\essinf{\te{ess inf}}

% Miscellaneous

\newtheorem{theorem}{Theorem}
\usetikzlibrary{matrix,arrows,decorations.pathmorphing}

% Title
\title{Homework}
\author{David Cyncynates \\ dcc57@case.edu}
\date{\today}

\begin{document}
\maketitle
\section{Scalar Modes}
In the following, lower case denotes a quantity in position space while capital letters denote their components in Fourier space.
\\\\
Denote the displacement field $u(\vec x,t)=\nabla\phi(\vec x,t)+\nabla\times a(\vec x,t)$\,. The linearized (acoustic) wave equation for $\phi$ is then
\begin{align}
\alpha^2\nabla^2\phi&=\partial_t^2\phi\,,
\end{align}
where $\alpha^2=\f{\lambda+2\mu}{\rho}$. We may write
\begin{align}
\phi(\vec x,t)=\int\f{\diff^3 k}{(2\pi)^3}\Phi(\vec k,\omega)e^{\I(\vec k\cdot\vec x-\omega t)}\,,
\end{align}
from which we obtain the dispersion relation $\alpha k=\omega$, and hence the solution
\begin{align}
\phi(\vec x,t)=\int\f{\diff^3 k}{(2\pi)^3}\Phi(\vec k,\omega)e^{\I(\vec k\cdot\vec x-\alpha k t)}\,.
\end{align}
We now impose the constraint equation
\begin{align}
\sigma_{ij}&=\delta_{ij}\lambda\nabla\cdot u+\mu(u_{i,j}+u_{j,i})
\end{align}
which, for the scalar modes becomes
\begin{align}
\sigma_{ij}&=\delta_{ij}\lambda\nabla^2\phi+2\mu\partial_i\partial_j\phi
\end{align}
and whose trace is
\begin{align}
-p=K\nabla^2\phi\,,
\end{align}
where $K=\lambda+\f23\mu$ is the bulk modulus and $p=-\f13\tr\sigma_{ij}$. We take this constraint as an initial condition at $t=0$. In Fourier space
\begin{align}
P(\vec k)=Kk^2\Phi(\vec K)\,,
\end{align}
from which we obtain the displacement field components
\begin{align}
U(\vec k)&=\I\f{P(\vec k)}{K}\f{\vec k}{k^2}\,,
\end{align}
and hence
\begin{align}
u(\vec x,t)=\int\f{\diff^3 k}{(2\pi)^3}\I\f{P(\vec k)}{K}\f{\vec k}{k^2}e^{\I(\vec k\cdot\vec x-\alpha k t)}\,,
\end{align}
The equation for the energy of the compressional modes is
\begin{align}
E&=\f12\int_V\diff^3x\p{\rho\abs{\partial_t u}^2+(\lambda+2\mu)\abs{\nabla\cdot u}^2}\,,\\
&=\f{\lambda+2\mu}{K^2}\int\f{\diff^3 k}{(2\pi)^3}\abs{P(\vec k)}^2
\end{align}
In the case that $P$ depends only on the frequency
\begin{align}
E&=4\pi\f{\lambda+2\mu}{K^2}\int\f{\diff\tilde\lambda}{\tilde\lambda^4}\abs{P(\tilde\lambda)}^2\,.
\end{align}
Now consider the case of a step function type pressure source - a cylinder of height $h$ and radius $r_X$
\begin{align}
p(\vec x,0)&=p_0\theta(r_X-r)\ps{\theta(z+h/2)-\theta(z-h/2)}\,,
\end{align}
whose Fourier components are
\begin{align}
P(\vec k)&=\f{4\pi r_X p_0}{\sqrt{k_x^2+k_y^2}k_z}J_1\p{\sqrt{k_x^2+k_y^2}r_X}\sin\p{\f{h}{2}k_z}\,,
\end{align}
which, in polar $k$-space is
\begin{align}
P(k,\varphi,\theta)&=\f{4\pi r_X p_0}{k^2\sin\theta\cos\theta}J_1\p{r_Xk\sin\theta}\sin\p{\f{h}{2}k\cos\theta}\,,
\end{align}
Directly integrating in Mathematica (over $\theta$ first) yields
\begin{align}
E_{\te{total}}&=\f{\lambda+2\mu}{K^2}4\pi^2p_0^2\sigma_Xh
\end{align}
where $\sigma_X=\pi r_X^2$. To calculate the energy deposition into the low frequency spectrum, we integrate $k$ from $0$ to $k_0$. Observe, we can make the following approximation
\begin{align}
P(\vec k)&\approx_{k\ll r_X}\f{2\pi r_Xp_0}{k^2\cos\theta\sin\theta}r_X k\sin\theta\sin\p{\f{h}{2}k\cos\theta}\,,\\
&=\f{2\pi r_X^2 p_0}{k\cos\theta}\sin\p{\f{h}{2}k\cos\theta}\,.
\end{align}
From this we obtain the portion of the energy relegated to the long wavelength spectrum
\begin{align}
E_{\te{propagated}}&=\f{\lambda+2\mu}{K^2}\int\f{\diff^3 k}{(2\pi)^3}\abs{\f{2\pi r_X^2 p_0}{k\cos\theta}\sin\p{\f{h}{2}k\cos\theta}}^2\,,\\
&=\ps{\f{1}{2h}(r_X^2p_0)^2\f{\lambda+2\mu}{K^2}}\ps{hk_0\cos(hk_0)+\sin(hk_0)+hk_0\p{-2+hk_0\te{Si}(hk_0)}}\,,\\
&=\ps{(r_X^2p_0)^2\f{\lambda+2\mu}{K^2}}\ps{\f{\pi}{\lambda_0}\cos\p{\f{2\pi h}{\lambda_0}}+\sin\p{\f{2\pi h}{\lambda_0}}+\p{\f{\pi}{\lambda_0}}\p{-2+\p{\f{2\pi h}{\lambda_0}}\te{Si}\p{\f{2\pi h}{\lambda_0}}}}\,,\\
&\approx_{h\gg\lambda_0\gg1}\ps{(r_X^2p_0)^2\f{\lambda+2\mu}{K^2}}\ps{\f{\pi^3 h}{\lambda_0^2}}\,,\\
&=\ps{(\sigma_Xp_0)^2\f{\lambda+2\mu}{K^2}}\ps{\f{\pi h}{\lambda_0^2}}\,.
\end{align}
From this we obtain the fractional energy deposition into the unattenuated wavelengths
\begin{align}
\Xi&=\f{\sigma_X}{4\pi\lambda_0^2}\,.
\end{align}
This approximation holds for $\lambda_0^2\gg\sigma_X$, which is appropriate for the case that $\lambda_0$ is on the order of kilometers and $\sigma_X$ is on the order of centimeters squared.
\pagebreak
\section{Vector Modes}
The following is the equation of motion for the vector potential $a$ without body force:
\begin{align}
\mu\nabla^2a-\rho\ddot a=\mu\nabla(\nabla\cdot a)-\mu\nabla\times(\nabla\times a)-\rho\ddot a=0\,,
\end{align}
where the leftmost term is the vector laplacian. This gives us three wave equations for the three components of $a$. The constraint equation takes the form
\begin{align}
\sigma_{ij}&=\lambda\delta_{ij}\nabla\cdot \nabla\times a+\mu(\partial_j (\nabla\times a)_i+\partial_i (\nabla\times a)_j)\,,\\
&=\mu(\partial_j (\nabla\times a)_i+\partial_i (\nabla\times a)_j)\,,\\
&=\mu\mat{ccc}{2\partial_x(\partial_ya_z-\partial_za_y)&(\partial_y^2-\partial_x^2)a_z+\partial_z(\partial_x a_x-\partial_y a_y)&(\partial_x^2-\partial_z^2)a_y+\partial_y(\partial_z a_z-\partial_x a_x)\\
&2\partial_y(\partial_z a_x-\partial_x a_z)&(\partial_z^2-\partial_y^2)a_x+\partial_x(\partial_ya_y-\partial_za_z)\\
&&2\partial_z(\partial_xa_y-\partial_ya_x)}\,.
\end{align}
We now write $a_i$ in terms of its Fourier components
\begin{align}
a_i(\vec x,t)&=\int\f{\diff^3k}{(2\pi)^3}A_i(\vec k,\omega)e^{\I(\vec k\cdot\vec x-\omega t)}\,.
\end{align}
Observe the dispersion relation $\beta k=\omega$, which holds for each component $a_i$, and that $\beta=\f\mu\rho$. Thus, the solutions to the wave equation are
\begin{align}
a_i(\vec x,t)&=\int\f{\diff^3k}{(2\pi)^3}A_i(\vec k,\omega)e^{\I(\vec k\cdot\vec x-\beta k t)}\,.
\end{align}
We now use the constraint equation to write down the initial vector potential in terms of the initial stress. Note, we will obtain three equations constraining the possible initial stresses (i.e., we will be able to eliminate three components of $\sigma$ from the equations), and three equations writing $A_i$ in terms of $\Sigma_{ij}$.
\\\\
In Fourier space we have
\begin{align}
-\Sigma_{xx}&=\mu2k_x(k_y A_z-k_z A_y)\,,\\
-\Sigma_{yy}&=\mu2k_y(k_z A_x-k_x A_z)\,,\\
-\Sigma_{zz}&=\mu2k_z(k_x A_y-k_y A_x)\,,\\
-\Sigma_{xy}&=\mu\ps{(k_y^2-k_x^2) A_z+k_z(k_x A_x-k_y A_y)}\,,\\
-\Sigma_{yz}&=\mu\ps{(k_z^2-k_y^2) A_x+k_x(k_y A_y-k_z A_z)}\,,\\
-\Sigma_{zx}&=\mu\ps{(k_x^2-k_z^2) A_y+k_y(k_z A_z-k_x A_x)}\,.
\end{align}
First observe that any three of these equations together forms a singular linear system in $A_i$. Thus we cannot invert to solve for $A_i$. However, we can get some information about the $\Sigma$'s.
\\\\
Taking the sum of 31-33 yields
\begin{align}
0&=\Sigma_{xx}+\Sigma_{yy}+\Sigma_{zz}\,.
\end{align}
Observe that
\begin{align}
\f{-\Sigma_{xx}k_y}{2\mu k_xk_z}&=\f{k_y^2}{k_z}A_z-k_yA_y\,,\\
\f{-\Sigma_{yy}k_x}{2\mu k_yk_z}&=k_xA_x-\f{k_x^2}{k_z}A_z\,,\\
\f{-\Sigma_{xy}}{\mu k_z}&=\ps{\f{k_y^2-k_z^2}{k_z}}A_z+k_xA_x-k_yA_y\,.
\end{align}
Taking the sum of the first two reveals that`
\begin{align}
2k_xk_y\Sigma_{xy}&=k_y^2\Sigma_{xx}+k_x^2\Sigma_{yy}\,.
\end{align}
If you want to be silly, we find
\begin{align}
0&=(\Sigma_{x}k_y-\Sigma_{y}k_x)^2\,.
\end{align}
Anyway, we then have the system of three equations
\begin{align}
2k_xk_y\Sigma_{xy}&=k_y^2\Sigma_{xx}+k_x^2\Sigma_{yy}\,,\\
2k_yk_z\Sigma_{yz}&=k_z^2\Sigma_{yy}+k_y^2\Sigma_{zz}\,,\\
2k_zk_x\Sigma_{zx}&=k_x^2\Sigma_{zz}+k_z^2\Sigma_{xx}\,.
\end{align}
which is non-singular. They yield
\begin{align}
\Sigma_{xx}&=\f{k_x}{k_yk_z}\p{k_z\Sigma_{xy}-k_x\Sigma_{yz}+k_y\Sigma_{zx}}\,,\\
\Sigma_{yy}&=\f{k_y}{k_zk_x}\p{k_z\Sigma_{xy}+k_x\Sigma_{yz}-k_y\Sigma_{zx}}\,,\\
\Sigma_{zz}&=\f{k_z}{k_xk_y}\p{-k_z\Sigma_{xy}+k_x\Sigma_{yz}+k_y\Sigma_{zx}}\,.
\end{align}
This, along with being traceless, shows that the shear only has two independent degrees of freedom: the polarizations.
\\\\
It is impossible to solve for $A_i$ in terms of the stress as is, hence we must fix a gauge. To do this, we fix $\nabla\cdot a=0$, whence
\begin{align}
0=k_xA_x+k_yA_y+k_zA_z\,.
\end{align}
From earlier
\begin{align}
-\Sigma_{yz}&=\mu\ps{(k_z^2-k_y^2) A_x+k_x(k_y A_y-k_z A_z)}\,,\\
-\Sigma_{zx}&=\mu\ps{(k_x^2-k_z^2) A_y+k_y(k_z A_z-k_x A_x)}\,.
\end{align}
Imposing the Gauge condition,
\begin{align}
-\Sigma_{yz}&=\mu\ps{(k_x^2-k_y^2+k_z^2) A_x+2k_xk_y A_y}\,,\\
-\Sigma_{zx}&=\mu\ps{(k_x^2-k_y^2-k_z^2) A_y-2k_x k_y A_x}\,.
\end{align}
Inverting:
\begin{align}
A_z=\f{k_x\Sigma_{yz}-k_y\Sigma_{zx}}{\mu k_z k^2}
\end{align}
Hence, by symmetry
\begin{align}
A_x&=\f{k_y\Sigma_{zx}-k_z\Sigma_{xy}}{\mu k_x k^2}\,,\\
A_y&=\f{k_z\Sigma_{xy}-k_x\Sigma_{yz}}{\mu k_y k^2}\,,\\
A_z&=\f{k_x\Sigma_{yz}-k_y\Sigma_{zx}}{\mu k_z k^2}\,.
\end{align}
\\\\
Now we take the curl of $A$ to find the shear mode displacement
\begin{align}
U_x&=\f{k_z^2(k_x\Sigma_{yz}-k_z\Sigma_{xy})+k_y^2(k_x\Sigma_{yz}-k_y\Sigma_{zx})}{\mu k_y k_z k^2}\,,\\
U_y&=\f{k_x^2(k_y\Sigma_{zx}-k_x\Sigma_{yz})+k_z^2(k_y\Sigma_{zx}-k_z\Sigma_{xy})}{\mu k_z k_x k^2}\,,\\
U_z&=\f{k_y^2(k_z\Sigma_{xy}-k_y\Sigma_{zx})+k_x^2(k_z\Sigma_{xy}-k_x\Sigma_{yz})}{\mu k_x k_y k^2}\,,
\end{align}
Assuming radial symmetry, we have $\Sigma=\Sigma_{zz}=-2\Sigma_{xx}=-2\Sigma_{yy}$.

The energy of these shear waves is given
\begin{align}
E&=\f12\int_V\p{\rho\abs{u_t}^2+\mu\abs{\nabla\cdot u}^2}\,,\\
&=\rho\int_V\abs{u_t}^2\,,\\
&=\mu\int_V\int\f{\diff^3 k}{(2\pi)^3}\int\f{\diff^3 k'}{(2\pi)^3}\p{k k'U(\vec k)\bar U(\vec k')}e^{-\I(\vec k\cdot\vec x-i\beta k t)}e^{\I(\vec k'\cdot\vec x-i\beta k' t)}\,,\\
&=\mu\int\f{\diff^3 k}{(2\pi)^3}\p{k^2\abs{U(\vec k)}^2}\,,\\
\end{align}
The integral is horribly divergent about the azimuth - there must have been a mistake...
\\\\
For shocks, check the laser physics book.


\pagebreak
\section{Comparison to ``Passage of Small BH...''}
Their considerations in the acoustic limit account for a body force generated by the gravitational pull of the miniature black hole. They discover that only supersonic blackholes would generate this type of radiation. Our calculations, on the other hand, seem independent of these considerations. In our case, the gravitational pull of something as dense as a Macro is not negligible, but should not generate as significant an effect this way. The principal mechanism for energy deposition is the scattering of matter off the macro. However, their methods do bring into question whether the free wave equation is appropriate. They also consider the transfer of energy due to pressure from the black hole to matter along its path. This is a different calculation from ours since black hole radiation is very weak, as opposed to collisions between baryonic matter.
\\\\
However, one key point is that they do neglect shear by using a scalar displacement potential $\psi$ below their equation (3). Moreover, they neglect the subtlety of the behavior of the collision near the boundary of the Earth.

\pagebreak
\section{Our Argument}
As a Macro passes through the earth, there are several phases to its journey. The first is its passage through the atmosphere. The second is its crossing the boundary from air to earth, which we suspect will be relatively uneventful, since the air is 1000$\times$ lower density than earth, and thus the air pressure would displace relatively little of earth. The third is the transit of the macro through the earth. The shockwave it produces should be generated primarily by elastic and inelastic scattering of baryonic matter off the Macro. If we estimate that the macro produces an initial overpressure in a tube of its own radius, we can calculate the initial overpressure from the total energy deposition and its cross section. Given that the macro is dense enough, its loss of momentum will be approximately linear, and we can approximate its energy loss as being approximately constant. Estimates have already been made of the stopping power, so the total energy deposition will just be
\begin{align}
\abs{\td{E}{x}{}}h=E_{\te{total}}=\f{\lambda+2\mu}{K^2}4\pi^2p_0^2\sigma_Xh
\end{align}
whence the initial overpressure is
\begin{align}
p_0=\f{K}{2\pi}\sqrt{\abs{\td{E}{x}{}}\f{1}{\sigma_X(\lambda+2\mu)}}
\end{align}
It is not appropriate to use the linear regime until the overpressure is close to 0 compared to some pressure, which we suspect may be $p_{\te{non-linear}}\approx60\te{GPa}$. It is reasonable to assume that the elastic limit holds when the pressures are an order of magnitude smaller than $p_{\te{non-linear}}$, denoted $p_{\te{linear}}\approx6\te{GPa}$. (This stuff needs to be checked). Note that energy losses due to heating, i.e. anelastic frequency attenuation, will dilute the energy density of the shock. However, for a conservative  estimate, we only consider the geometric attenuation, which will dilute the energy density as $r^{-1}$, where $r$ is the radial distance from the Macro's trajectory to the shock front. The shape of the shockwave will rapidly achieve a right-triangular profile. Geometric considerations show that the overpressure will go as $r^{-1}$. Thus, suppose that the peak pressure in the shock is $\bar p$. The overpressure is then given by
\begin{align}
p=\bar p\f{r-r_0}{\Delta r}\ps{\theta(r_0+\Delta r-r)-\theta(r+\Delta r)}\ps{\theta(z+h/2)-\theta(z-h/2)}
\end{align}
where $\Delta r$ is the radial length of the shock and $r_0$ is its base. In the limit where $h$ is much larger than all other fixed lengths, we have (after a large but easy mathematica calculation (triangle pressure.nb))
\begin{align}
E_{\te{propagated}}&\approx \f{\lambda+2\mu}{K^2}\f{h \bar p^2\pi^2r_0^6}{8\Delta r^2\lambda_0^2}\,.
\end{align}
Generously, we assume that the energy of the wave stayed constant through its non-linear evolution, and calculate the fraction energy deposited into the low frequency modes
\begin{align}
\Xi&=\f{h \bar p^2\pi^2r_0^6}{8\Delta r^2\lambda_0^2}\f{1}{4\pi^2p_0^2\sigma_Xh}\,,\\
&=\f{1}{32}\f{\bar p^2}{p_0^2}\f{r_0^6}{\Delta r^2\sigma_X\lambda_0^2}\,,\\
&=\f{\pi^2(\lambda+2\mu)}{8K^2}\f{\bar p^2}{\abs{\td{E}{x}{}}}\f{r_0^6}{\Delta r^2\lambda_0^2}\,.
\end{align}
$\bar p$ can be taken to be the maximum stress such that the elastic limit holds i.e. the elastic limit. Some reading suggests that this is about 13 GPa. $K$ and $\lambda+2\mu$ can be taken to be on the order of $10^{10}$ GPa (obviously we should look into this). $\Delta r$ should be very small, on the order of the radius of the macro. $\lambda_0$ is about $6$ to $10$ kilometers, and is the wavelength at which modes propagate without significant attenuation. $r_0$ can be determined by geometric means. A first guess yields
\begin{align}
\Xi&\approx\f{K^4}{(\lambda+2\mu)^2}\abs{\td{E}{x}{}}^2\f{1}{512\bar p^4\lambda_0^2\sigma_X}\,,\\
&\approx10^{-18}
\end{align}
This is within one order of magnitude of the answer gotten from (24). Note, my estimate for $r_0$ turned out to be about 1 kilometer. I'm still not entirely sure of the correct way to calculate $r_0$. If we take $\bar p\approx 10^{8}$ Pa, then $\Xi\approx 10^{-10}$. That said, it's 2:00 am and I'm doing these estimates in my head, so don't take any of this too seriously. The only really important thing is the idea that we can ignore the non-linear evolution since we can reasonably guess its end state and assume no energy loss in between.

%Setting this (less than or) equal to the earlier $E_{\te{propagated}}$ yields
%\begin{align}
%\f{h \bar p^2\pi^2r_0^6}{8\Delta r^2\lambda_0^2}&\leq (\sigma_Xp_0)^2\f{\pi h}{\lambda_0^2}\,,\\
%\bar p&\leq\p{2\sqrt{2\pi}\f{r_X^2\Delta r}{r_0^3}}p_0\,.
%\end{align}
%We know that shock waves are rapidly attenuated in the earth. Therefore, the estimate of the energy propagated by the initial overpressure serves as an upper bound to the energy propagated during the non-linear phase. Thus, this value for $\bar p$ as a function of $p_0$ serves as an upper bound to the maximum pressure in the shock. By fixing an acceptable value for $\bar p$ to begin the linear propagation, we can find suitable values of $r_0$ and $\Delta r$. Note, $\Delta r$ is quite small, and should be on the order of $r_X$, since the width of the shock will not change much from the original (half) width of the overpressure. Hence, a reasonable estimate is
%\begin{align}
%\bar p&\leq\p{2\sqrt{2\pi}\f{r_X^3}{r_0^3}}p_0\,.
%\end{align}
%Using this, we can find an upper bound for $r_0$
%\begin{align}
%r_0&\leq r_X\sqrt[3]{\p{2\sqrt{2\pi}}\f{p_0}{\bar p}}\,.
%\end{align}
%It is therefore sufficient to take the linear radius to be equality in the above expression. But all this is really saying is that it is sufficient to estimate the fraction of energy propagated by our original calculation using the cylindrical overpressure.
\pagebreak
\section{What we still need to do}
What we've calculated is the the pressure from the total energy deposition. We need to figure out the total energy, pressure, and shape of the shock after it becomes sufficiently acoustic. It would also be good to know the velocity of the shock. From this, we can calculate the energy deposited in the measurable range.
\\\\
It appears as though *nobody* considers the shear waves... Should we?
\\\\
The gravitational effects can be taken into account as in the Black Hole paper.
\end{document}
