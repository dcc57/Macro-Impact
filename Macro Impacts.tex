\documentclass{article}

% Packages

\usepackage{fullpage}
\usepackage{amsmath, amsthm, amsfonts, amssymb, mathtools, calrsfs, tensor, physics,tikz-cd}
\usepackage[mathscr]{euscript}
\usepackage{graphicx}
\graphicspath{ {images/} }
\usepackage{enumitem}
\setlist[description]{font=\normalfont}

% Custom Commands

\newcommand*\diff{\mathop{}\!\mathrm{d}}
\newcommand*\Diff[1]{\mathop{}\!\mathrm{d^#1}}
\newcommand*\nrml{\vartriangleleft}
\newcommand*\scr[1]{\mathscr{#1}}
\newcommand*\bb[1]{\mathbb{#1}}
\newcommand*\la{\langle}
\newcommand*\ra{\rangle}
\newcommand*\gen[1]{\langle #1 \rangle}
\newcommand*\x{\times}
\newcommand*\st{\text{ s.t. }}
\newcommand*\ord[1]{\left\vert#1\right\vert}
\newcommand*\aut{\text{Aut}}
\newcommand*\lcm{\text{lcm}}
\newcommand*\mcal{\mathcal}
\newcommand*\es{\emptyset}
\newcommand*\im{\text{ Im }}
\newcommand*\N{\mathbb N}
\newcommand*\Z{\mathbb Z}
\newcommand*\R{\mathbb R}
\newcommand*\Q{\mathbb Q}
\newcommand*\C{\mathbb C}
\newcommand*\te[1]{\text{#1}}
\newcommand*\en[1]{\begin{enumerate}#1\end{enumerate}}
\newcommand*\e{\varepsilon}
\newcommand*\p[1]{\left(#1\right)}
\newcommand*\ps[1]{\left[#1\right]}
\newcommand*\pc[1]{\left\{#1\right\}}
\newcommand*\f[2]{\frac{#1}{#2}}
\newcommand*\mat[2]{\left(\begin{array}{#1}#2\end{array}\right)}
\newcommand*\ocross{\otimes}
\newcommand*\I{\te{i}}
\newcommand*\pd[3]{\frac{\partial^{#3} #1}{\partial {#2}^{#3}}}
\newcommand*\td[3]{\frac{d^{#3}#1}{d #2^{#3}}}
\newcommand*\m{\te{Mat}}
\newcommand*\End{\te{End}}
\newcommand*\irr{\te{Irr}}
\newcommand*\sgn{\te{sgn}}
\newcommand*\pn[2]{\left\|#1\right\|_{#2}}
\newcommand*\esssup{\te{ess sup}}
\newcommand*\essinf{\te{ess inf}}

% Miscellaneous

\newtheorem{theorem}{Theorem}
\usetikzlibrary{matrix,arrows,decorations.pathmorphing}

% Title
\title{Homework}
\author{David Cyncynates \\ dcc57@case.edu}
\date{\today}

\begin{document}
\maketitle
\section{Scalar Modes}
In the following, lower case denotes a quantity in position space while capital letters denote their components in Fourier space.
\\\\
Denote the displacement field $u(\vec x,t)=\nabla\phi(\vec x,t)+\nabla\times a(\vec x,t)$\,. The linearized (acoustic) wave equation for $\phi$ is then
\begin{align}
\alpha^2\nabla^2\phi&=\partial_t^2\phi\,,
\end{align}
where $\alpha^2=\f{\lambda+2\mu}{\rho}$. We may write
\begin{align}
\phi(\vec x,t)=\int\f{\diff^3 k}{(2\pi)^3}\Phi(\vec k,\omega)e^{\I(\vec k\cdot\vec x-\omega t)}\,,
\end{align}
from which we obtain the dispersion relation $\alpha k=\omega$, and hence the solution
\begin{align}
\phi(\vec x,t)=\int\f{\diff^3 k}{(2\pi)^3}\Phi(\vec k,\omega)e^{\I(\vec k\cdot\vec x-\alpha k t)}\,.
\end{align}
We now impose the constraint equation
\begin{align}
\sigma_{ij}&=\delta_{ij}\lambda\nabla\cdot u+\mu(u_{i,j}+u_{j,i})
\end{align}
which, for the scalar modes becomes
\begin{align}
\sigma_{ij}&=\delta_{ij}\lambda\nabla^2\phi+2\mu\partial_i\partial_j\phi
\end{align}
and whose trace is
\begin{align}
-p=K\nabla^2\phi\,,
\end{align}
where $K=\lambda+\f23\mu$ is the bulk modulus and $p=-\f13\tr\sigma_{ij}$. We take this constraint as an initial condition at $t=0$. In Fourier space
\begin{align}
P(\vec k)=Kk^2\Phi(\vec K)\,,
\end{align}
from which we obtain the displacement field components
\begin{align}
U(\vec k)&=\I\f{P(\vec k)}{K}\f{\vec k}{k^2}\,,
\end{align}
and hence
\begin{align}
u(\vec x,t)=\int\f{\diff^3 k}{(2\pi)^3}\I\f{P(\vec k)}{K}\f{\vec k}{k^2}e^{\I(\vec k\cdot\vec x-\alpha k t)}\,,
\end{align}
The equation for the energy of the compressional modes is
\begin{align}
E&=\f12\int_V\diff^3x\p{\rho\abs{\partial_t u}^2+(\lambda+2\mu)\abs{\nabla\cdot u}^2}\,,\\
&=\f{\lambda+2\mu}{K^2}\int\f{\diff^3 k}{(2\pi)^3}\abs{P(\vec k)}^2
\end{align}
In the case that $P$ depends only on the frequency
\begin{align}
E&=4\pi\f{\lambda+2\mu}{K^2}\int\f{\diff\tilde\lambda}{\tilde\lambda^4}\abs{P(\tilde\lambda)}^2\,.
\end{align}
Now consider the case of a step function type pressure source - a cylinder of height $h$ and radius $r_X$
\begin{align}
p(\vec x,0)&=p_0\theta(r_X-r)\ps{\theta(z+h/2)-\theta(z-h/2)}\,,
\end{align}
whose Fourier components are
\begin{align}
P(\vec k)&=\f{4\pi r_X p_0}{\sqrt{k_x^2+k_y^2}k_z}J_1\p{\sqrt{k_x^2+k_y^2}r_X}\sin\p{\f{h}{2}k_z}\,,
\end{align}
which, in polar $k$-space is
\begin{align}
P(k,\varphi,\theta)&=\f{4\pi r_X p_0}{k^2\sin\theta\cos\theta}J_1\p{r_Xk\sin\theta}\sin\p{\f{h}{2}k\cos\theta}\,,
\end{align}
Directly integrating in Mathematica (over $\theta$ first) yields
\begin{align}
E_{\te{total}}&=\f{\lambda+2\mu}{K^2}4\pi^2p_0^2\sigma_Xh
\end{align}
where $\sigma_X=\pi r_X^2$. To calculate the energy deposition into the low frequency spectrum, we integrate $k$ from $0$ to $k_0$. Observe, we can make the following approximation
\begin{align}
P(\vec k)&\approx_{k\ll r_X}\f{2\pi r_Xp_0}{k^2\cos\theta\sin\theta}r_X k\sin\theta\sin\p{\f{h}{2}k\cos\theta}\,,\\
&=\f{2\pi r_X^2 p_0}{k\cos\theta}\sin\p{\f{h}{2}k\cos\theta}\,.
\end{align}
From this we obtain the portion of the energy relegated to the long wavelength spectrum
\begin{align}
E_{\te{propagated}}&=\f{\lambda+2\mu}{K^2}\int\f{\diff^3 k}{(2\pi)^3}\abs{\f{2\pi r_X^2 p_0}{k\cos\theta}\sin\p{\f{h}{2}k\cos\theta}}^2\,,\\
&=\ps{\f{1}{2h}(r_X^2p_0)^2\f{\lambda+2\mu}{K^2}}\ps{hk_0\cos(hk_0)+\sin(hk_0)+hk_0\p{-2+hk_0\te{Si}(hk_0)}}\,,\\
&=\ps{(r_X^2p_0)^2\f{\lambda+2\mu}{K^2}}\ps{\f{\pi}{\lambda_0}\cos\p{\f{2\pi h}{\lambda_0}}+\sin\p{\f{2\pi h}{\lambda_0}}+\p{\f{\pi}{\lambda_0}}\p{-2+\p{\f{2\pi h}{\lambda_0}}\te{Si}\p{\f{2\pi h}{\lambda_0}}}}\,,\\
&\approx_{h\gg\lambda_0\gg1}\ps{(r_X^2p_0)^2\f{\lambda+2\mu}{K^2}}\ps{\f{\pi^3 h}{\lambda_0^2}}\,,\\
&=\ps{(\sigma_Xp_0)^2\f{\lambda+2\mu}{K^2}}\ps{\f{\pi h}{\lambda_0^2}}\,.
\end{align}
From this we obtain the fractional energy deposition into the unattenuated wavelengths
\begin{align}
\Xi&=\f{\sigma_X}{4\pi\lambda_0^2}\,.
\end{align}
This approximation holds for $\lambda_0^2\gg\sigma_X$, which is appropriate for the case that $\lambda_0$ is on the order of kilometers and $\sigma_X$ is on the order of centimeters squared.
\pagebreak
\section{Vector Modes}
The following is the equation of motion for the vector potential $a$ without body force:
\begin{align}
\mu\nabla^2a-\rho\ddot a=\mu\nabla(\nabla\cdot a)-\mu\nabla\times(\nabla\times a)-\rho\ddot a=0\,,
\end{align}
where the leftmost term is the vector laplacian. This gives us three wave equations for the three components of $a$. The constraint equation takes the form
\begin{align}
\sigma_{ij}&=\lambda\delta_{ij}\nabla\cdot \nabla\times a+\mu(\partial_j (\nabla\times a)_i+\partial_i (\nabla\times a)_j)\,,\\
&=\mu(\partial_j (\nabla\times a)_i+\partial_i (\nabla\times a)_j)\,,\\
&=\mu\mat{ccc}{2\partial_x(\partial_ya_z-\partial_za_y)&(\partial_y^2-\partial_x^2)a_z+\partial_z(\partial_x a_x-\partial_y a_y)&(\partial_x^2-\partial_z^2)a_y+\partial_y(\partial_z a_z-\partial_x a_x)\\
&2\partial_y(\partial_z a_x-\partial_x a_z)&(\partial_z^2-\partial_y^2)a_x+\partial_x(\partial_ya_y-\partial_za_z)\\
&&2\partial_z(\partial_xa_y-\partial_ya_x)}\,.
\end{align}
We now write $a_i$ in terms of its Fourier components
\begin{align}
a_i(\vec x,t)&=\int\f{\diff^3k}{(2\pi)^3}A_i(\vec k,\omega)e^{\I(\vec k\cdot\vec x-\omega t)}\,.
\end{align}
Observe the dispersion relation $\beta k=\omega$, which holds for each component $a_i$, and that $\beta=\f\mu\rho$. Thus, the solutions to the wave equation are
\begin{align}
a_i(\vec x,t)&=\int\f{\diff^3k}{(2\pi)^3}A_i(\vec k,\omega)e^{\I(\vec k\cdot\vec x-\beta k t)}\,.
\end{align}
We now use the constraint equation to write down the initial vector potential in terms of the initial stress. Note, we will obtain three equations constraining the possible initial stresses (i.e., we will be able to eliminate three components of $\sigma$ from the equations), and three equations writing $A_i$ in terms of $\Sigma_{ij}$.
\\\\
In Fourier space we have
\begin{align}
-\Sigma_{xx}&=2k_x(k_y A_z-k_z A_y)\,,\\
-\Sigma_{yy}&=2k_y(k_z A_x-k_x A_z)\,,\\
-\Sigma_{zz}&=2k_z(k_x A_y-k_y A_x)\,,\\
-\Sigma_{xy}&=(k_y^2-k_x^2) A_z+k_z(k_x A_x-k_y A_y)\,,\\
-\Sigma_{yz}&=(k_z^2-k_y^2) A_x+k_x(k_y A_y-k_z A_z)\,,\\
-\Sigma_{zx}&=(k_x^2-k_z^2) A_y+k_y(k_z A_z-k_x A_x)\,.
\end{align}
First observe that any three of these equations together forms a singular linear system in $A_i$. Thus we cannot invert to solve for $A_i$. However, we can get some information about the $\Sigma$'s.
\\\\
Taking the sum of 31-33 yields
\begin{align}
0&=\Sigma_{xx}+\Sigma_{yy}+\Sigma_{zz}\,.
\end{align}
Observe that
\begin{align}
\f{-\Sigma_{xx}k_y}{2k_xk_z}&=\f{k_y^2}{k_z}A_z-k_yA_y\,,\\
\f{-\Sigma_{yy}k_x}{2k_yk_z}&=k_xA_x-\f{k_x^2}{k_z}A_z\,,\\
\f{-\Sigma_{xy}}{k_z}&=\ps{\f{k_y^2-k_z^2}{k_z}}A_z+k_xA_x-k_yA_y\,.
\end{align}
Taking the sum of the first two reveals that`
\begin{align}
2k_xk_y\Sigma_{xy}&=k_y^2\Sigma_{xx}+k_x^2\Sigma_{yy}\,.
\end{align}
If you want to be silly, we find
\begin{align}
0&=(\Sigma_xk_y-\Sigma_yk_x)^2\,.
\end{align}
Anyway, we then have the system of three equations
\begin{align}
2k_xk_y\Sigma_{xy}&=k_y^2\Sigma_{xx}+k_x^2\Sigma_{yy}\,,\\
2k_yk_z\Sigma_{yz}&=k_z^2\Sigma_{yy}+k_y^2\Sigma_{zz}\,,\\
2k_zk_x\Sigma_{zx}&=k_x^2\Sigma_{zz}+k_z^2\Sigma_{xx}\,.
\end{align}
which is non-singular. They yield
\begin{align}
\Sigma_{xx}&=\f{k_x}{k_yk_z}\p{k_z\Sigma_{xy}-k_x\Sigma_{yz}+k_y\Sigma_{zx}}\,,\\
\Sigma_{yy}&=\f{k_y}{k_zk_x}\p{k_z\Sigma_{xy}+k_x\Sigma_{yz}-k_y\Sigma_{zx}}\,,\\
\Sigma_{zz}&=\f{k_z}{k_xk_y}\p{-k_z\Sigma_{xy}+k_x\Sigma_{yz}+k_y\Sigma_{zx}}\,.
\end{align}
This, along with being traceless, shows that the shear only has two independent degrees of freedom: the polarizations.
\\\\
It is impossible to solve for $A_i$ in terms of the stress as is, hence we must fix a gauge. To do this, we fix $\nabla\cdot a=0$, whence
\begin{align}
0=k_xA_x+k_yA_y+k_zA_z\,.
\end{align}
From earlier, we have
\begin{align}
-\Sigma_{yy}&=2k_y(k_z A_x-k_x A_z)\,,\\
-\Sigma_{zz}&=2k_z(k_x A_y-k_y A_x)\,,
\end{align}
which becomes
\begin{align}
-\Sigma_{yy}&=2k_y\ps{-\f{k_z}{k_x} \p{k_yA_y+k_zA_z}-k_x A_z}\,,\\
-\Sigma_{zz}&=2k_z\ps{k_x A_y+\f{k_y}{k_x} \p{k_yA_y+k_zA_z}}\,,
\end{align}
which simplifies to
\begin{align}
-\Sigma_{yy}&=-\f{2k_y}{k_x}\ps{k_y k_z A_y+\p{k_z^2+k_x^2} A_z}\,,\\
-\Sigma_{zz}&=\f{2k_z}{k_x}\ps{\p{k_x^2 + k_y^2}A_y+k_yk_zA_z}\,,
\end{align}
Hence
\begin{align}
A_x&=\f{(k_y^2-k_z^2)(\Sigma_{yy}-\Sigma_{zz})+k_x^2(\Sigma_{yy}+\Sigma_{zz})}{2k_x^2k^2}\,,\\
&=\f{(k_y^2-k_z^2)(\Sigma_{yy}-\Sigma_{zz})-k_x^2\Sigma_{xx}}{2k_x^2k^2}\,,\\
&=\f{(k_y^2-k_z^2)}{k_x^2}\f{(\Sigma_{yy}-\Sigma_{zz})}{2k^2}-\f{\Sigma_{xx}}{2k^2}\,,\\
\end{align}
By symmetry, we have
\begin{align}
A_x&=\f{(k_y^2-k_z^2)}{k_x^2}\f{(\Sigma_{yy}-\Sigma_{zz})}{2k^2}-\f{\Sigma_{xx}}{2k^2}\,,\\
A_y&=\f{(k_z^2-k_x^2)}{k_y^2}\f{(\Sigma_{zz}-\Sigma_{xx})}{2k^2}-\f{\Sigma_{yy}}{2k^2}\,,\\
A_z&=\f{(k_x^2-k_y^2)}{k_z^2}\f{(\Sigma_{xx}-\Sigma_{yy})}{2k^2}-\f{\Sigma_{zz}}{2k^2}\,.
\end{align}
Observe, in the case that all stresses are equal, we obtain that $A_x=A_y=A_z=0$, as we should.
\\\\
For shocks, check the laser physics book.
\end{document}
