\documentclass[prd,reprint,10pt,tightenlines]{revtex4-1}

% Packages

\usepackage{fullpage}
\usepackage{amsmath, amsthm, amsfonts, amssymb, mathtools, calrsfs, tensor, physics,tikz-cd}
\usepackage[mathscr]{euscript}
\usepackage{graphicx}
\graphicspath{ {images/} }
\usepackage{enumitem}
\setlist[description]{font=\normalfont}

% Custom Commands

\newcommand*\diff{\mathop{}\!\mathrm{d}}
\newcommand*\Diff[1]{\mathop{}\!\mathrm{d^#1}}
\newcommand*\nrml{\vartriangleleft}
\newcommand*\scr[1]{\mathscr{#1}}
\newcommand*\bb[1]{\mathbb{#1}}
\newcommand*\la{\langle}
\newcommand*\ra{\rangle}
\newcommand*\gen[1]{\langle #1 \rangle}
\newcommand*\x{\times}
\newcommand*\st{\text{ s.t. }}
\newcommand*\ord[1]{\left\vert#1\right\vert}
\newcommand*\aut{\text{Aut}}
\newcommand*\lcm{\text{lcm}}
\newcommand*\mcal{\mathcal}
\newcommand*\es{\emptyset}
\newcommand*\im{\text{ Im }}
\newcommand*\N{\mathbb N}
\newcommand*\Z{\mathbb Z}
\newcommand*\R{\mathbb R}
\newcommand*\Q{\mathbb Q}
\newcommand*\C{\mathbb C}
\newcommand*\te[1]{\text{#1}}
\newcommand*\en[1]{\begin{enumerate}#1\end{enumerate}}
\newcommand*\e{\varepsilon}
\newcommand*\p[1]{\left(#1\right)}
\newcommand*\ps[1]{\left[#1\right]}
\newcommand*\pc[1]{\left\{#1\right\}}
\newcommand*\f[2]{\frac{#1}{#2}}
\newcommand*\mat[2]{\left(\begin{array}{#1}#2\end{array}\right)}
\newcommand*\ocross{\otimes}
\newcommand*\I{\te{i}}
\newcommand*\pd[3]{\frac{\partial^{#3} #1}{\partial {#2}^{#3}}}
\newcommand*\td[3]{\frac{d^{#3}#1}{d #2^{#3}}}
\newcommand*\m{\te{Mat}}
\newcommand*\End{\te{End}}
\newcommand*\irr{\te{Irr}}
\newcommand*\sgn{\te{sgn}}
\newcommand*\pn[2]{\left\|#1\right\|_{#2}}
\newcommand*\esssup{\te{ess sup}}
\newcommand*\essinf{\te{ess inf}}

% Miscellaneous

\newtheorem{theorem}{Theorem}
\usetikzlibrary{matrix,arrows,decorations.pathmorphing}

% Title

\begin{document}

\title{New seismological constraints on the available parameter space of macroscopic dark matter}
\author{David Cyncynates \\ dcc57@case.edu}
\date{\today}
\begin{abstract}
Recent limits have been proposed constraining the available parameter space of macroscopic dark matter (Macros). Previous limits were based on the energy deposition of Macros being akin to that of a nuclear bomb. Our estimates take into account the effect of seismometer sensitivity and energy distribution across frequencies, anelastic attenuation, geometric effects, internal anisotropy of the Earth or Moon, as well as energy loss due to melting. We produce new constraints on the Macro parameter space based on the total lunar seismic event rate.
\end{abstract}
\maketitle
\section{Introduction}

\section{The Seismic Source}
\subsection{The Melt Wave}
The section of parameter space we are interested in contains Macros whose cross sectional area is small enough that the energy lost due to the macro displacing rock is negligible compared to the total energy deposition. When this scenario holds, the energy deposited by the Macro can be treated as heating a column of cross sectional area $\sigma_X$ and length $\ell$.

We immediately see that this corresponds to a temperature distribution
\begin{align}
T(r,0) = \f{v_X^2}{c_p}\theta(r_X-r)\,.
\end{align}
It is reasonable to approximate this by a $\delta$ source of the form
\begin{align}
T(r,0) = \f{v_X^2}{c_p}r_X\delta(r)\,.
\end{align}
The diffusion equation then provides that the radius of the melt front is
\begin{align}
r_\te{melt}= \sqrt{4t\alpha\ln\ps{\f{r_X v_X^2}{2 c_p t \alpha T_\te{melt}}}}\,,
\end{align}
where $\alpha$ is the thermal diffusivity. 
\subsection{Seismic Wave Generation}
The energy from the initial impact generates a column of melted rock. The super-heated rock, held at constant volume, generates a pressure which sources the seismic waves that can be felt at detectors. Assuming the Macro's velocity is insignificantly changed by its encounter with the Moon, we should expect it to source a pressure of the form
\begin{align}
p = p_0 f(r)\chi_{(r,z)\in [a,b]\times [-\f \ell 2,\f \ell 2]}\,,
\end{align}
where $f$ is an analytic function whose range is a subset of $[-1,1]$, and $p_0$ is the maximum pressure of the source. Moreover, after a short time, the pressure should still be of this form, since the anisotropy of the Moon is significant only on very large scales.

The seismometers on the moon have a net useful range (the union of all useful ranges of all seismometers) of $0.004-20\text{\,Hz}$ and are sensitive to displacements as small as $0.3\text{\,nm}$. The energy measured by seismometers is a subset of that in the net useful range.

The Fourier transform of $p$ to leading order in $kb$ is
\begin{align}
P = \f{4\pi p_0}{k\cos\theta}\sin\p{\f{h k\cos\theta}{2}}F+\scr O(k^0)\,,
\end{align}
where $\theta,\phi$ are the momentum space polar angles, $F = b^2f_1(b)-a^2f_1(a)$ and
\begin{align}
f_1(x) = \sum_{n=0}^\infty\sum_{m=0}^n\binom{n}{m}\f{f^{(n)}(x_0)}{n!}\f{(-x_0)^{n-m}}{2+m}x^{m}\,,
\end{align}
for some $x_0\in[a,b]$. The energy of a seismic $p$-wave is
\begin{align}
E = \int\diff^3 x\p{ \rho\abs{\partial_t u}^2+(\lambda+2\mu)\abs{\nabla u}^2}\,,
\end{align}
where $u$ is the displacement field, $\rho,\lambda,\mu$ are the density and Lam\`e coefficients respectively. For $p$-waves, $p = -K\nabla\cdot u$ where $K$ is the bulk modulus. Define $\kappa = K^2/(\lambda + 2\mu)$ Fourier transforming, we find that $U = \I P\vec k/K k^2$ and hence
\begin{align}
E =\f1\kappa\int\f{\diff^3 k}{(2\pi)^3}\abs{P}^2\,.
\end{align}
Since a seismometer can only detect lower frequency modes, the detectible energy is given by 
\begin{align}
E_k = \f1\kappa\int_0^k\f{\diff k'\diff\theta}{(2\pi)^2}k'^2\sin\theta\abs{P}^2\,.
\end{align}
Substituting in our $P$, we find that
\begin{align}
\notag E_k&=\f1\kappa 2p_0^2F^2\ell k^2\\\notag&\times\p{\f{\sin(\ell k)+k\ell(\cos(\ell k)-2)}{\ell^2 k^2}+\te{Si}(\ell k)}+\scr O(k^4)\,.
\end{align}
When $\ell k>1$ it is a good approximation to take
\begin{align}
E_k&=\f1\kappa \pi\ell p_0^2F^2 k^2:=C_1 k^2\,.
\end{align}

Although the pressure initial conditions are fixed by the fast behavior of the melt wave, linear elasticity may not be appropriate for the pressures involved. As the pressure wave propagates, it will be geometrically attenuated until the pressure differential is within the linear regime. Given that the Macro travels at super-sonic speeds, it will generate a shock wave. Often these shock waves have pressure profiles resembling a right triangle i.e. $f = \p{r - r_0}/\Delta r$ where $a = r_0$ and $b = r_0 + \Delta r$. This shape is also chosen for its simplicity. In this case $f_1(x) = \p{2 x-3r_0}/6\Delta r$, and hence
\begin{align}
E_k = \f{\Delta r^2(3r_0+2\Delta r)^2}{36 \kappa} \pi\ell p_0^2k^2\,.
\end{align}
The total energy of the pressure wave is
\begin{align}
E \notag&= \f{2\pi p_0^2\ell}{\kappa}\int_a^b r\diff r f(r)^2\,,\\
&=\f{p_0^2}{6\kappa}\pi\Delta r \ell(4r_0 + 3\Delta r)\,.
\end{align}
The Macro deposits energy
\begin{align}
E_\te{initial} = \abs{\td{E}{x}{}}\ell = \rho\sigma_X v_X^2\ell\,,
\end{align}
and loses some energy due to melting. Thus, the pressure waves have initial energy
\begin{align}
E_\te{propagated} \notag&= E_\te{initial} - E_\te{melt}\,,\\
&= \p{\rho\sigma_X v_X^2 - \epsilon_\te{melt}}\ell:=\epsilon\ell\,.
\end{align}
A very unrealistic assumption is that no energy is lost during the non-linear evolution. Nonetheless, it will produce a generous over-estimate of the energy in the linear regime. We thus set $E = E_\te{propagated}$ and obtain an expression for $r_0$
\begin{align}
r_0 =  \f{3\kappa\epsilon}{2\pi p_0^2\Delta r}-\f{3\Delta r}{4}\,.
\end{align}
$r_0$ demarcates the end of the non-linear regime. As the length of the pulse tends to zero, the non-linear regime extends without bound. We require $p_0\leq 10^8\te{\,Pa}$, corresponding to an order of magnitude below the elastic limit of granite. A lower bound on $r_0$ is 0, corresponding to entirely linear behavior. This, in turn, imposes an upper bound on the pulse width
\begin{align}
\Delta r\leq\sqrt{\f{2\kappa \epsilon}{\pi p_0^2}}\,.
\end{align}
We now form $\Xi:=E_k/E_\te{propagated}$ representing the fraction of deposited energy detectable to seismometers
\begin{align}
\Xi = k^2\f{\p{\pi p_0^2\Delta r^2-18\kappa \epsilon}^2}{576 \pi p_0^2 \kappa \epsilon}\,.
\end{align}
When $\Delta r$ is restricted to its physical range, $\Xi$ is a monotonic decreasing function in $\Delta r$. Thus
\begin{align}
\f{4}{9}\f{\kappa\epsilon k^2}{\pi  p_0^2} \leq \Xi <\f{9}{16}\f{\kappa\epsilon k^2}{\pi  p_0^2}\,.
\end{align}
It is important to remember that these expressions only hold for $k(r_0+\Delta r)\ll 1$, however they will always provide an over-estimate of the fraction of detectable energy. Moreover, this condition holds for a wide variety of reasonable parameters.

\section{Seismic Wave Propagation}
The velocity of $p$-waves as a function of radius within the Earth and Moon (and presumably other spherical rocky celestial bodies) is of the form $v = a^2 - b^2 r^2$. Using Snell's law, we obtain the differential equations for the trajectory of $p$-wave rays within such bodies
\begin{align}
v^2 \notag&= \dot r^2 + \f{p_\te{ray}^2 v^4}{r^2}\,,\\
0 &=\dot\theta\pm\f{p_\te{ray} v^2}{r^2}\,.
\end{align}
where $r$ is the radial coordinate of the ray, $\theta$ is the polar angle of the ray measured from the center of the body, and $p_\te{ray}$ is the ray parameter, whose value is fixed along a ray trajectory. These equations can be integrated. In the case that $v = a^2 - b^2 r^2$, we obtain for some constants $q$ and $\theta_0$.
\begin{align}
\notag&r(t) = \f{a\sqrt{(q^2e^{2abt}-b)^2 + 4a^2b^4 p_\te{ray}}}{b\sqrt{(qe^{2abt}+b)^2 + 4a^2b^4 p_\te{ray}}}\,,\\
&\tan(\theta(t) - \theta_0) = \f{qe^{4abt}-b^2}{4ab^3p_\te{ray}}+abp_\te{ray}\,.
\end{align}
\textit{A priori} these are geodesics on the Poincar\'e disc of radius $\f ab$ based on the form of $v$. 

The Earth and Moon are stratified. At each boundary, a ray will be reflected and transmitted. We assume that the reflection and transmission coefficients are frequency independent for simplicity. 

The last effect to account for is anelastic attenuation, which does depend on frequency. Anelastic attenuation is characterized by the quality factor $Q$, which is, in general, dependent on $r$. For a given mode, the ratio of final to initial amplitude is
\begin{align}
\exp\ps{-k \int_{t_0}^t\diff t'\f{v(r(t))}{2Q(r(t))}}:=\exp\ps{-k \te{Att}}\,.
\end{align}
The VPREMOON model provides piecewise constant data for $Q$, so it is reasonable to subdivide the moon further into strata of different $Q$. For propagation within a given layer $i$, the factor $\te{Att}$ is given by $\Delta t_i v_i/Q_i$, where we take $v_i$ to be the average velocity within a given strata. For the case of the Earth and Moon, this is a good approximation since $v$ doesn't change much within a given layer of constant $Q$. Thus, the amplitude of a ray can be computed by knowing the two numbers
\begin{align}
\te{Ref} \notag&= \prod_i\te{Ref}_i\,,\\
\te{Att} &=\sum_i\f{\Delta t_i v_i}{Q_i}\,,
\end{align}
where $\te{Ref}_i$ are the reflection or transmission coefficients of boundary $i$ on which the ray is incident.

\section{Seismic Wave Detection}
The seismometers on the Moon are sensitive to displacements as small as $0.3\te{\,nm}$. We calculate the displacement due to each ray, and describe how they add. Consider a $p$-wave traveling towards positive $x$ and of compact support $S$ in the plane normal to to its motion. Denote $A =\int_S\diff y\diff z$ The displacement field is
\begin{align}
u(\vec x,t) &= \chi_{\R\times S}\int\f{\diff k}{2\pi}U(k)e^{-\I k(x-v_p t)}\,,\\
\abs{u(\vec x,t)}&\leq\int\f{\diff k}{2\pi}\abs{U(k)}\,,
\end{align}
and the energy is
\begin{align}
E = \rho v_p^2 A\int\f{\diff k}{2\pi}k^2\abs{U(k)}^2\,.
\end{align}
As before, we denote
\begin{align}
E_k = \rho v_p^2 A\int_0^k\f{\diff k'}{2\pi}k'^2\abs{U(k')}^2\,.
\end{align}
It follows that
\begin{align}
\abs{U(k)} = \sqrt{\f{2\pi}{\rho v_p^2 A}}\f1k\sqrt{\td{E_k}{k}{}}\,,
\end{align}
and from the estimate above
\begin{align}
\abs{u(\vec x,t)}\leq(2\pi\rho v_p^2 A)^{-1/2}\int_0^k\f{\diff k'}{k'}\sqrt{\td{E_{k'}}{k'}{}}\,.
\end{align}
In our case $\Xi E= C_1 k^2$ before attenuation. After anelastic attenuation, $E_k =\te{Ref}\,2C_1 k e^{-k\,\te{Att}}$
\begin{align}
\abs{u(\vec x,t)}\leq\sqrt{\f{2}{\rho v_p^2 A}}\sqrt{\f{\te{Ref}\,\Xi E}{\te{Att}}}\f1k\te{Erf}\ps{\sqrt{\f{\te{Att}\,k}{2}}}\,.
\end{align}
\section{Simulation}
The simulation is broken down into two natural parts: data generation and data analysis. The data generation consists of propagating a large number of rays to approximate a line source of $p$-waves and tracking the positions and energies at which they contact the surface of the moon. The data analysis makes use of HEALPix to approximate the energy density on the sphere and then makes use of section IV to measure displacements.
\subsection{Data Generation}
To generate the data, we consider a Macro trajectory which passes a distance $D$ from the center of the moon. This trajectory has a length $L$. Along this trajectory, we consider a set of $M$ points, from each of which we shoot $N$ rays, each endowed with $E_m = \rho(r_{m})\sigma_Xv_X^2 L/(MN)$ energy, where $m\in\{1,\dots,M\}$ labels the point. Because of the spherical symmetry, each ray is confined to a subspace of two dimensions. We use the trajectories in section III to propagate the rays. Since the moon is stratified, we must halt the propagation at the boundary of each layer and reflect/transmit rays. During a ray's propagation through a given stratum $\Delta t_i$ is recorded and added to $\Delta t$, $\te{ATT}_i$ is recorded and added to $\te{ATT}$, and $\te{REF}_i$ is recorded at the boundary and is multiplied by $\te{REF}$. Each time a ray makes contact with the surface of the Moon, its position, $\te{ATT}$, $\te{REF}$, $\Delta t$, and $E_m$ are recorded. The final data is a table of these values corresponding to each ray intersection with the surface. We iterate 16 times since it takes 16 iterations for a ray to pass entirely through the VPREMOON model stratification.
\subsection{Data Analysis}
We now convert the position data to HEALPix coordinates for a given $n_\te{side}$. Define $A_\te{pix} = 4\pi R_\te{moon}^2/12\times 2^{n_{\te{side}}}$, the area of a HEALPix pixel. The time it takes a ray to cross a HEALPix pixel can be estimated $t_\te{crossing}^2 = A_\te{pix}/\pi v_p^2$. Rays that contact a pixel within $t_\te{crossing}$ of one another are considered to add constructively. Since the integral in section IV is unreasonable to do for more than one ray, we make use of Minkowski's inequality and obtain an upper bound on displacement by simply adding the displacements from each ray. We suppose that each ray distributes its energy evenly over the area of the HEALPix pixel, and thus set $A = A_\te{pix}$. This is a good approximation when $N$ is the same as the number of HEALPix pixels. 

Using (29) we compute an upper bound for the displacement within a given HEALPix pixel, and compare that to the sensitivity of the lunar seismometers. If the displacement is less than the sensitivity then the pixel is marked undetectable and is otherwise marked detectable. The number of detectable pixels is divided by the total number of pixels to obtain the fraction $\scr F_D$ of the lunar surface which could plausibly register a signal. 

This procedure is repeated for 17 values of $D$ spaced evenly by $100\,\te{km}$. The average fraction of detectable surface area is then
\begin{align}
\scr F = \sum_D\f{\scr F_D D}{D}\,.
\end{align}
If there are $n$ lunar seismometers, suppose each occupies its own HEALPix pixel. Then $\scr N = n/12\times 2^{n_\te{side}}$ is the fraction of the moon which could potentially measure a signal. If $\scr M$ Macros impact the moon, the likelihood of any of them being detected is
\begin{align}
\scr P = 1-\ps{(1-\scr F)(1-\scr N)}^\scr M\,.
\end{align}

If the energy of a ray from every point along the line hitting a single pixel (with no reflection, refraction, or attenuation) is less than the displacement threshold, there is no chance of detection and no need to run the simulation. (If two rays from the same point on the line source hit the same pixel as $n_\te{side}\to\infty$, then they must be the same ray).
\begin{align}
u\leq\sqrt{\f{\Xi E}{\pi^2\rho k v_p^2 R_\te{Moon}^2}}
\end{align}
This is independent of the choice of partition. For the triangle wave, we need not test $\sigma_X\leq10^{-9}\te{m}^2$ (I WILL VERIFY THIS).
\section{Results}

\section{Conclusion}
\end{document}
