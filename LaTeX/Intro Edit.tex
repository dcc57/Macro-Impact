Dark matter constitutes a large fraction of the Universe today. However, no theory about its composition has yet be theoretically confirmed. Some more prominent candidates are particles which exist outside the Standard Model, such as weakly interacting massive particles (WIMPS) and axions. However, plausible candidates for dark matter still exist inside the current theoretical paradigm. These candidates are dense clumps of matter, including strange quark nuggets (SQNs), small black holes, and so on. In this paper, we consider model independent seismological constraints on the available parameter space of macroscopic dark matter candidates. The theoretical motivation from this stems from the work of Witten (1984), in which

macroscopic clumps of strange matter could be stable and make up the amount

of dark matter needed.

Previous work on this topic has focused on both the moon and the Earth.

Here we consider lunar seismometer limits before returning to the Earth in fu-
ture work. Teplitz et al(2005) have placed some limits on a fraction of the

parameter space of macros by considering the average total seismic energy re-
leased and the average number of seismic events on the moon. These data

were obtained from the Apollo lunar seismometers in the 70?s. In this paper,

we revise these limits based on a computation of the seismic efficiency, which

is the coupling constant representing the fraction of energy deposited by our

macro on its passage through the earth that is available for seismic waves. The

mechanism for this is anelastic damping, which is the preferential damping of

higher frequency components of a signal on passage through an object. Note

that this work also relies on a delta function distribution of the dark matter

nuggets. However, one should expect some extended mass distribution of the

dark matter nuggets.

In addition to that, we also seek to calculate the plasma zone, vapor zone and

melt zone, which are the zones of rock that are converted into plasma, vaporized

and melted respectively due to the extreme temperature of the rapidly expand-
ing shockwave formed by the energy deposited due to the macro. This further

reduces the seismic efficiency. Lastly, we ran computer simulations to determine

if the stratified structure of the moon would focus the energy such that the sig-
nal received could be magnified in amplitude. Prior works neglected reflection

and refraction in modelling the passage of the seismic waves. (However, these

focusing effects proved to be of minimal significance, with the reduction in the

seismic efficiency being orders of magnitude more.)