\documentclass[10pt,a4paper]{article}
\usepackage[utf8]{inputenc}
\usepackage{amsmath}
\usepackage{amsfonts}
\usepackage{amssymb}
\author{Jagjit}
\title{MACros2}
\begin{document}
\maketitle{Introduction}\par
	Dark matter constitutes a large fraction of the Universe today. However, we are still unclear as to what exactly makes up this constituent. Some more prominent candidates are particles such as WIMPS and axions. In this paper, we consider a class of macroscopic dark matter candidates, which would yield detectable seismological signals if they were to pass through the Earth. The theoretical motivation from this stems from the work of Witten (1984), in which macroscopic clumps of strange matter could be stable and make up the amount of dark matter needed.\par
	 Previous work on this topic has focused on both the moon and the Earth. Here we consider lunar seismometer limits before returning to the Earth in future work. Teplitz et al(2005) have placed some limits on a fraction of the parameter space of macros by considering the average total seismic energy released and the average number of seismic events on the moon. These data were obtained from the Apollo lunar seismometers in the 70's. In this paper, we revise these limits based on a computation of the seismic efficiency, which is the coupling constant representing the fraction of energy deposited by our macro on its passage through the earth that is available for seismic waves. The mechanism for this is anelastic damping, which is the preferential damping of higher frequency components of a signal on passage through an object. Note that this work also relies on a delta function distribution of the dark matter nuggets. However, one should expect some extended mass distribution of the dark matter nuggets.\par
	In addition to that, we also seek to calculate the plasma zone, vapor zone and melt zone, which are the zones of rock that are converted into plasma, vaporized and melted respectively due to the extreme temperature of the rapidly expanding shockwave formed by the energy deposited due to the macro. This further reduces the seismic efficiency. Lastly, we ran computer simulations to determine if the stratified structure of the moon would focus the energy such that the signal received could be magnified in amplitude. Prior works neglected reflection and refraction in modelling the passage of the seismic waves. (However, these focusing effects proved to be of minimal significance, with the reduction in the seismic efficiency being orders of magnitude more.)\par
	
\end{document}