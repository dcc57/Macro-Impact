\documentclass{article}

% Packages

\usepackage{fullpage}
\usepackage{amsmath, amsthm, amsfonts, amssymb, mathtools, calrsfs, tensor, physics,tikz-cd}
\usepackage[mathscr]{euscript}
\usepackage{graphicx}
\graphicspath{ {images/} }
\usepackage{enumitem}
\setlist[description]{font=\normalfont}

% Custom Commands

\newcommand*\diff{\mathop{}\!\mathrm{d}}
\newcommand*\Diff[1]{\mathop{}\!\mathrm{d^#1}}
\newcommand*\nrml{\vartriangleleft}
\newcommand*\scr[1]{\mathscr{#1}}
\newcommand*\bb[1]{\mathbb{#1}}
\newcommand*\la{\langle}
\newcommand*\ra{\rangle}
\newcommand*\gen[1]{\langle #1 \rangle}
\newcommand*\x{\times}
\newcommand*\st{\text{ s.t. }}
\newcommand*\ord[1]{\left\vert#1\right\vert}
\newcommand*\aut{\text{Aut}}
\newcommand*\lcm{\text{lcm}}
\newcommand*\mcal{\mathcal}
\newcommand*\es{\emptyset}
\newcommand*\im{\text{ Im }}
\newcommand*\N{\mathbb N}
\newcommand*\Z{\mathbb Z}
\newcommand*\R{\mathbb R}
\newcommand*\Q{\mathbb Q}
\newcommand*\C{\mathbb C}
\newcommand*\te[1]{\text{#1}}
\newcommand*\en[1]{\begin{enumerate}#1\end{enumerate}}
\newcommand*\e{\varepsilon}
\newcommand*\p[1]{\left(#1\right)}
\newcommand*\ps[1]{\left[#1\right]}
\newcommand*\pc[1]{\left\{#1\right\}}
\newcommand*\f[2]{\frac{#1}{#2}}
\newcommand*\mat[2]{\left(\begin{array}{#1}#2\end{array}\right)}
\newcommand*\ocross{\otimes}
\newcommand*\I{\te{i}}
\newcommand*\pd[3]{\frac{\partial^{#3} #1}{\partial {#2}^{#3}}}
\newcommand*\td[3]{\frac{d^{#3}#1}{d #2^{#3}}}
\newcommand*\m{\te{Mat}}
\newcommand*\End{\te{End}}
\newcommand*\irr{\te{Irr}}
\newcommand*\sgn{\te{sgn}}
\newcommand*\pn[2]{\left\|#1\right\|_{#2}}
\newcommand*\esssup{\te{ess sup}}
\newcommand*\essinf{\te{ess inf}}

% Miscellaneous

\newtheorem{theorem}{Theorem}
\usetikzlibrary{matrix,arrows,decorations.pathmorphing}

% Title
\title{Macro Impacts}
\date{\today}

\begin{document}
\maketitle
\section{Scalar Modes}
In the following, lower case denotes a quantity in position space while capital letters denote their components in Fourier space.
\\\\
Denote the displacement field $u(\vec x,t)=\nabla\phi(\vec x,t)+\nabla\times a(\vec x,t)$\,. The linearized (acoustic) wave equation for $\phi$ is then
\begin{align}
\alpha^2\nabla^2\phi&=\partial_t^2\phi\,,
\end{align}
where $\alpha^2=\f{\lambda+2\mu}{\rho}$. We may write
\begin{align}
\phi(\vec x,t)=\int\f{\diff^3 k}{(2\pi)^3}\Phi(\vec k,\omega)e^{\I(\vec k\cdot\vec x-\omega t)}\,,
\end{align}
from which we obtain the dispersion relation $\alpha k=\omega$, and hence the solution
\begin{align}
\phi(\vec x,t)=\int\f{\diff^3 k}{(2\pi)^3}\Phi(\vec k,\omega)e^{\I(\vec k\cdot\vec x-\alpha k t)}\,.
\end{align}
We now impose the constraint equation
\begin{align}
\sigma_{ij}&=\delta_{ij}\lambda\nabla\cdot u+\mu(u_{i,j}+u_{j,i})
\end{align}
which, for the scalar modes becomes
\begin{align}
\sigma_{ij}&=\delta_{ij}\lambda\nabla^2\phi+2\mu\partial_i\partial_j\phi
\end{align}
and whose trace is
\begin{align}
-p=K\nabla^2\phi=K\nabla\cdot u\,,
\end{align}
where $K=\lambda+\f23\mu$ is the bulk modulus and $p=-\f13\tr\sigma_{ij}$. We take this constraint as an initial condition at $t=0$. In Fourier space
\begin{align}
P(\vec k)=Kk^2\Phi(\vec K)\,,
\end{align}
from which we obtain the displacement field components
\begin{align}
U(\vec k)&=\I\f{P(\vec k)}{K}\f{\vec k}{k^2}\,,
\end{align}
and hence
\begin{align}
u(\vec x,t)=\int\f{\diff^3 k}{(2\pi)^3}\I\f{P(\vec k)}{K}\f{\vec k}{k^2}e^{\I(\vec k\cdot\vec x-\alpha k t)}\,,
\end{align}
The equation for the energy of the compressional modes is
\begin{align}
E&=\f12\int_V\diff^3x\p{\rho\abs{\partial_t u}^2+(\lambda+2\mu)\abs{\nabla\cdot u}^2}\,,\\
&=\f{\lambda+2\mu}{K^2}\int\f{\diff^3 k}{(2\pi)^3}\abs{P(\vec k)}^2
\end{align}
In the case that $P$ depends only on the frequency
\begin{align}
E&=4\pi\f{\lambda+2\mu}{K^2}\int\f{\diff\tilde\lambda}{\tilde\lambda^4}\abs{P(\tilde\lambda)}^2\,.
\end{align}
\\\\
Observe, the total energy can be calculated directly from $p$ as it is always true that half the energy is spring potential and half is kinetic.
\begin{align}
E&=\f{\lambda+2\mu}{K^2}\int_V\diff^3 x (p^2)
\end{align}
Now consider the case of a step function type pressure source - a cylinder of height $h$ and radius $r_X$
\begin{align}
p(\vec x,0)&=p_0\theta(r_X-r)\ps{\theta(z+h/2)-\theta(z-h/2)}\,,
\end{align}
whose Fourier components are
\begin{align}
P(\vec k)&=\f{4\pi r_X p_0}{\sqrt{k_x^2+k_y^2}k_z}J_1\p{\sqrt{k_x^2+k_y^2}r_X}\sin\p{\f{h}{2}k_z}\,,
\end{align}
which, in polar $k$-space is
\begin{align}
P(k,\varphi,\theta)&=\f{4\pi r_X p_0}{k^2\sin\theta\cos\theta}J_1\p{r_Xk\sin\theta}\sin\p{\f{h}{2}k\cos\theta}\,,
\end{align}
The total energy is clearly
\begin{align}
E_{\te{total}}&=\f{\lambda+2\mu}{K^2}p_0^2\sigma_Xh
\end{align}
where $\sigma_X=\pi r_X^2$. To calculate the energy deposition into the low frequency spectrum, we integrate $k$ from $0$ to $k_0$. Observe, we can make the following approximation
\begin{align}
P(\vec k)&\approx_{k\ll r_X}\f{2\pi r_Xp_0}{k^2\cos\theta\sin\theta}r_X k\sin\theta\sin\p{\f{h}{2}k\cos\theta}\,,\\
&=\f{2\pi r_X^2 p_0}{k\cos\theta}\sin\p{\f{h}{2}k\cos\theta}\,.
\end{align}
From this we obtain the portion of the energy relegated to the long wavelength spectrum
\begin{align}
E_{\te{propagated}}&=\f{\lambda+2\mu}{K^2}\int\f{\diff^3 k}{(2\pi)^3}\abs{\f{2\pi r_X^2 p_0}{k\cos\theta}\sin\p{\f{h}{2}k\cos\theta}}^2\,,\\
&=\ps{\f{1}{2h}(r_X^2p_0)^2\f{\lambda+2\mu}{K^2}}\ps{hk_0\cos(hk_0)+\sin(hk_0)+hk_0\p{-2+hk_0\te{Si}(hk_0)}}\,,\\
&\approx_{h\gg\lambda_0\gg1}\ps{(r_X^2p_0)^2\f{\lambda+2\mu}{K^2}}\ps{\f{\pi^3 h}{\lambda_0^2}}\,,\\
&=\ps{(\sigma_Xp_0)^2\f{\lambda+2\mu}{K^2}}\ps{\f{\pi h}{\lambda_0^2}}\,.
\end{align}
From this we obtain the fractional energy deposition into the unattenuated wavelengths
\begin{align}
\Xi&=\f{\sigma_X}{\lambda_0^2}\,.
\end{align}
This approximation holds for $\lambda_0^2\gg\sigma_X$, which is appropriate for the case that $\lambda_0$ is on the order of kilometers and $\sigma_X$ is on the order of centimeters squared.
\pagebreak
\section{A General Pressure Wave}
We now do the same calculation assuming that the pressure wave is of finite height, cylindrically symmetric, and of finite radial support. Moreover, we assume that the radial pressure distribution can be written as a taylor series. Write
\begin{align}
p=p_0f(r)\ps{\theta\p{z+\f h2}-\theta\p{z-\f h2}}
\end{align}
We may expand $f$ as
\begin{align}
f(r)&=\sum_{n=0}^\infty\f{f^{(n)}(r_0)}{n!}(r-r_0)^n\,,\\
&=\sum_{n=0}^\infty\sum_{m=0}^n(-1)^{n-m}\binom{n}{m}\f{f^{(n)}(r_0)}{n!}r^{m}r_0^{n-m}\,.
\end{align}
We now Fourier transform
\begin{align}
P(\vec k)=\f{4\pi p_0}{k_z}\sin\p{\f{h k_z}{2}}\sum_{n=0}^\infty\sum_{m=0}^n\binom{n}{m}\f{f^{(n)}(r_0)}{n!}\f{(-r_0)^{n-m}}{2+m}\ps{x^{2+m}\,_q\te{F}_p\p{\pc{1+\f m2},\pc{1,2+\f m2},-\f14 x^2 k_r^2}}_{x=a}^b\,,
\end{align}
where $_q\te{F}_p$ is the generalized hypergeometric function. We now convert to spherical coordinates in $k$-space, and write
\begin{align}
P(\vec k)=\f{4\pi p_0}{k\cos\theta}\sin\p{\f{h k\cos\theta}{2}}\sum_{n=0}^\infty\sum_{m=0}^n\binom{n}{m}\f{f^{(n)}(r_0)}{n!}\f{(-r_0)^{n-m}}{2+m}\ps{x^{2+m}\,_q\te{F}_p\p{\pc{1+\f m2},\pc{1,2+\f m2},-\f14 x^2 k^2\sin^2\theta}}_{x=a}^b\,,
\end{align}
We now expand the hypergeometric function to second order (the first order term is zero, so all that remains is the zeroth order term)
\begin{align}
P(\vec k)&=\f{4\pi p_0}{k\cos\theta}\sin\p{\f{h k\cos\theta}{2}}\sum_{n=0}^\infty\sum_{m=0}^n\binom{n}{m}\f{f^{(n)}(r_0)}{n!}\f{(-r_0)^{n-m}}{2+m}\ps{x^{2+m}}_{x=a}^b+\scr O(k)\,,\\
&=\f{4\pi p_0}{k\cos\theta}\sin\p{\f{h k\cos\theta}{2}}\ps{x^2\underbrace{\sum_{n=0}^\infty\sum_{m=0}^n\binom{n}{m}\f{f^{(n)}(r_0)}{n!}\f{(-r_0)^{n-m}}{2+m}x^{m}}_{f_1(x)}}_{x=a}^b+\scr O(k)\,,\\
&=\f{4\pi p_0}{k\cos\theta}\sin\p{\f{h k\cos\theta}{2}}(b^2f_1(b)-a^2f_1(a))+\scr O(k)\,.
\end{align}
From this, we find that the energy to leading order is
\begin{align}
E_{\te{propagated}}&=\f{\lambda+2\mu}{K^2}2p_0^2(b^2f_1(b)-a^2f_1(a))^2\p{-2k_0+k_0\cos(h k_0)+\f{\sin(h k_0)}{h}+h k_0^2\te{Si}(h k_0)}+\scr O(k_0^4)\,,\\
&\approx_{h\gg k_0}\f{\lambda+2\mu}{K^2} \f{4 p_0^2h\pi^3(b^2 f_1(b)-a^2f_1(a))^2}{\lambda_0^2}\,.
\end{align}
Let's do some checks. First consider the basic cylinder pressure where $f=1$ on $[0,r_X]$. Then $f_1=\f12$, and we see that
\begin{align}
E_{\te{propagated}}&\approx\f{\lambda+2\mu}{K^2}\f{p_0^2h\pi^3r_X^4}{\lambda_0^2}\,,\\
&=\f{\lambda+2\mu}{K^2}(p_0\sigma_X)^2\f{\pi h}{\lambda_0^2}\,,
\end{align}
which is exactly the result from before. Now consider $f(r)=\f{r-r_0}{\Delta r}$ on $[r_0,r_0+\Delta r]$. Then, taking the endpoint $r_0$ as the base of the Taylor series, $f_1(x)=\f{-r_0}{2\Delta r}+\f{x}{3\Delta r}$. Hence
\begin{align}
E_{\te{propagated}}&\approx\f{\lambda+2\mu}{K^2}\f{h p_0^2\pi^3\Delta r^2(3r_0+2\Delta r)^2}{9\lambda_0^2}
\end{align}
This is what we find out later! I'm legitimately surprised that I didn't make a mistake! Woo!
\\\\
The total energy of this general pressure wave is
\begin{align}
E_{\te{total}} = 2\pi p_0^2h\f{\lambda+2\mu}{K^2}\int_0^\infty r\diff r f(r)^2\,.
\end{align}
Hence, in general
\begin{align}
\Xi = \f{2\pi^2}{\lambda_0^2}\f{(b^2 f_1(b)-a^2f_1(a))^2}{\int_0^\infty r\diff r f(r)^2}
\end{align}
\pagebreak
\\
It is wrong, however, to assume that the wave evolves linearly close to the source. Modes will be coupled to one another, and our expression (24) will only hold in a small neighborhood of the event. Effects such as heating and rock fracturing will cause the high frequency modes in the shock to rapidly attenuate. They are, however, difficult to quantify. It is not controversial to say that the energy of the shock after its non-linear evolution (when the overpressure exceeds the elastic limit of the Earth) will be less than its initial energy. It is also well known that the behavior of shockwaves for long time tends towards a sharp wave-front with a linearly decreasing tail. To produce an over-estimate of the detectible energy, we hypothesize an approximate pressure waveform and endow it with energy equivalent to that of the initial blast. Then, we calculate the energy per mode, and sum over only those modes whose frequency is in the regime that will not rapidly attenuate.
\\\\
We describe the long-time waveform of the shock by
\begin{align}
p=\bar p\f{r-r_0}{\Delta r}\ps{\theta(r_0+\Delta r-r)-\theta(r+\Delta r)}\ps{\theta(z+h/2)-\theta(z-h/2)}\,.
\end{align}
$\Delta r$ is the length of the tail, and $r_0$ is its base. $\bar p$ is the peak pressure of the shock, and we can take it to be the stress corresponding to the elastic limit of rock. The total energy of this pressure wave is
\begin{align}
E_{\te{total}}=\f{\lambda+2\mu}{K^2}\f16 \bar p^2 \pi\Delta r h(4 r_0+3\Delta r)\,.
\end{align}
We require that this be equal to the initial total energy. From this, we can obtain an expression for $r_0$ in terms of $\Delta r$.
\begin{align}
r_0&=\f{3(2p_0^2r_X^2-\bar p^2\Delta r^2)}{4\bar p^2\Delta r}\,.
\end{align}
A lower bound for $r_0$ is clearly 0, which sets an upper bound for $\Delta r$, i.e.
\begin{align}
\Delta r&\leq \f{\sqrt{2}p_0r_X}{\bar p}\,,\\
&=\f{1}{\bar p}\sqrt{\f{2}{\pi}\f{K^2}{\lambda+2\mu}\abs{\td{E}{x}{}}}\,.
\end{align}
Plugging in some numbers (from below) yields
\begin{align}
\Delta r\leq 1.2\te{m}\,.
\end{align}
The energy in the longest modes is, for $h\gg\lambda_0\gg r_0+\Delta r$
\begin{align}
E_{\te{propagated}}&\approx\f{\lambda+2\mu}{K^2}\f{\bar p^2 \pi^3 h \Delta r^2(3r_0+2\Delta r)^2}{9\lambda_0^2}\,.
\end{align}
Then we have that the fraction of energy that will reach seismometers is
\begin{align}
\Xi&=\f{2\pi^2\Delta r(3r_0+2\Delta r)^2}{3(4 r_0+3\Delta r)\lambda_0^2}\,,\\
&=\pi^2\p{\f{18 p_0^2 r_X^2-\bar p^2\Delta r^2}{12 p_0\bar p r_X\lambda_0}}^2
\end{align}
Since we know that the total energy is (assuming that the velocity of the macro remains approximately constant throughout its journey through Earth) we can calculate the initial overpressure.
\begin{align}
\f{\lambda+2\mu}{K^2}p_0^2\sigma_Xh=E_{\te{total}}=h\abs{\td{E}{x}{}}=h\rho_{\oplus}\sigma_Xv_X^2\,.
\end{align}
Thus we obtain the following expression for $p_0$
\begin{align}
p_0&=\sqrt{\f{\rho}{\lambda+2\mu}}Kv_X\,.
\end{align}
Thus, the suppression factor is
\begin{align}
\Xi&=\pi \f{(\bar p^2\pi \Delta r^2(\lambda+2\mu)-18K^2\sigma_Xv_X^2\rho_{\oplus})^2}{144K^2(\lambda+2\mu)\rho_{\oplus}\bar p^2 \sigma_X v_X^2\lambda_0^2}\,.
\end{align}
Using the upper bound for $\Delta r$ from earlier
\begin{align}
\Xi&\geq\f{16}{9}\p{\f{K^2}{\lambda+2\mu}}\p{\f{\pi}{\bar p\lambda_0}}^2\abs{\td{E}{x}{}}\,.
\end{align}
Taking $\Delta r$ to be 0, we find
\begin{align}
\Xi&<\f{9}{4}\p{\f{K^2}{\lambda+2\mu}}\p{\f{\pi}{\bar p\lambda_0}}^2\abs{\td{E}{x}{}}\,.
\end{align}
Note the strict inequality: $\Delta r=0$ corresponds to $r_0=\infty$.
\\\\
%Reviewing a USGS document (Rock Failure and Earthquakes) suggests that $\bar p$ be set at $10^8$. To calculate $\Delta r$, notice that the Earth is initially displaced by $r_X$. The Earth will then collapse back into the hole due to ambient pressure at the speed of sound $\alpha$, yielding a time of collapse $r_X/\alpha$. The speed of the shockwave initially produced by the object is approximately the speed of the object itself. Hence $\Delta r/v_X=r_X/\alpha$. Thus $\Delta r\approx\f{v_X}{\alpha}r_X$. Thus
%\begin{align}
%\Xi&=\pi \f{(\bar p^2\pi \f{v_X^2}{\alpha^2}r_X^2(\lambda+2\mu)-18K^2\sigma_Xv_X^2\rho_{\oplus})^2}{144K^2(\lambda+2\mu)\rho_{\oplus}\bar p^2 \sigma_X v_X^2\lambda_0^2}\,,\\
%&=\p{\pi r_X v_X\f{\bar p^2 -18K^2}{12K\alpha\bar p \lambda_0}}^2\,.
%\end{align}
%Given the relative magnitudes of $\bar p$ and $K^2$,
%\begin{align}
%\Xi&=\p{\f{3\pi r_X v_XK}{2\alpha\bar p \lambda_0}}^2\,,\\
%&=\p{\f{K^2}{\lambda+2\mu}}\p{\f{3\pi}{2\bar p\lambda_0}}^2\abs{\td{E}{x}{}}\,.
%\end{align}
The following are the approximate quantities used
\begin{align}
h&=1.2\times 10^7\te{m}\,,\\
v_X&=2.5\times 10^6\te{m s}^{-1}\,,\\
\rho_{\oplus}&=6\times 10^3\te{kg m}^{-3}\,,\\
\sigma_X&=10^{-11}\te{m}^2\,,\\
\abs{\td{E}{x}{}}&=\rho_{\oplus}\sigma_X v_X^2=3.75\times 10^5\te{J m}^{-1}\,,\\
E_{\te{total}}&=4.5\times 10^{12}\te{J}\,,\\
r_X&=1.8\times 10^{-6}\te{m}\,,\\
\alpha&=6\times 10^3\te{m s}^{-1}\,,\\
\beta&=3.5\times 10^{3}\te{m s}^{-1}\,,\\
\mu&=7.2\times 10^{10}\te{Pa}\,,\\
\lambda&=7.2\times 10^{10}\te{Pa}\,,\\
K&=1.2\times 10^{11}\te{Pa}\,,\\
\bar p&=10^8\te{Pa}\,.
\end{align}
These lead to the results, (using my estimation scheme $\Delta r=\f{v_X}{\alpha}r_X$ from earlier, which may be incorrect)
\begin{align}
\Delta r&=7.4\times 10^{-4}\te{m}\,,\\
r_0&=1.6\times 10^3\te{m}\,,\\
E_{\te{propagated}}&=\f{8\times 10^{13}\te{J m}^2}{\lambda_0^2}\,,\\
\Xi&=\f{17.6\te{m}^2}{\lambda_0^2}\,.
\end{align}
Conservatively we set $\lambda_0$ at about $6$ kilometers. This yields a total suppression factor of $\Xi=4.9\times 10^{-7}$. Keep in mind that this estimate is incredibly conservative, and does not take into account any non-conservative processes in the nonlinear regime $r<r_0$. Note, even though both $\lambda_0$ and $r_0$ are on the same order of magnitude, the approximations used are still good since they are overestimates of the Bessel $J$ and Struve $H$ functions.
\\\\
Our result shows that the suppression factor only depends on the stopping factor, which in turn only depends on the velocity and cross-section of the macro. This is sensible since the velocity and cross-section of the macro are the defining characteristics of the impact. The only factor missing is the density of the macro. However, we are assuming that the density is large enough that the average velocity of the macro on its journey through the earth is close to its initial velocity.
\\\\
It is significant that $\Xi$ is proportional to $\sigma_X$. For the very largest macros, our very conservative suppression factor is near unity. However, it may still be enough to rule out detection.
\\\\
Note, it is incorrect to consider the expression for $\Xi$ in the small and large parameter limits: we required (in some approximations) that $r_0\ll \lambda_0$, which places a lower bound on $\bar p$. Moreover $\bar p$ is supposed to be less than $p_0$.
\pagebreak
\section{Vector Modes}
The following is the equation of motion for the vector potential $a$ without body force:
\begin{align}
\mu\nabla^2a-\rho\ddot a=\mu\nabla(\nabla\cdot a)-\mu\nabla\times(\nabla\times a)-\rho\ddot a=0\,,
\end{align}
where the leftmost term is the vector laplacian. This gives us three wave equations for the three components of $a$. The constraint equation takes the form
\begin{align}
\sigma_{ij}&=\lambda\delta_{ij}\nabla\cdot \nabla\times a+\mu(\partial_j (\nabla\times a)_i+\partial_i (\nabla\times a)_j)\,,\\
&=\mu(\partial_j (\nabla\times a)_i+\partial_i (\nabla\times a)_j)\,,\\
&=\mu\mat{ccc}{2\partial_x(\partial_ya_z-\partial_za_y)&(\partial_y^2-\partial_x^2)a_z+\partial_z(\partial_x a_x-\partial_y a_y)&(\partial_x^2-\partial_z^2)a_y+\partial_y(\partial_z a_z-\partial_x a_x)\\
&2\partial_y(\partial_z a_x-\partial_x a_z)&(\partial_z^2-\partial_y^2)a_x+\partial_x(\partial_ya_y-\partial_za_z)\\
&&2\partial_z(\partial_xa_y-\partial_ya_x)}\,.
\end{align}
We now write $a_i$ in terms of its Fourier components
\begin{align}
a_i(\vec x,t)&=\int\f{\diff^3k}{(2\pi)^3}A_i(\vec k,\omega)e^{\I(\vec k\cdot\vec x-\omega t)}\,.
\end{align}
Observe the dispersion relation $\beta k=\omega$, which holds for each component $a_i$, and that $\beta^2=\f\mu\rho$. Thus, the solutions to the wave equation are
\begin{align}
a_i(\vec x,t)&=\int\f{\diff^3k}{(2\pi)^3}A_i(\vec k,\omega)e^{\I(\vec k\cdot\vec x-\beta k t)}\,.
\end{align}
We now use the constraint equation to write down the initial vector potential in terms of the initial stress. Note, we will obtain three equations constraining the possible initial stresses (i.e., we will be able to eliminate three components of $\sigma$ from the equations), and three equations writing $A_i$ in terms of $\Sigma_{ij}$.
\\\\
In Fourier space we have
\begin{align}
-\Sigma_{xx}&=\mu2k_x(k_y A_z-k_z A_y)\,,\\
-\Sigma_{yy}&=\mu2k_y(k_z A_x-k_x A_z)\,,\\
-\Sigma_{zz}&=\mu2k_z(k_x A_y-k_y A_x)\,,\\
-\Sigma_{xy}&=\mu\ps{(k_y^2-k_x^2) A_z+k_z(k_x A_x-k_y A_y)}\,,\\
-\Sigma_{yz}&=\mu\ps{(k_z^2-k_y^2) A_x+k_x(k_y A_y-k_z A_z)}\,,\\
-\Sigma_{zx}&=\mu\ps{(k_x^2-k_z^2) A_y+k_y(k_z A_z-k_x A_x)}\,.
\end{align}
First observe that any three of these equations together forms a singular linear system in $A_i$. Thus we cannot invert to solve for $A_i$. However, we can get some information about the $\Sigma$'s.
\\\\
Taking the sum of 31-33 yields
\begin{align}
0&=\Sigma_{xx}+\Sigma_{yy}+\Sigma_{zz}\,.
\end{align}
Observe that
\begin{align}
\f{-\Sigma_{xx}k_y}{2\mu k_xk_z}&=\f{k_y^2}{k_z}A_z-k_yA_y\,,\\
\f{-\Sigma_{yy}k_x}{2\mu k_yk_z}&=k_xA_x-\f{k_x^2}{k_z}A_z\,,\\
\f{-\Sigma_{xy}}{\mu k_z}&=\ps{\f{k_y^2-k_z^2}{k_z}}A_z+k_xA_x-k_yA_y\,.
\end{align}
Taking the sum of the first two reveals that`
\begin{align}
2k_xk_y\Sigma_{xy}&=k_y^2\Sigma_{xx}+k_x^2\Sigma_{yy}\,.
\end{align}
If you want to be silly, we find
\begin{align}
0&=(\Sigma_{x}k_y-\Sigma_{y}k_x)^2\,.
\end{align}
Anyway, we then have the system of three equations
\begin{align}
2k_xk_y\Sigma_{xy}&=k_y^2\Sigma_{xx}+k_x^2\Sigma_{yy}\,,\\
2k_yk_z\Sigma_{yz}&=k_z^2\Sigma_{yy}+k_y^2\Sigma_{zz}\,,\\
2k_zk_x\Sigma_{zx}&=k_x^2\Sigma_{zz}+k_z^2\Sigma_{xx}\,.
\end{align}
which is non-singular. They yield
\begin{align}
\Sigma_{xx}&=\f{k_x}{k_yk_z}\p{k_z\Sigma_{xy}-k_x\Sigma_{yz}+k_y\Sigma_{zx}}\,,\\
\Sigma_{yy}&=\f{k_y}{k_zk_x}\p{k_z\Sigma_{xy}+k_x\Sigma_{yz}-k_y\Sigma_{zx}}\,,\\
\Sigma_{zz}&=\f{k_z}{k_xk_y}\p{-k_z\Sigma_{xy}+k_x\Sigma_{yz}+k_y\Sigma_{zx}}\,.
\end{align}
This, along with being traceless, shows that the shear only has two independent degrees of freedom: the polarizations.
\\\\
It is impossible to solve for $A_i$ in terms of the stress as is, hence we must fix a gauge. To do this, we fix $\nabla\cdot a=0$, whence
\begin{align}
0=k_xA_x+k_yA_y+k_zA_z\,.
\end{align}
From earlier
\begin{align}
-\Sigma_{yz}&=\mu\ps{(k_z^2-k_y^2) A_x+k_x(k_y A_y-k_z A_z)}\,,\\
-\Sigma_{zx}&=\mu\ps{(k_x^2-k_z^2) A_y+k_y(k_z A_z-k_x A_x)}\,.
\end{align}
Imposing the Gauge condition,
\begin{align}
-\Sigma_{yz}&=\mu\ps{(k_x^2-k_y^2+k_z^2) A_x+2k_xk_y A_y}\,,\\
-\Sigma_{zx}&=\mu\ps{(k_x^2-k_y^2-k_z^2) A_y-2k_x k_y A_x}\,.
\end{align}
Inverting:
\begin{align}
A_z=\f{k_x\Sigma_{yz}-k_y\Sigma_{zx}}{\mu k_z k^2}
\end{align}
Hence, by symmetry
\begin{align}
A_x&=\f{k_y\Sigma_{zx}-k_z\Sigma_{xy}}{\mu k_x k^2}\,,\\
A_y&=\f{k_z\Sigma_{xy}-k_x\Sigma_{yz}}{\mu k_y k^2}\,,\\
A_z&=\f{k_x\Sigma_{yz}-k_y\Sigma_{zx}}{\mu k_z k^2}\,.
\end{align}
\\\\
Now we take the curl of $A$ to find the shear mode displacement and use the relations (73) through (75) to obtain
\begin{align}
U_x&=-\I\f{\Sigma_{xx}}{2k_x\mu}\,,\\
U_y&=-\I\f{\Sigma_{yy}}{2k_y\mu}\,,\\
U_z&=-\I\f{\Sigma_{zz}}{2k_z\mu}\,,
\end{align}
The energy of these shear waves is given
\begin{align}
E&=\f12\int_V\p{\rho\abs{u_t}^2+\mu\abs{\nabla\cdot u}^2}\,,\\
&=\rho\int_V\abs{u_t}^2\,,\\
&=\rho\beta^2\int_V\int\f{\diff^3 k}{(2\pi)^3}\int\f{\diff^3 k'}{(2\pi)^3}\p{k k'U(\vec k)\bar U(\vec k')}e^{-\I(\vec k\cdot\vec x-i\beta k t)}e^{\I(\vec k'\cdot\vec x-i\beta k' t)}\,,\\
&=\mu\int\f{\diff^3 k}{(2\pi)^3}\p{k^2\abs{U(\vec k)}^2}\,,\\
&=\mu\int\f{\diff^3 k}{(2\pi)^3}\p{\f{\Sigma_{zz}^2\sec^2\theta+\csc^2\theta(\Sigma_{yy}^2\csc^2\phi+\Sigma_{xx}^2\sec^2\phi)}{4\mu^2}}\,.
\end{align}
Assuming axial symmetry, we have $\Sigma_{xx}=\Sigma\cos^2\phi$, $\Sigma_{yy}=\Sigma\sin^2\phi$, and $\Sigma_{zz}=-\Sigma$. Note that this choice is justified since the full stress tensor is invariant under rotations about the $z$-axis. Hence, 
\begin{align}
E&=\mu\int\f{\diff^3 k}{(2\pi)^3}\Sigma^2\p{\f{\sec^2\theta+\csc^2\theta}{4\mu^2}}\,,\\
&=\f{1}{\mu}\int\f{\diff^3 k}{(2\pi)^3}\Sigma^2\csc^2(2\theta)\,,\\
&=\f{1}{\mu}\int\f{\diff k\diff\theta}{(2\pi)^2}\Sigma^2k^2\csc^2(2\theta)\sin\theta\,,\\
&=\f{1}{4\mu}\int\f{\diff k\diff\theta}{(2\pi)^2}\Sigma^2k^2\csc(\theta)\sec^2(\theta)\,.
\end{align}
In cartesian coordinates:
\begin{align}
E&=\f{1}{4\mu}\int\f{\diff^3 k}{(2\pi)^3}\Sigma^2\f{k^4}{(k_x^2+k_y^2)k_z^2}\,.
\end{align}
Something makes me think that $\csc^2(2\theta)$ should be absorbed into the definition of $\Sigma$, but I don't know how to show this.
\pagebreak
\section{P-Wave Vector Propagation}
Observe, by Snell's law, we have that
\begin{align}
p=\f{r\sin\theta}{v}\,,
\end{align}
is constant along any fixed ray, and is given by the above form in a spherically symmetric velocity field. $\theta$ is the angle of incidence, $r$ is the distance from the origin, and $v$ is the velocity at radius $r$. Let $\sigma(s)$ be the parametrized curve followed by a p-wave. Then
\begin{align}
\cos\theta=\f{\hat r\cdot\dot\sigma(s)}{\|\sigma(s)\|}=\f{\sigma(s)\cdot\dot\sigma(s)}{\|\sigma(s)\|\|\dot\sigma(s)\|}\,.
\end{align}
From this we obtain
\begin{align}
\p{\f{pv}{r}}^2=1-\f{(\sigma\cdot\dot\sigma)^2}{(\|\sigma(s)\|\|\dot\sigma(s)\|)^2}\,,
\end{align}
If we denote the radial component as $r(s)$ and the angular component $\theta(s)$ we obtain the differential equation
\begin{align}
\p{\f{pv}{r}}^2=\f{r^2\dot \theta^2}{(\dot r^2+\dot\theta^2 r^2)}\,.
\end{align}
But observe, the velocity of the curve $\sigma$ is just the velocity in the medium, hence
\begin{align}
\p{\f{pv}{r}}^2=r^2\f{\dot \theta^2}{v^2}\,,
\end{align}
from which we obtain the equation for $\theta$,
\begin{align}
\dot \theta&=\pm\f{p v^2}{r^2}
\end{align}
We now use the velocity definition to obtain an implicit equation for $r$. Observe
\begin{align}
v^2&=\dot r^2+\dot\theta^2r^2\,,\\
&=\dot r^2+\f{p^2 v^4}{r^2}\,,
\end{align}
hence
\begin{align}
1&=\pm\f{r\dot r}{\sqrt{v^2r^2-p^2v^4}}
\end{align}
This can be integrated directly, yielding
\begin{align}
s+C&=\pm\int\f{r\diff r}{\sqrt{v^2r^2-p^2v^4}}\,.
\end{align}
In the case that $v$ is a power law, i.e.
\begin{align}
v(r)&=v_0\p{\f{r_0}{r}}^\alpha
\end{align}
we can directly integrate for integer $\alpha$, yielding
\begin{align}
s+C&=\f{r^{2\alpha}r_0^{-2\alpha}\sqrt{r^{-4\alpha}r_0^{2\alpha}v_0^2(r^{2+2\alpha}-p^2r_0^{2\alpha}v_0^2)}}{v_0^2(1+\alpha)}\,.
\end{align}
It is possible to invert this equation for any particular $\alpha$. One solution which holds for all $\alpha$ is
\begin{align}
r(s,\alpha)&=r_0^{\f{\alpha}{1+\alpha}}v_0^{\f{1}{1+\alpha}}(p^2+(C+s)^2(1+\alpha)^2)^{\f{1}{2(1+\alpha)}}\,.
\end{align}
We thus obtain the following expression for $\theta$:
\begin{align}
\theta(s,\alpha)&=\f{1}{1+\alpha}\arctan\ps{\f{1+\alpha}{p}(C+s)}+\theta_0
\end{align}
WOAH! We expect straight lines when $\alpha=0$. Let's check:
\begin{align}
r(s,0)&=v_0\sqrt{p^2+(C+s)^2}\,,\\
\theta(s,0)&=\arctan\ps{\f{C+s}{p}}+\theta_0
\end{align}
Setting $\theta_0$ to zero reveals the equation for a vertical line. Thus, adding $\theta_0$ rotates the line, and $p$ sets the initial distance from the origin.
\\\\
Now we consider the realistic velocity distribution $v(r)=a^2-b^2r^2$. Proper application of mathematica to the above formulas reveals
\begin{align}
r(s)&=\f{a\sqrt{b^2-2be^{2ab(s+C)}+e^{4ab(s+C)}+4a^2b^4p^2}}{b\sqrt{b^2+2be^{2ab(s+C)}+e^{4ab(s+C)}+4a^2b^4p^2}}\,,\\
\theta(s)&=\arctan\ps{\f{-b^2+e^{4ab(s+C)}+4a^2b^4p^2}{4ab^3p}}+\theta_0
\end{align}
These are arcs of circles which touch the boundary $r=\f ab$ at right angles. That is, the trajectory of rays is given by geodesics in the Poincare disk. This can be seen easily by inspection of the metric a modified Poincare Sphere:
\begin{align}
\diff s^2&=\f{\sum_i\diff x_i^2}{\p{a^2-b^2\sum_i x_i^2}^2}\,,\\
&=\f{\sum_i\diff x_i^2}{\p{a^2-b^2\sum_i x_i^2}^2}\,,\\
&=\f1{v^2}\sum_i\diff x_i^2
\end{align}
It is then obvious that $b=0$ corresponds to straight lines. The importance of this observation is that $b^2\approx 10^{-7}\te{s}^{-1}$ and $a^2\approx 10\te{km s}^{-1}$, so the radius of the Poincare sphere corresponding to the trajectories of $p$-waves in earth is approximately $7\times10^3-1\times 10^4\te{km}$, while the earth is of radius $10^3\te{km}$. In particular, for the different layers of the earth, the ratio radius of the layer's outer boundary and the radius of its corresponding Poincare sphere is 7.0 for the inner core, 2.0 for the outer core, and 1.8 for the mantle. Thus, it is reasonable to approximate the trajectories of $P$-waves in the inner core as straight lines, but necessary to treat those in the other layers as bending.
\\\\
It is clear that numerical methods will be necessary to get an answer. We will now consider a sequence of increasingly realistic models. I will begin typing up the first few tomorrow.

\pagebreak
\section{Numerical Methods}
From symmetry, it is clear that it is enough to understand the behavior of rays from a point source to understand the propagation of rays in the whole sphere. Let $\vec p=(r,\theta)$ denote the position vector of our ray, whose initial coordinates are $\vec p(0)=(x_0,0)$. We also insist that the tangent vector of $\vec p$ be at an angle $\alpha$ from the x axis. The parametrization in terms of these initial conditions is
\begin{align}
r(s)&=\f{a}{b}\sqrt{\f{a^2\p{e^{2abs}-1}^2+b^2\p{e^{2abs}+1}^2x_0^2+2(-1)^{s_1}ab\p{e^{4abs}-1}x_0\abs{\cos\alpha}}{a^2\p{e^{2abs}+1}^2+b^2\p{e^{2abs}-1}^2x_0^2+2(-1)^{s_1}ab\p{e^{4abs}-1}x_0\abs{\cos\alpha}}}\,,\\
\notag\theta(s)&=\arccot\ps{\f{4(-1)^{s_2}abx_0(a^2-b^2x_0^2)\sin\alpha}{4a^2b^2(1+e^{4abs})x_0^2+4(-1)^{s_1}abe^{4abs}x_0(a^2+b^2x_0^2)\cos\alpha+(e^{4abs}-1)(a^4+b^4x_0^4+2a^2b^2x_0^2\cos(2\alpha))}}-\\
&\hspace{1cm}-\arctan\ps{\f{(-1)^{s_2}(2abx_0+(-1)^{s_1}(a^2+b^2x_0^2)\cos\alpha)\csc\alpha}{a^2-b^2x_0^2}}
\end{align}
Note that $s_1$ and $s_2$ change the quadrant of the ray propagation with respect to the starting position. We now need to solve for $s$ such that $r$ is some desired value at that point, call it $r_m$ (where we have chosen $m$ to stand for mantle).
\\\\
The time $s_m$ it takes to get to the radius $r_m$ is
\begin{align}
s_m=\f{1}{2ab}\ln\ps{\f{a^4-b^4 r_m^2 x_0^2+a^2b^2(r_m^2-x_0^2)+(-1)^{s_3} 2ab\sqrt{(r_m^2-x_0^2)(a^4-b^4 r_m^2x_0^2)+(a^2-b^2r_m^2)^2x_0^2\cos^2\alpha}}{(a^2-b^2r_m^2)(a^2+b^2x_0^2+2(-1)^{s_1}abx_0\cos\alpha)}}
\end{align}
We now want to compute the sine of the angle between the tangent vector to the curve and the surface normal at $s_m$. Observe
\begin{align}
T\cdot n&=\f{\dot r}{\sqrt{\dot r^2+r^2\dot\theta^2}}\,.
\end{align}
Hence
\begin{align}
\sin\phi&=\f{r\dot\theta}{\sqrt{\dot r^2+r^2\dot\theta^2}}\,.
\end{align}
We now write a theoretical program for computing the energy density on the surface of the earth. First, define a list of $M$ points from which we want to propagate rays. In our specific case, these will be evenly spaced points along a chord through the earth. Store the coordinates of these points. From each of these points, we will want to propagate some number $N$ of rays. These rays must have uniform angular separation. This is the question of tiling the sphere by arbitrarily small polygons. This is, of course, only possible for finitely many polygons on a given sphere. Thus, we need some other way of uniformly distributing points on a sphere. {http://neilsloane.com/packings/dim3/pack.3.130.txt} contains the coordinates of 130 points placed on a sphere such that their angular separation is minimized (18.5 degrees). If this isn't enough, there is a tiling on the same website with over 1000 points, but that would be absurd. The vectors given by the 130 points determine the plane of propagation (given by the span of the direction vector and the vector pointing from the origin) and the initial propagation angle $\alpha$ with respect to the vector pointing from the origin. Once we have the set of points and rays to consider, our problem turns into many simpler problems, i.e. rays propagating in two dimensions. There are four cases to consider.
\begin{enumerate}
\item $r(0)>r_m$
\begin{enumerate}
\item $s_m\in\C\setminus\R$. For these, it is enough to know $\theta(s_E)$, where $s_E$ is the time at which the rays reach the surface. In the case that we want to consider surface reflection, we should store the angle of incidence, but that is likely unnecessary for our purposes.
\item $s_m\in\R$. These rays will reflect off the core. The angle of reflection with respect to the surface normal is simply minus the angle of incidence. Store these data and re-propagate ray after all first-iteration propagation is done.
\end{enumerate}
\item $r(0)<r_m$
\begin{enumerate}
\item If the angle of incidence is less than the critical angle, then the ray refracts. Store these data and re-propagate after all first iteration propagation is done.
\item If the angle of incidence is greater than the critical angle, the rays will internally reflect forever, so we can throw these out.
\end{enumerate}
\end{enumerate}
The first round of propagation will produce some rays that hit the surface and some which need to be propagated a second time. All those rays propagating in the second round of propagation will originate on the core-mantle boundary, and thus computation will be greatly simplified. Once we have stored the final polar coordinate, we must place these final coordinates properly on the sphere. This is done by two rotations. The first is the inverse of the rotation which took the initial propagation direction vector into the $xy$-plane. The second is the inverse of the rotation which took the position vector of the point source to the $x$-axis. I believe rotations of points is commutative, but I'm too lazy to check, so it's safest to just do these in the proper order.
\pagebreak
\section{The Procedure Made More Explicit}
\begin{enumerate}
\item We wish to consider a line source. First, determine the number $N$ of point sources with which we want to approximate the line.
\item Next we determine the coordinates of these points. If the Earth has radius $R_E$, and the line is a distance $D$ from the center, then the length of the line intersecting the earth is $L=2\sqrt{R_E^2-D^2}$. we must space the points evenly, so each point will be $\f{L}{N}$ apart from one another. To maintain symmetry, we insist that $N$ be odd, and that one point exist in the plane of symmetry. The coordinate of the center point is $(x,y,z)=(0,D,0)$. The set of points is described as $P=\{(x,D,0):x=\f{L}{N}(n-\f{N-1}{2}),n=\{1,\dots,N\}\}$\,.
\item Now that we have a set of points, we must describe the set of rays originating at those points. The website listed above has a nice set of optimally spaced points on spheres, so we will use that. We will import the coordinates of the points from a text document. In order to apply our methods, we need to rotate the ray into the $xy$-plane. Denote
\begin{align}
\ell(n)=\f{L}{N}\p{n-\f{N}{2}}
\end{align}
Suppose we want to propagate the ray whose coordinates w.r.t the starting point are
\begin{align}
(x,y,z)=\p{\ell(n),D,0}\,.
\end{align}
in the direction $(a,b,c)$ (given by the text document). We need to take this to a point of the form $(x,0)$. It is clear that
\begin{align}
x_0=\sqrt{\ell(n)^2+D^2}
\end{align}
where the rotation is by the angle
\begin{align}
\alpha=\arctan2\p{D,\ell(n)}
\end{align}
and whose matrix is
\begin{align}
\mat{ccc}{\cos\alpha&\sin\alpha&0\\-\sin\alpha&\cos\alpha&0\\0&0&1}\,.
\end{align}
In this new frame, the initial coordinate of the ray is in the $xy$-plane. The direction of propagation, however, may not be. In this new frame, the ray propagates as
\begin{align}
\mat{c}{a'\\b\\c'}=\mat{ccc}{\cos\alpha&\sin\alpha&0\\-\sin\alpha&\cos\alpha&0\\0&0&1}\mat{c}{a+\ell(n)\\b+D\\c}
\end{align}
We now rotate so that the $z$ coordinate is zero, and we extract the angle $\alpha$. The angle we want to rotate by is
\begin{align}
\beta=\arctan2\p{c',b'}\,,
\end{align}
and the rotation matrix is
\begin{align}
\mat{ccc}{1&0&0\\0&\cos\beta&\sin\beta\\0&-\sin\beta&\cos\beta}\,.
\end{align}
We now have a propagation direction in the $xy$-plane:
\begin{align}
\mat{c}{a''\\b''\\0}=\mat{ccc}{1&0&0\\0&\cos\beta&\sin\beta\\0&-\sin\beta&\cos\beta}\mat{c}{a'\\b'\\c'}\,.
\end{align}
From this we can write down the propagation angle
\begin{align}
\alpha_{\te{ray}}=\arctan\p{\f{b''}{a''-x_0}}\,.
\end{align}
We can now use the work from above. 
\item CHECK: Is $x_0>R_{\te{mantle}}$?
\begin{enumerate}
\item If yes, then CHECK: is $s_{\te{mantle}}$ real and $s_{\te{mantle}}>s_{x_0}$? In these cases, we take $s_2=2$. The choice of $s_1=0$ or $1$ is determined by the value of $\alpha$. We assume that the propagation begins by heading towards positive $y$ (so we'll need to account for this for points whose $\alpha$ is in the other direction by taking $\alpha\to2\pi-\alpha$). If this is the case, where $\alpha=0$ corresponds to propagation towards positive $x$, for $\alpha\in[0,\f\pi2]$, we take $s_1=0$. For $\alpha\in[\f\pi2,\pi]$, we take $s_1=1$, and $\alpha\to\pi-\alpha$. In the case that $\alpha>\f\pi2$, there is the possibility that the ray hits the mantle. The point of intersection corresponds to $s_{\te{mantle}}$ with $s_3=1$.
\begin{enumerate}
\item If yes compute $\theta(s_{\te{mantle}})$ and $\sin\phi(s_{\te{mantle}})$ from the formulas in the Numerical Methods section. We now have a new starting angle $\alpha=-\arcsin\phi(s_{\te{mantle}})$. This information is then saved along side the angles $\alpha$ and $\beta$ to be used in the next round of computation.
\item If no, then compute $\theta(s_{\te{Earth}})$, the angle after the ray has propagated to the surface of the earth. This is the final data for this ray.
\end{enumerate}
\item If no, then CHECK: is $\phi(s_{\te{mantle}})$ less than $\phi_{\te{critical}}=\arcsin\p{\f{v_{\te{core}}}{v_{\te{mantle}}}}$?
\begin{enumerate}
\item If yes, then compute $\theta(s_{\te{mantle}})$ and the new starting angle $\alpha=\arcsin\p{\f{v_{\te{mantle}}}{v_{\te{core}}}\sin\phi(s_{\te{mantle}})}$, which will be used in the next round of computation.
\item If no, then throw the point away, as it will be internally reflected forever.
\end{enumerate}
\end{enumerate}
\item Now, all the remaining points have trajectories originating on the surface $r_{\te{mantle}}$ and will intersect the surface of the earth. Thus, all that is left to compute is $\theta(s_{\te{Earth}})$ for each of these points, making sure to keep track of all relevant angles from the previous round of computation (i.e. $\theta(s_{\te{mantle}})$).
\item At this point, we have computed final angles for all rays in their respective frames of reference. What is left is to rotate these points back into the original frame. Since all of these coordinates have the same radius, it is only important to keep track of the angles.
\end{enumerate}
\pagebreak
\section{Code}
The actual code that I've implemented is quite a bit more general than that described above. The general procedure uses the same ideas, but considers both reflection and transmission, as well as an arbitrary amount of layers. The code works by an iterative process, which propagates each ray to the layer next in its trajectory, using the groundwork I've laid out in mathematica. I will elaborate further if necessary.
\pagebreak
\section{Quality Factor}
Since we've determined that it is desirable to model anelastic attenuation, I've determined a method by which we can calculate it without reference to the specific trajectory of the ray. Observe, if $A(t)$ is the amplitude of a ray at a given time, we can relate it its future amplitude for small time by the following relation
\begin{align}
A(t+\delta t) = A(t)e^{K\f{\delta t}{Q(r(t))}}\,.
\end{align}
It makes sense to write $Q(r(t))$, since I've already derived the expression $r(t)$ for a given set of initial parameters. Hence, as part of the above algorithm, it will not require additional code to implement the quality factor. Now, we form a derivative
\begin{align}
\f{A(t+\delta t)- A(t)}{\delta t} &= A(t)\f{e^{K\f{\delta t}{Q(r(t))}}-1}{\delta t}\,,\\
\td{A}{t}{}&= A(t)\lim_{\delta t\to 0}\f{e^{K\f{\delta t}{Q(r(t))}}-1}{\delta t}\,,\\
\td{A}{t}{}&= A(t)e^{-K\f{t}{Q(r(t))}}\lim_{\delta t\to 0}\f{e^{K\f{t +\delta t}{Q(r(t))}}-{e^{K\f{t}{Q(r(t))}}}}{\delta t}\,,\\
\f{1}{A}\td{A}{t}{}&= \f{K}{Q(r(t))}\,,\\
A &= A(t_0)e^{\int_{t_0}^t\diff t'\f{K}{Q(r(t'))}}\,.
\end{align}
We can do the above integral numerically, and perhaps analytically (in the case that we can reasonably write $Q$ as a simple smooth function of $r$). Next we must determine roughly what $Q$ looks like. In the literature, $Q$ is written $Q_p$ when in reference to $p$-waves. *Relevant paper on my phone*
\\\\
We begin with an amplitude $A(k,t)$, a function of amplitude and wave number, gotten from the Fourier decomposition of the source. Since the source depends on the wave vector, it is important to use spherical coordinates in phase space. It may be reasonable to just use the first order terms in $k$ since the lunar seismometers measure from  0.004 Hz to 2 Hz. However, second and third order terms may be desirable, given that signifiant portions of the measurable range lie at order unity. These are gotten by selectively truncating the hypergeometric series in section 2.
\\\\
According to ``Seismic viscoelastic attenuation,''
\begin{align}
Q_p = \f94 Q_s\,,
\end{align}
for a Poisson solid ($v_p =\sqrt3v_s$) which is close enough for our purposes.
\\\\
On the moon, the very preliminary model has $Q_p$ as a piecewise constant function. Therefore, we can do away with the complication of path dependence, and simply break the moon into layers corresponding to these regions.
\pagebreak
\section{Specifics of the Moon Model}
For the region $r \in [380\te{km},1709.1\te{km}]$ the velocity is given by $a^2 - b^2 r^2$ where $a^2 = 8.26955$ and $b^2 = 2.47697\times 10^{-7}$. The velocity in the core is not well known, but because it is small, our calculation will be minimally affected by it, so we extrapolate from the mantle. In the last 28km, the VPREMOON indicates that for $r\in [1709.1,1725.1]$ the velocity is $5.50\te{km\,s}^{-1}$, for $r\in[1725.1,1736.1]$ the velocity is $3.20\te{km\,s}^{-1}$, and for $r\in[1736.1,1737.1]$ the velocity is $1.00\te{km\,s}^{-1}$.
\\\\
Unfortunately for my computer, this means that we must break down the moon into four separate layers simply because of the discontinuity of the velocity field. The other component of the model that requires additional iteration is the quality factor $Q_p$ whose value is piecewise constant as a function of $r$. There are 5 regions. Unfortunately, none of these regions share their boundaries with those of the velocity field. Therefore we have a total of 8 regions. Therefore, the algorithm must be run 16 times if we are to get propagation from one side of the moon to the other. That's quite a bit...
\\\\
To keep track of the quality factor, we need to be able to pointwise multiply $e^{-\f{vk}{2Q}t}$ by $A(k)$. The problem is that Matlab doesn't really do functions... Well... Suppose we start with an amplitude $A(0,k)$, and that the ray propagates through a region of quality factor $Q_1$ for time $\Delta t_1$. Then the amplitude after the propagation is $A(\Delta t_1, k)=A(0,k)F_1 e^{-\f{kv}{2Q_1}\Delta t_1}$, where $F_1$ is a transmission or reflection coefficient, which is not a function of $k$. Now, suppose it propagates through another region with quality factor $Q_2$ for $\Delta t_2$. Then we have $A(\Delta t_1 + \Delta t_2, k)=A(0,k)F_1F_2 e^{-\f{k}{2}\p{\f{\Delta t_1v_1}{Q_1}+\f{\Delta t_2v_2}{Q_2}}}$. We take $v_i$ to be the average velocity in a given region. This is a fine approximation since the velocity doesn't vary much within the moon. That's straight forward enough. Therefore, to understand what happens to a general ray's amplitude, we only need to know two things:
\begin{align}
F &= \prod_i F_i\,,\\
\f{T}{Q} &= \sum_i\f{\Delta t_i v_i}{Q_i}\,.
\end{align}
Now, after the algorithm has been run, and we want to know how much energy is left in a particular ray with initial amplitude distribution $A(k)$, we calculate the energy associated to $\tilde A(k) = A(k)Fe^{-\f{k}{2}\f{T}{Q}}$. That's not hard. Moreover, we only want to obtain the total detectible energy i.e. the energy of those modes in the the frequency range $[0.004,2]\te{Hz}$. This translates to the wave number range $[0.025,12.6]$ (since we use the velocity near the surface for measurement, which is $1\te{km\,s}^{-1}$).
\\\\
The notion of ray tracing is based in the Eikonal approximation, where a wave evolves in a coherent fashion. Each ray can be thought of as containing $1/N$ the total amount of energy, where $N$ is the total number of rays (in the program this is just $N\times M$). Thus, the energy of an individual ray will be multiplied by $N^{-1}$.


Now, assuming we've got a pressure wave of the form (that I describe in section 2)
\begin{align}
P(\vec k)=\f{4\pi p_0}{\sqrt{N}k\cos\theta}\sin\p{\f{h k\cos\theta}{2}}(b^2f_1(b)-a^2f_1(a))
\end{align}
After attenuation, the form of the wave is
\begin{align}
\tilde P(\vec k) = P(\vec k)Fe^{-\f{k}{2}\f{T}{Q}}\,.
\end{align}
Observe that truncating the hypergeometric function at leading order is an overestimate for all $k$ and $\theta$ (since $F:[0,\infty)\to[0,1]$), so using my approximation is fine. The energy of a ray is not really a well defined quantity since a ray itself is somewhat of an infinitesimal wave packet, and is somewhat analogous to a test particle etc. One way, however, to extract a usable quantity is to think of a ray as representing one $N$th of the whole wave, and to simply calculate its energy as though it exists throughout the whole space. A complication arises from the $r$ dependence of $\rho$ and $\lambda + 2\mu$. Because we are only really interested in orders of magnitude, it's fairly reasonable to take $\rho$ and $\lambda + 2\mu$ as their weighted averages, i.e.
\begin{align}
\bar\rho = \f{3}{R_{\te{Moon}}^3}\int r^2\diff r\rho(r)
\end{align}
and
\begin{align}
\overline{\lambda + 2\mu} =  \f{3}{R_{\te{Moon}}^3}\int r^2\diff r\p{\lambda(r) + 2\mu(r)}\,.
\end{align}
\pagebreak
\section{Inhomogeneous}
Given that we are only looking for numerical results, let's consider the inhomogeneous Moon. The energy deposition is
\begin{align}
\td{E}{x}{} = \rho(r)\sigma_X v_X^2\,.
\end{align}
Now, the total energy deposition of a macro is
\begin{align}
E =2\sigma_X v_X^2 \int_0^{R_{\te{Moon}}}\diff r\rho(r)\,.
\end{align}
$\rho = a-b(x-c)^2$ where $a=3.446$, $b = 5\times 10^{-8}$ and $c = 70$, ($\rho$ is given in $\te{kg\,cm}^{-1}$), except for $r<380\te{km}$ where $\rho =5.171\te{kg\,cm}^{-1}$.
\\\\
Now, if we're interested in the energy allocated to each ray, let $M$ be the number of points on the source line from which rays will propagate, and let $N$ be the number of rays from each point. The energy in the slice (of length $l$) corresponding to some point $m$ is then
\begin{align}
E_m = 2\sigma_X v_X^2\int_{x_0(m)- l/2}^{x_0(m) + l/2}\diff r\rho(r)\,.
\end{align}
Since $\rho$ is approximately constant over the interval $x_0(m)$, it is reasonable to write
\begin{align}
E_m = 2\sigma_Xv_X^2 l\rho(x_0(m))\,.
\end{align}
This above expression will be used in the program.
\\\\
Now, in the neighborhood of the point, the moon is approximately homogeneous, and so the machinery from earlier works. We consider a pressure source of the form
\begin{align}
p=p_0f(r)\ps{\theta\p{z+\f l2}-\theta\p{z-\f l2}}
\end{align}
where we've aligned our axes with the trajectory of the macro. Since $l$ is not necessarily large with respect to $\lambda_0$, we have to use the full expression from section 2. The energy of this pressure source is
\begin{align}
E_m = \f{\lambda_m + 2\mu_m}{K_m^2} l p_0(m)^2\int r\diff r f(r)^2
\end{align}
We can now fix $p_0(m)$,
\begin{align}
p_0(m)^2 = \f{ K_m^2}{\lambda_m + 2\mu_m}\f{2\rho_m\sigma_Xv_X^2}{\int r\diff r f(r)^2}\,.
\end{align}
In order to evaluate the effect of the anelastic attenuation we need the spectrum of a ray, and how that relates to the energy of the ray. Note, if each of the $M$ segments has $E_m$ energy, and each segment spawns $N$ rays, then each ray has $E_m/N$ energy.
\\\\
The energy of a ray is
\begin{align}
\f{E_m}{N}=\f{1}{N}\f{\lambda_m+2\mu_m}{K_m^2}\int\f{\diff^3 k}{(2\pi)^3}\abs{P_m(\vec k)}^2
\end{align}
where $P(\vec k)$ is given in section 2. The quantities out front are that of the neighborhood of the segment since the pressure wave is initially localized in space. Since we are only concerned with the measurable energy the integral is only taken over $[0.025,12.6]$. Then, after anelastic attenuation and refraction/reflection
\begin{align}
\f{E_m'}{N}=\f{1}{N}\f{\lambda_m+2\mu_m}{K_m^2}F_{m,n}\int\f{\diff^3 k}{(2\pi)^3}\abs{P_m(\vec k)}^2e^{-k\f{T_{m,n}}{Q_{m,n}}}
\end{align}
Because $Q_{m,n}$ and $T_{m,n}$ take into account the material properties along the path of the ray, the coefficients out front, which are those of the origin of the ray, act as a normalization, so that in the case of no refraction/reflection and anelastic attenuation the energy is that of the ray at the start of its journey.
\\\\
The approximation (32) holds when $bk_0\ll 2$, and is always an overestimate. Since we know $b$ is on the order of meters, this is, in fact, a good approximation ($k_0$ is of order millimeters). Thus
\begin{align}
\f{E_m'}{N} &= \f{1}{N}\f{\lambda_m+2\mu_m}{K_m^2}F_{m,n}\int\f{\diff^3 k}{(2\pi)^3}\abs{\f{4\pi p_0}{k\cos\theta}\sin\p{\f{h k\cos\theta}{2}}(b^2f_1(b)-a^2f_1(a))}^2e^{-k\f{T_{m,n}}{Q_{m,n}}}\,,\\
\notag &=\f{Q}{NT^2}e^{-\f{k_0^2T}{Q}}\ps{4T(1-\cos(l k_0)) + \I e^{\f{k_0 T}{Q}} l Q\p{2\te{Ei}\ps{-\I l \notag k_0-\f{k_0 T}{Q}} - 2\te{Ei}\ps{\I l k_0 - \f{k_0 T}{Q}} + \right.\right.\\&\left.\left.+\ln\ps{-\f{Q}{\I l Q+T}}+\ln\notag \ps{-\f{\I l Q + T}{Q}} + 2\ln\ps{\I l - \f{T}{Q}}} - 4 l(Q + k_0 T)\te{Si}\ps{l k_0}}\times\\
&\times (b^2f_1(b)-a^2f_1(a))^2\f{\lambda_m+2\mu_m}{K_m^2}F_{m,n}p_0(m)^2\,,\\
\notag &=\f{Q}{NT^2}e^{-\f{k_0^2T}{Q}}\ps{4T(1-\cos(l k_0)) + \I e^{\f{k_0 T}{Q}} l Q\p{2\te{Ei}\ps{-\I l \notag k_0-\f{k_0 T}{Q}} - 2\te{Ei}\ps{\I l k_0 - \f{k_0 T}{Q}} + \right.\right.\\&\left.\left.+\ln\ps{-\f{Q}{\I l Q+T}}+\ln\notag \ps{-\f{\I l Q + T}{Q}} + 2\ln\ps{\I l - \f{T}{Q}}} - 4 l(Q + k_0 T)\te{Si}\ps{l k_0}}\times\\
&\times \underbrace{\f{(b^2f_1(b)-a^2f_1(a))^2}{\int r\diff r f(r)^2}}_{A}F_{m,n} 2\rho_m\sigma_Xv_X^2\,.
\end{align}
$A$ is a constant which characterizes the explosion for small $k$. In the case that $f$ is constant, $A = f\times (b^2-a^2)$. If $f$ is the triangle from section 2, then $A = \f16\Delta r(3r_0 + 2\Delta r)$. When $T$ is zero, we get the result from section 2, which goes approximately as $k_0^2$. A quick plot of $E_m'$ shows that it goes as $k_0^2$ for small $k_0$ but asymptotes to a constant value (more precisely, it goes as the expression in section 2). This constant value is achieved slower for larger $Q$ (or smaller $T$). In essence, this shows how the higher frequency terms stop contributing, and the small $k$ terms dominate. Note that $Q$ is on the order of $10^2$ or $10^3$, while $T$ is likely on the order of $10^3$ (seconds).
\\\\
In the case that $\f{T}{Q}\gg 1$, then $k_0\to\infty$ is close to the actual value. In this case, the limit is just
\begin{align}
E_m' =  4 l  \f{Q^2}{T^2} \arctan\ps{l\f{Q}{T}} F_{m,n} A2\rho_m\sigma_Xv_X^2\,.
\end{align}
Unfortunately, this approximation does not hold for rays originating near the surface (which always exist for our line source), since $Q_p$ near the surface is large and $T$ is small. In this case, we can use the expression from section 1. But when our computers are super fast, why approximate?
\\\\
How do we remove $A$ from this equation?
\\\\
Now, for a ray that immediately hits the surface, its energy must be the same as when it started. This corresponds to $T \to 0$ and $F_{m,n} = 1$
\begin{align}
E_m = 4\p{-2k_0+k_0\cos(l k_0)+\f{\sin(h k_0)}{l}+l k_0^2\te{Si}(l k_0)} A \td{E}{x}{}(m)\,.
\end{align}
Hence
\begin{align}
A = \f{l}{4} \p{-2k_0+k_0\cos(l k_0)+\f{\sin(h k_0)}{l}+l k_0^2\te{Si}(l k_0)}^{-1}\,.
\end{align}
Clearly, a seismometer will not detect all the energy from a macro impact, since it only detects a certain range of frequencies. Thus, by fixing $A$ as we have, we are insisting that the energy of the impact be concentrated at lower $k$. We are assuming that all the energy from the explosion is initially concentrated in the lower frequencies below $k_0$, and that it is distributed according to the leading order expression inside the integral of (172). This is good for an upper estimate of the energy felt by seismometers. Note, for bigger $k_0$, $E_m'$ experiences a faster drop-off as $T$ gets big. The choice of $k_0$ is then very important, since it fixes the rate of attenuation. As long as $k_0 l<1$, this will be an overestimate. 
\\\\
One potential flaw with this model is that it DOES over-estimate the effect of anelastic attenuation on the shorter of the long modes, due to the nature of the approximation to the hypergeometric function I used.
\\\\
POSSIBLE REASONING: This computation is justified in itself, since $b$, even after non-linear evolution, is at most on the order of 10s to 100s of meters - and based on the size of the plastic zone potentially millimeters, $k_0 = 12\te{km}^{-1}$ times $b$ is at most $10^{-1}\ll 2$ so the leading order term in the Hypergeometric series dominates.
\\\\
AHA! $k_0$ should be the LOWER frequency i.e. $0.012\te{km}^{-1}$, since any lower would mean it is undetectable, but any higher and we could be over-estimating the amount of attenuation!!!! HAHA!
\pagebreak
\section{A Good Method of Overestimation}
Forget about the hypergeometric. Take the $f$ source to be of the optimal form to produce seismic waves that will reach the detector, i.e. take $f= A\delta(k-k_0)$ where $k_0$ is the lowest frequency of response of the moon seismometers. In this case, the total energy of the ray just goes as $e^{-k_0\f{T_{m,n}}{Q_{m,n}}}$. According to http://nssdc.gsfc.nasa.gov/nmc/experimentDisplay.do?id=1971-008C-04, the lowest useful range of any of the lunar seismometers (some had higher) was $0.004\te{Hz}$. The velocity of p-waves near the surface is $1\te{km\,s}^{-1}$, and so this corresponds to $k_0 = 2\pi\times 0.004\te{km}^{-1}$. However, another useful quantity may be the peak response, which is $0.45\te{Hz}$. I'll run the model with both of these quantities.
\\\\
Unfortunately, the above overestimate is too much of an overestimate. It is likely the best line of thinking to use the guessed wave-form. In this case, we have
\begin{align}
P(\vec k) = \f{4\pi p_0}{k\cos\theta}\sin\p{\f{h}{2}k\cos\theta}
\end{align}
We may just want to do numerical integration. The problem is that these integrals oscillate rather quickly... Another idea is to simply drop all terms that are suppressed by some cutoff factor.
\pagebreak
\section{Better Estimate}
The frequency distribution of the rays is to first order $\f{kb}{2}\ll 1$
\begin{align}
P(\vec k) = \f{4\pi p_0}{k\cos\theta}\sin\p{\f{h k\cos\theta}{2}}(b^2f_1(b)-a^2f_1(a))
\end{align}
Indeed, since $b$ is on the order of $1\te{m}$, and the frequencies measurable to the lunar seismometers are at most $2\te{Hz}$, we have that $\f{bk}{2}\approx 10^{-3}\ll 1$. Now, the attenuated energy in this part of the spectrum goes as
\begin{align}
\f{\lambda+2\mu}{K^2}\int\f{\diff^3k}{2\pi}P(\vec k)^2e^{-k\f{T}{Q}}
\end{align}
The energy in the range $[k_0,k_1]$ where $\f{k_1 b}{2}\ll 1$. is then
\begin{align}
\notag E_{\te{attenuated}} =&\f{\lambda + 2\mu}{K^2}2p_0^2(b^2 f_1(b) - a^2 f_1(a))^2 \f{1}{T^2}e^{-\f{(k_0+k_1)(\I h Q+T)}{Q}} Q\p{-\p{e^{\I h k_0+\f{k_0T}{Q}} \notag -2e^{\I h(k_0+k_1)+\f{k_0 T}{Q}} \right.\right.\\&\left.\left.+ e^{\I h(k_0 + 2k_1) + \f{k_0 T}{Q}} - e^{\I h k_1 \notag + \f{k_1 T}{Q}} + 2 e^{\I h(k_0 + k_1) + \f{k_1 T}{Q}} - e^{\I h(2k_0 + k_1)+\f{k_1 T}{Q}}}T \right.\\&\left.- \I \notag e^{\f{(k_0+k_1)(\I h Q + T)}{Q}}h Q\p{\te{Ei}\ps{-\I h k_0 - \f{k_0 T}{Q}} - \te{Ei}\ps{\I h k_0 - \f{k_0 T}{Q}} - \te{Ei}\ps{-\I h k_1 - \f{k_1 T}{Q}} + \te{Ei}\ps{\I h k_1 - \f{k_1 T}{Q}}   } \right.\\&\left.+ 2e^{\I h(k_0+k_1)} h \p{e^{\f{k_1 T}{Q}}(Q + k_0 T)\te{Si}\ps{h k_0} - e^{\f{k_0 T}{Q}}(Q + k_1 T)\te{Si}\ps{h k_1}}}\,.
\end{align}
We know that the energy in this range before attenuation is
\begin{align}
E_{\te{detectable}} = \f{\lambda+2\mu}{K^2}2p_0^2(b^2f_1(b)-a^2f_1(a))^2\ps{-2k+k\cos(h k)+\f{\sin(h k)}{h}+h k^2\te{Si}(h k)}_{k=k_0}^{k_1}
\end{align}
What is the measurable attenuated energy in terms of the total deposited energy?
\begin{align}
E_{\te{total}} = 2\pi p_0^2h\f{\lambda+2\mu}{K^2}\int_0^\infty r\diff r f(r)^2\,.
\end{align}
We need to pick a function $f$. A shock wave tends to be a right triangle, so let's pick the triangle from before. We then get
\begin{align}
E_{\te{total}}=\f{\lambda+2\mu}{K^2}\f16 p_0^2 \pi\Delta r h(4 r_0+3\Delta r)\,,
\end{align}
and
\begin{align}
f_1(x)=\f{-r_0}{2\Delta r}+\f{x}{3\Delta r}\,.
\end{align}
Hence, for $b = r_0+\Delta r$ and $a = r_0$, we have
\begin{align}
b^2 f_1(b)-a^2f_1(a) = \f16\Delta r(3r_0+2\Delta r)\,.
\end{align}
The initial energy deposition is
\begin{align}
E = \rho\sigma_Xv_X^2h\,.
\end{align}
Thus
\begin{align}
2p_0^2\f{\lambda+2\mu}{K^2} =  \f{12}{\pi}\f{\rho\sigma_Xv_X^2}{\Delta r (4r_0+3\Delta r)}\,.
\end{align}
Hence
\begin{align}
\notag E_{\te{attenuated}} =&\f{1}{3\pi}\f{\rho\sigma_Xv_X^2}{\Delta r (4r_0+3\Delta r)}\ps{ \Delta r(3r_0+2\Delta r)}^2 \f{1}{T^2}e^{-\f{(k_0+k_1)(\I h Q+T)}{Q}} Q\p{-\p{e^{\I h k_0+\f{k_0T}{Q}} \notag -2e^{\I h(k_0+k_1)+\f{k_0 T}{Q}} \right.\right.\\&\left.\left.+ e^{\I h(k_0 + 2k_1) + \f{k_0 T}{Q}} - e^{\I h k_1 \notag + \f{k_1 T}{Q}} + 2 e^{\I h(k_0 + k_1) + \f{k_1 T}{Q}} - e^{\I h(2k_0 + k_1)+\f{k_1 T}{Q}}}T \right.\\&\left.- \I \notag e^{\f{(k_0+k_1)(\I h Q + T)}{Q}}h Q\p{\te{Ei}\ps{-\I h k_0 - \f{k_0 T}{Q}} - \te{Ei}\ps{\I h k_0 - \f{k_0 T}{Q}} - \te{Ei}\ps{-\I h k_1 - \f{k_1 T}{Q}} + \te{Ei}\ps{\I h k_1 - \f{k_1 T}{Q}}   } \right.\\&\left.+ 2e^{\I h(k_0+k_1)} h \p{e^{\f{k_1 T}{Q}}(Q + k_0 T)\te{Si}\ps{h k_0} - e^{\f{k_0 T}{Q}}(Q + k_1 T)\te{Si}\ps{h k_1}}}\,,\\
=&\f{1}{3\pi}\f{\rho\sigma_Xv_X^2\Delta r}{(4r_0+3\Delta r)}\ps{(3r_0+2\Delta r)}^2 \f{1}{T^2}e^{-\f{(k_0+k_1)(\I h Q+T)}{Q}} Q\p{-\p{e^{\I h k_0+\f{k_0T}{Q}} \notag -2e^{\I h(k_0+k_1)+\f{k_0 T}{Q}} \right.\right.\\&\left.\left.+ e^{\I h(k_0 + 2k_1) + \f{k_0 T}{Q}} - e^{\I h k_1 \notag + \f{k_1 T}{Q}} + 2 e^{\I h(k_0 + k_1) + \f{k_1 T}{Q}} - e^{\I h(2k_0 + k_1)+\f{k_1 T}{Q}}}T \right.\\&\left.- \I \notag e^{\f{(k_0+k_1)(\I h Q + T)}{Q}}h Q\p{\te{Ei}\ps{-\I h k_0 - \f{k_0 T}{Q}} - \te{Ei}\ps{\I h k_0 - \f{k_0 T}{Q}} - \te{Ei}\ps{-\I h k_1 - \f{k_1 T}{Q}} + \te{Ei}\ps{\I h k_1 - \f{k_1 T}{Q}}   } \right.\\&\left.+ 2e^{\I h(k_0+k_1)} h \p{e^{\f{k_1 T}{Q}}(Q + k_0 T)\te{Si}\ps{h k_0} - e^{\f{k_0 T}{Q}}(Q + k_1 T)\te{Si}\ps{h k_1}}}\,.
\end{align}
Note, the bigger $r_0$ and $\Delta r$ are, the bigger $E_{\te{attenuated}}$ will be. A very generous over estimate is to say that $\Delta r = r_0 = 1\te{m}$. Another way to do this is as we did earlier, by starting the propagation once the wave has reached a certain pressure $\bar p \approx 10^8\te{Pa}$. Then
\begin{align}
 r_0 = \f{K^2}{\lambda+2\mu} \f{3}{2\pi\Delta r}\f{\rho\sigma_Xv_X^2 }{p_0^2}-\f{3}{4}\Delta r
\end{align}
A quick plot shows that the attenuation here is pretty much negligible.
\pagebreak
\section{What Counts as Detectable?}
The most sensitive of the moon seismometers could measure $0.008\te{mgal}=8\times 10^{-8}\te{m\,s}^{-2}$. Their mass is approximately $10\te{kg}$, and they are sensitive to displacements down to $0.3\times 10^{-9}\te{m}$. If in a HEALPix pixel, the energy density is $\epsilon$, the energy on the seismometer is given by the surface area the seismometer occupies. It's radius is $11.5\te{cm}$. Call the radius $r_s$, the displacement sensitivity $D$, the mass of the seismometer $M$, and the acceleration sensitivity $A$. Then the seismometer will sense $\epsilon$ such that
\begin{align}
\epsilon\f{\pi r_s^2}{MD} \geq A
\end{align}
i.e.
\begin{align}
\epsilon \geq 6\times 10^{-15} \te{J\,m}^{-2}\,.
\end{align}
This is probably wrong. They can measure a displacement of $3\times 10^{-10}\te{m}$. We can calculate the displacement due to a seismic $p$-wave. Suppose $P$ is the Fourier transform of the pressure wave. Then the displacement field's Fourier transform is 
\begin{align}
U(\vec k) =\I\f{P(\vec k)}{K}\f{\vec k}{k^2}\,. 
\end{align}
The displacement is then... I FORGOT AN $\omega t$!!! I'll redo later tonight
\begin{align}
u &= \int\f{\diff^3 k}{(2\pi)^3}\I\f{P(\vec k)}{K}\f{\vec k}{k^2}e^{\I \vec k\cdot\vec x - \I\omega t}\,,\\
&=\f{1}{K}\int\f{\diff^3 k}{(2\pi)^3}\I\f{4\pi p_0}{k\cos\theta}\sin\p{\f{h k\cos\theta}{2}}(b^2f_1(b)-a^2f_1(a))\f{\vec k}{k^2}e^{\I \vec k\cdot\vec x}\,,\\
&=\I2 p_0\f{(b^2f_1(b)-a^2f_1(a))}{K}\int\f{\diff k\diff\theta\diff\phi}{(2\pi)^2}\f{k^2\sin\theta}{k\cos\theta}\sin\p{\f{h k\cos\theta}{2}}\f{\hat k}{k}e^{\I (kx\cos\varphi)}\,,\\
&=\I2 p_0\f{(b^2f_1(b)-a^2f_1(a))}{K}\int\f{\diff k\diff\theta\diff\phi}{(2\pi)^2}\tan\theta\sin\p{\f{h k\cos\theta}{2}}\hat k e^{\I (kx\cos\varphi)}\,,\\
&=\I2 p_0\f{(b^2f_1(b)-a^2f_1(a))}{K}\int\f{\diff\theta\diff\phi}{(2\pi)^2}\ps{\f{2h\cos\theta-2e^{\I k_0 x\cos\varphi}\p{h\cos\theta\cos\p{\f{h}{2}k_0\cos\theta} - 2\I x\cos\varphi\sin\p{\f{h}{2}k_0\cos\theta}}}{h^2\cos^2\theta-4x^2\cos^2\varphi}}\,,\\
&=\I2 p_0\f{(b^2f_1(b)-a^2f_1(a))}{K}\int\f{\diff\theta}{2\pi}\ps{\f{2h-2e^{\I k_0 x\cos\theta}\p{h\cos\p{\f{h}{2}k_0\cos\theta} - 2\I x\sin\p{\f{h}{2}k_0\cos\theta}}}{(h^2-4x^2)\cos\theta}}\,.
\end{align}
By proper choice of coordinates, we can insist that $\varphi = \theta$. Moreover, we know that $u$ is real, and hence we can write
\begin{align}
u &=\I2 p_0\f{(b^2f_1(b)-a^2f_1(a))}{K}\int\f{\diff\theta}{2\pi}\ps{\f{-2\I\p{h\sin\p{k_0 x\cos\theta}\cos\p{\f{h}{2}k_0\cos\theta} - 2 x\cos\p{k_0 x\cos\theta}\sin\p{\f{h}{2}k_0\cos\theta}}}{(h^2-4x^2)\cos\theta}}\,,\\
&=4 p_0\f{(b^2f_1(b)-a^2f_1(a))}{K}\int\f{\diff\theta}{2\pi}\ps{\f{\p{h\sin\p{k_0 x\cos\theta}\cos\p{\f{h}{2}k_0\cos\theta} - 2 x\cos\p{k_0 x\cos\theta}\sin\p{\f{h}{2}k_0\cos\theta}}}{(h^2-4x^2)\cos\theta}}\,,\\
&=k_0\f{p_0}{2}\f{(b^2f_1(b)-a^2f_1(a))}{K}\p{_p\te{F}_q\ps{\pc{\f12},\pc{1,\f32},-\p{\f{k_0(h+2x)}{4}}^2} -_p\te{F}_q\ps{\pc{\f12},\pc{1,\f32},-\p{\f{k_0(h-2x)}{4}}^2}}
\end{align}

CORRECTION
\\\\
Note that we want to know the amplitude as a function of cylindrical $r$, hence choose $k\cdot x = k r \sin\theta$, since they should be aligned when $\theta$ is $\pi/2$. That integral doesn't seem to work, but the displacement as a function of $z$ is
\begin{align}
u &= \int\f{\diff^3 k}{(2\pi)^3}\I\f{P(\vec k)}{K}\f{\vec k}{k^2}e^{\I \vec k\cdot\vec x - \I\omega t}\,,\\
&=\f{1}{K}\int\f{\diff^3 k}{(2\pi)^3}\I\f{4\pi p_0}{k\cos\theta}\sin\p{\f{h k\cos\theta}{2}}(b^2f_1(b)-a^2f_1(a))\f{\vec k}{k^2}e^{\I \vec k\cdot\vec x - \I k v t}\,,\\
&=\f{\I4\pi p_0}{K}(b^2f_1(b)-a^2f_1(a))\int\f{\diff^3 k}{(2\pi)^3}\f{1}{k\cos\theta}\sin\p{\f{h k\cos\theta}{2}}\f{\vec k}{k^2}e^{\I k(z\cos\theta -  v t)}\,,\\
&=\f{\I2 p_0}{K}(b^2f_1(b)-a^2f_1(a))\int\f{\diff k\diff\theta\diff\phi}{(2\pi)^2}\tan\theta\sin\p{\f{h k\cos\theta}{2}}e^{\I k(z\cos\theta -  v t)}\hat k\,,\\
\end{align}
Note, then that $u\cdot\vec  = \abs{u}z\cos\theta$ and that
\begin{align}
\abs{u}&=\f{\I2 p_0}{K}(b^2f_1(b)-a^2f_1(a))\int\f{\diff k\diff\theta\diff\phi}{(2\pi)^2}\tan\theta\sin\p{\f{h k\cos\theta}{2}}e^{\I k(r\cos\theta -  v t)}\,,\\
\end{align}
A priori we know that only the complex part of the integral survives, and we are left with
\begin{align}
\notag\abs{u} =& \f{ p_0}{K}(b^2f_1(b)-a^2f_1(a))\f{1}{tv}\ps{-2\cos(k_0 tv)\p{\te{Si}\ps{\f12 k_0(h - 2z)} + \te{Si}\ps{\f12 k_0(h + 2z)}}\right.\\&\left. + \te{Si} \ps{\f12 k_0(h+2z-2tv)}+ \te{Si}\ps{\f12 k_0(h-2z+2tv)}+\te{Si}\ps{\f12 k_0(h+2z+2tv)}+\te{Si}\ps{\f12 k_0(h-2z-2tv)}}
\end{align}
Note that $r$ is the cylindrical coordinate. Taking $z = vt$, we get the maximum amplitude of the propagating displacement wave. For large $t$, the stuff in the square bracket tends to $2\te{Si}\ps{h k_0/2}$. The convergence is quite rapid. Note, the cosine term dominates the behavior of the background in response to the source, and the independent sine integral terms dominate the propagation of the wave.
\pagebreak
\section{Another Estimate}
Suppose that we have the energy in the frequencies less than $f$ is
\begin{align}
E = Cf^2\,.
\end{align}
This means that each frequency contributes energy proportional to itself, i.e.
\begin{align}
E(f) = 2 Cf
\end{align}
This frequency is then attenuated
\begin{align}
E(f) =  2 C_1 f e^{- C_2 f}
\end{align}
and so the energy in the spectrum below $f$ is
\begin{align}
E = \f{2 C_1(1-e^{-C_2 f}(1+C_2 f))}{C_2^2}
\end{align}
To see this more formally, observe that if the energy below frequency $f$ is
\begin{align}
E=C_1f^2\,,
\end{align}
then the energy in the interval $[f,f+\delta]$ is
\begin{align}
\delta E = E(f+\delta)-E(f)
\end{align}
This energy is then approximately attenuated as though all the energy were at frequency $f$, i.e.
\begin{align}
\delta E = e^{-C_2f}(E(f+\delta)-E(f))\,.
\end{align}
Hence the total attenuated energy is
\begin{align}
E = \sum_f\delta Ee^{-C_2f} = \sum_f\delta e^{-C_2f}\f{(E(f+\delta)-E(f))}{\delta}\,.
\end{align}
Taking the limit as $\delta\to 0$ then leads to
\begin{align}
E = \int_0^f\diff f' \td{E(f')}{f'}{}e^{-C_2f}\,.
\end{align}
Now that we have attenuated $E$ as a function of $f$, we try to determine displacement. This is possible because we know how the energy is distributed as a function of $f$.
\\\\
Suppose we have a displacement wave at wavenumber $k$ and energy $E$. We can, in particular, write the wave in the form
\begin{align}
u_f = A_fe^{-i(\vec k\cdot \vec x - k v t)}\,.
\end{align}
Then the energy is
\begin{align}
E(f+\delta)-E(f) &= \rho\int\diff^3 x \abs{\partial_t A_fe^{-\I(\vec k\cdot \vec x - k v t)}}^2\,,\\
&= \rho\int\diff^3 x \abs{\I k v A_fe^{-\I(\vec k\cdot \vec x - k v t)}}^2\,,\\
&= \rho\int\diff^3 x  k^2 v^2 A_f^2\,,\\
&= \rho V  k^2 v^2 A_f^2\,,\\
&= \rho V  (2\pi)^2 f^2 A_f^2\,.
\end{align}
\begin{align}
A^2 = \lim_{\delta\to 0}\sum_f (A_f^2) &= \lim_{\delta\to 0}\sum_f \f{E(f+\delta)-E(f)}{f^2}\p{\rho V  (2\pi)^2}^{-1}\,,\\
&=\f{1}{\rho V  (2\pi)^2} \int \f{\diff E}{f^2}\,,\\
&=\f{1}{\rho V  (2\pi)^2} \int \f{\diff f}{f^2}\td{E}{f}{}\,,
\end{align}
OR?
\begin{align}
A = \lim_{\delta\to 0}\sum_f (A_f) &= \lim_{\delta\to 0}\sum_f\sqrt{ \f{E(f+\delta)-E(f)}{f^2}\p{\rho V  (2\pi)^2}^{-1}}\,,\\
&=\f{1}{\sqrt{\rho V}  2\pi} \int \f{\diff \sqrt{E}}{f}\,,\\
&=\f{1}{\sqrt{\rho V}  2\pi} \int \f{\diff f}{f}\td{\sqrt{E}}{f}{}\,.
\end{align}
so,
\begin{align}
A = \sqrt{\f{1}{\rho V  (2\pi)^2} \int \f{\diff f}{f^2}\td{E}{f}{}}\,.
\end{align}
In our case,
\begin{align}
A &= \sqrt{\f{1}{\rho V  (2\pi)^2} \int_{f_0}^{f_1} \diff f' \f{C_1 e^{-C_2 f'}}{f}}\,,\\
&= \sqrt{\f{1}{\rho V  (2\pi)^2}C_1\p{\te{Ei}\ps{-C_2f_1}-\te{Ei}\ps{-C_2f_0}}}\,,\\
&= \f{1}{f}\sqrt{\f{E_{ray}}{\rho V  (2\pi)^2}\p{\te{Ei}\ps{-C_2f_1}-\te{Ei}\ps{-C_2f_0}}}\,,\\
&= \f{1}{2\pi f}\sqrt{\f{E_{ray}}{\rho A_{\te{ray}}\Delta T v_p}\p{\te{Ei}\ps{-C_2f_1}-\te{Ei}\ps{-C_2f_0}}}\,,
\end{align}
which coincides with Prof. Starkman's estimate for $C_2 = 0$.
\pagebreak
\section{Revised Estimate}
Suppose we have the displacement field
\begin{align}
u &= \int\f{\diff^3 k}{(2\pi)^3}U(\vec k)e^{-\I(\vec k\cdot \vec x-k v_p t)}\,.
\end{align}
Since we want the maximum displacement, we set $t=0$. In the case that $U$ doesn't depend on $\hat k$, we have
\begin{align}
u &=\int\f{\diff^3 k}{(2\pi)^3}U(k)e^{-\I(\vec k\cdot \vec x)}\,,\\
&=\int\f{\diff k\diff\theta\diff\phi}{(2\pi)^3}k^2\sin\theta U( k)e^{-\I k r \cos\phi}\,,\\
&=\int\f{\diff k}{(2\pi)^2}k^2U( k) J_0(k r)\,.
\end{align}
When $kr\ll 1$, we have
\begin{align}
u&\approx \int\f{\diff k}{(2\pi)^2}k^2U(k)\,.
\end{align}
The requirement that $kr\ll 1$ may determine the maximum size of the pixels. Then
\begin{align}
\partial_t u =\I v_p\int\f{\diff^3 k}{(2\pi)^3}kU(\vec k)e^{-\I(\vec k\cdot \vec x-k v_p t)}\,,
\end{align}
and
\begin{align}
\abs{\partial_t u}^2 = v_p^2\int\f{\diff^3 k}{(2\pi)^3}\int\f{\diff^3 k'}{(2\pi)^3}kk'U(\vec k)\bar U(\vec k')e^{-\I(\vec k'\cdot \vec x-k' v_p t)}e^{-\I(\vec k\cdot \vec x-k v_p t)}
\end{align}
Thus
\begin{align}
E = \rho\int\diff^3 x (\partial_t u)^2 &= v_p^2\rho\int\diff^3 x\int\f{\diff^3 k}{(2\pi)^3}\int\f{\diff^3 k'}{(2\pi)^3}kk'U(\vec k)\bar U(\vec k')e^{\I(\vec k'\cdot \vec x-k' v_p t)}e^{-\I(\vec k\cdot \vec x-k v_p t)}\,,\\
&=v_p^2\int\f{\diff^3 k}{(2\pi)^3}\int\f{\diff^3 k'}{(2\pi)^3}kk'U(\vec k)\bar U(\vec k')\rho\int\diff^3 xe^{\I(\vec k'\cdot \vec x-k' v_p t)}e^{-\I(\vec k\cdot \vec x-k v_p t)}\,,\\
&=v_p^2\int\f{\diff^3 k}{(2\pi)^3}\int\f{\diff^3 k'}{(2\pi)^3}kk'U(\vec k)\bar U(\vec k')\rho\int\diff^3 xe^{\I((\vec k'-\vec k)\cdot \vec x-(k'-k) v_p t)}\,,\\
&=v_p^2\int\f{\diff^3 k}{(2\pi)^3}\int\f{\diff^3 k'}{(2\pi)^3}kk'U(\vec k)\bar U(\vec k')\rho \delta(\vec k-\vec k')e^{\I(k-k') v_p t}\,,\\
&=v_p^2\int\f{\diff^3 k}{(2\pi)^3}\int\f{\diff^3 k'}{(2\pi)^3}kk'U(\vec k)\bar U(\vec k')\rho \delta(\vec k-\vec k')\,,\\
&=\rho v_p^2\int\f{\diff^3 k}{(2\pi)^3}k^2\abs{U(\vec k)}^2\,.
\end{align}
Assuming $U$ has no $\hat k$ dependence
\begin{align}
E&=2\rho v_p^2\int_{0}^{\infty}\f{\diff k}{(2\pi)^2}k^4\abs{U(k)}^2\,.
\end{align}
Define
\begin{align}
E_k &=2\rho v_p^2\int_{0}^{k}\f{\diff k'}{(2\pi)^2}k'^4\abs{U(k')}^2\,.
\end{align}
Then
\begin{align}
\td{E_k}{k}{} =\f{2\rho v_p^2}{(2\pi)^2}k^4\abs{U(k)}^2
\end{align}
Hence
\begin{align}
\abs{U(k)} = \f{2\pi}{\sqrt{2\rho v_p^2}}\f{1}{k^2}\sqrt{\td{E_k}{k}{}}\,.
\end{align}
Whence
\begin{align}
u \leq \int_0^\infty \f{\diff k}{(2\pi)^2} k^2\abs{U(k)} = \f{1}{2\pi\sqrt{2\rho v_p^2}}\int_0^\infty\diff k\sqrt{\td{E_k}{k}{}}\,.
\end{align}
In our case, when we want to only consider modes up to some $k$, we have
\begin{align}
u_k \leq \f{1}{2\pi\sqrt{2\rho v_p^2}}\int_0^k\diff k'\sqrt{\td{E_k'}{k'}{}}
\end{align}
For us, before attenuation we have
\begin{align}
E_k = C_1 k^2\,,
\end{align}
so before attenuation we have
\begin{align}
\td{E_k'}{k'}{} = 2 C_1 k'\,,
\end{align}
and after attenuation
\begin{align}
\td{E_k'}{k'}{} = 2 C_1 k'e^{-k' v_p\f{T}{Q}}\,,
\end{align}
whence, the maximum amplitude in the vicinity of the ray impact is
\begin{align}
u &\leq \f{\sqrt{C_1}}{2\pi\sqrt{\rho v_p^2}}\int_0^k\diff k' \sqrt{k'}e^{-\f{k v_p}{2}\f{T}{Q}}\,,\\
&=\f{\sqrt{E_{\te{ray}}}}{2\pi k\sqrt{\rho v_p^2}}\int_0^k\diff k' \sqrt{k'}e^{-\f{k v_p}{2}\f{T}{Q}}\,,\\
&=\f{1}{2\pi kv_p}\sqrt{\f{E_{\te{ray}}}{\rho}}\f{Q}{v_p T}\p{\sqrt{2\pi}\sqrt{ \f{Q}{v_p T}}\te{Erf}\ps{\sqrt{\f{k v_p T}{2 Q}}} -2\sqrt{k} e^{-\f{k v_p T}{2 Q}}}\,.
\end{align}
This equation tells us the maximum displacement within the healpix pixel. Note that this is not a function of the area of the pixel. This is reasonable because the energy distribution is not a function of our partition of the sphere.
\pagebreak
\section{Directed Ray}
This time, we assume the oscillation is just in the $x$ direction.
\begin{align}
u(\vec x,t) =\int\f{\diff k}{2\pi}\, U(k)\chi_{A_{\te{ray}}} e^{-\I k(x -v_p t)}\,.
\end{align}
The part of $u$ that detectors will pick up is
\begin{align}
u(\vec x,t) =\int_0^k\f{\diff k'}{2\pi}\, U(k')\chi_{A_{\te{ray}}} e^{-\I k'(x -v_p t)}\,.
\end{align}
We approximate in the $x$ direction
\begin{align}
u(x,t) &\leq \int_0^k\f{\diff k'}{2\pi}\, \abs{U(k')}\,.
\end{align}
Now, we make use of the orthogonality of the basis functions to eliminate the spatial integral as above. Denote the width of the pulse as $L$.
\begin{align}
E &=\rho\int\diff^3 x\abs{\partial_t u}^2\,,\\
&=\rho v_p^2\int\diff^3 x\chi_{A_{\te{ray}}}\int\f{\diff k}{2\pi}\int\f{\diff k'}{2\pi}k k'U(k)\bar U(k') e^{-\I (k-k')(x -v_p t)}\,,\\
&=\rho v_p^2 A_{\te{ray}}\int\diff x\int\f{\diff k}{2\pi}\int\f{\diff k'}{2\pi}k k'U(k)\bar U(k') e^{-\I (k-k')(x -v_p t)}\,,\\
&=\rho v_p^2 A_{\te{ray}}\int\f{\diff k}{2\pi}\int \diff k'k k'U(k)\bar U(k') \delta(k-k')e^{\I (k-k')v_p t}\,,\\
&=\rho v_p^2 A_{\te{ray}}\int\f{\diff k}{2\pi}k^2\abs{U(k)}^2\,.
\end{align}
We denote, as before,
\begin{align}
E_k = \rho v_p^2 A_{\te{ray}}\int_0^k\f{\diff k}{2\pi}k^2\abs{U(k)}^2\,,
\end{align}
so
\begin{align}
\td{E_k}{k}{} =\f{\rho v_p^2 A_{\te{ray}}k^2}{2\pi}\abs{U(k)}^2\,.
\end{align}
Thus
\begin{align}
\abs{U(k)} = \sqrt{\f{2\pi}{\rho v_p^2 A_{\te{ray}}}}\f{1}{k}\sqrt{\td{E_k}{k}{}}\,,
\end{align}
and so
\begin{align}
u(x,t) \leq \sqrt{\f{1}{2\pi \rho v_p^2 A_{\te{ray}}}}\int_0^k\f{\diff k'}{k'}\sqrt{\td{E_k'}{k'}{}}\,.
\end{align}
Hence, in our case, where
\begin{align}
\td{E_k}{k}{} = 2C_1 k e^{-C_2 k}\,.
\end{align}
we have
\begin{align}
u(x,t) \leq \sqrt{\f{2}{\rho v_p^2 A_{\te{ray}}}}\sqrt{\f{C_1}{C_2}}\te{Erf}\ps{\sqrt{\f12 C_2 k}}
\end{align}
We can write
\begin{align}
C_1 = \f{E_\te{ray}}{k^2}\,,
\end{align}
hence
\begin{align}
u(x,t) \leq \sqrt{\f{2}{\rho v_p^2 A_{\te{ray}}}}\sqrt{\f{E_\te{ray}}{C_2}}\f1k\te{Erf}\ps{\sqrt{\f12 C_2 k}}\,.
\end{align}
Now, suppose that the time width of the wave is $\Delta T$. This corresponds to the minimum angular frequency associated to the wave. 
\begin{align}
\Delta T \geq \f{2\pi}{\omega} = \f{2\pi}{v_p k}\,.
\end{align}
Thus
\begin{align}
k \geq\f{2\pi}{v_p\Delta T}\,.
\end{align}
\pagebreak
\section{Maximum Detectible Surface Area}
\indent

The model above produces a set of positions on a sphere along with their associated times of impact and relative amplitudes at particular frequencies. The detectable energy from any given ray is the energy that is deposited in the frequencies detectable to a seismometer. According to Wikipedia, some seismometers can measure frequencies from 500 Hz to 0.00118 Hz. The lunar seismometers are considerably less sensitive, and have useful range 0.004 Hz to 2 Hz according to Wikipedia. There seems to be a total of 12 lunar seismometers.

Along with the use of the HEALPy package, we will be able to create a map on the sphere of maximum intensity for every distance $D$ of closest approach of a macro. The average detectable surface area of a macro impact is then a weighted average of these detectable surface areas for specific $D$. How do we weight the different $D$'s?

Suppose now that the macros impact the earth from all directions with equal likelihood. Consider any one of these macros. It sees a line, and it has equal likelihood of being any distance away from the origin. Now, since each point on the line exists as part of a circular cross-section, the associated distance has weight $2\pi R_{\te{Earth}}$. That's it! Hence, if $A_D$ is a finite set of detectible area measurements over $D = \{D_i:D_i\in[0,R_\te{Earth}],i\in\{1,\dots,N\}\}$. Assuming $D_i - D_{i+1}$ is fixed for all $i$, and that $D_N\approx R_{\te{Earth}}$ and $D_1 \approx 0$, then 
\begin{align}
A=\f{\sum_i A_{D_i}D_i}{\sum_i D_i}\,,
\end{align}
is the average detectable surface area. In the continuous case, this turns into the obvious integral.

Observe that this means that the events closest to the boundary of the earth are weighted the heaviest. These are also the events which deposit the least energy.

\pagebreak
\section{Thermal Source}
As relayed in an email to Josh, observe that the thermal source is
\begin{align}
T = \f{v_0^2}{c_p}\theta(r_X-r)\,,
\end{align}
where $c_p$ is the specific heat capacity of the material. The equation governing the evolution of the temperature field in 2 dimensions is, with no driving term,
\begin{align}
\partial_t T - \alpha\nabla^2 T = 0
\end{align}
where $\alpha$ is known as the thermal diffusivity, and $\alpha=\f{k}{c_p\rho}$. In cylindrical coordinates and for $T = T(r)$, we have
\begin{align}
\partial_t T - \alpha\nabla_r T = 0
\end{align}
We now Fourier transform in $r$ and $\phi$, yielding
\begin{align}
\partial_t\tilde T =-\alpha k_r^2 \tilde T\,,
\end{align}
where
\begin{align}
\tilde T = 2\pi \int_0^\infty r \diff r  T J_0(k_r r)\,.
\end{align}
Now, we integrate
\begin{align}
\tilde T = C_1e^{-\alpha k_r^2 t}
\end{align}
Now, to determine $C_1$, we set $T = \f{v_0^2}{c_p}\theta(r_X-r)$ at $t = 0$. Observe, this initial condition corresponds to the transformed statement
\begin{align}
\tilde T(0) = \f{r_X}{k_r}J_1(k_r r_X)\,.
\end{align}
This fixes $C_1$. We now inverse transform and obtain
\begin{align}
T &= \int_0^\infty k_r\diff k_r e^{-\alpha k_r^2 t}\f{r_X}{k_r}J_1(k_r r_X) J_0(k_r r)\,,\\
&=r_X\int_0^\infty \diff k_r e^{-\alpha k_r^2 t} J_1(k_r r_X) J_0(k_r r)\,.
\end{align}
We'll have to do this integral numerically. However, we are interested in solving for $r(t)$ such that $T(r(t),t) = T_{\te{melt}}$. We then want to find $\dot r$. Denote this special $r(t)$ as $r_\te{melt}$. Then
\begin{align}
\f{T_\te{melt}}{r_X} = \int_0^\infty \diff k_r e^{-\alpha k_r^2 t} J_1(k_r r_X) J_0(k_r r_\te{melt})
\end{align}
and
\begin{align}
0&=\int_0^\infty \diff k_r \ps{e^{-\alpha k_r^2 t} J_1(k_r r_X) \p{-k_r\dot r_\te{melt}}J_1(k_r r_\te{melt}) +  \p{-\alpha k_r^2}e^{-\alpha k_r^2 t} J_1(k_r r_X)J_0(k_r r_\te{melt})}\,,\\
0&=\int_0^\infty k_r \diff k_r e^{-\alpha k_r^2 t} J_1(k_r r_X) \ps{\p{\dot r_\te{melt}}J_1(k_r r_\te{melt}) +  \p{\alpha k_r}J_0(k_r r_\te{melt})}\,.
\end{align}
\pagebreak\\
Eh... None of this is producing useful results, i.e. those we can use to solve for $r_\te{melt}$. Let's consider a delta source i.e.
\begin{align}
T = \f{v_0^2}{c_p}r_X\delta(r)\,.
\end{align}
Then
\begin{align}
C_1 =2\pi \f{v_0^2}{c_p}r_x
\end{align}
Thus,
\begin{align}
T &= \int_0^\infty k_r\diff k_r e^{-\alpha k_r^2 t}J_0(k_r r)2\pi \f{v_0^2}{c_p}r_x\,,\\
&=\f{r_X v_0^2}{2 c_p t \alpha} e^{-\f{r^2}{4t\alpha}}\,.
\end{align}
Setting $T = T_\te{melt}$, we have
\begin{align}
r_\te{melt} = \sqrt{4t\alpha\ln\ps{\f{r_X v_0^2}{2 c_p t \alpha T_\te{melt}}}}\,.
\end{align}
where we've taken the positive branch, since we only care about the melt radius increasing. Then
\begin{align}
\dot r_\te{melt} &=\sqrt{\f{\alpha}{t}}\f{\ln\p{\f{r_Xv_0^2}{2 c_p t\alpha T}} - 1}{\sqrt{\ln\p{\f{r_Xv_0^2}{2 c_p t\alpha T}}}}\,.
\end{align}
For our parameters, it is a good approximation to take
\begin{align}
\dot r_\te{melt} &=\sqrt{\f{\alpha}{t}\ln\p{\f{r_Xv_0^2}{2 c_p\alpha T}}}\,.
\end{align}
Thus, the time at which the velocity of the melt front is that of the speed of sound is
\begin{align}
t_{\te{fast}} = \f{\alpha}{v_p^2}\ln\p{\f{r_Xv_0^2}{2 c_p\alpha T_\te{melt}}}
\end{align}
at a radius, which is a good approximation in our case, of
\begin{align}
r_\te{fast melt} =  2\f{\alpha}{v_p}\ln\ps{\f{r_X v_0^2}{2 c_p \alpha T_\te{melt}}}
\end{align}
which is on the order of micrometers to tenths of micrometers for our case.
\end{document}
