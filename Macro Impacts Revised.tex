\documentclass{article}

% Packages

\usepackage{fullpage}
\usepackage{amsmath, amsthm, amsfonts, amssymb, mathtools, calrsfs, tensor, physics,tikz-cd}
\usepackage[mathscr]{euscript}
\usepackage{graphicx}
\graphicspath{ {images/} }
\usepackage{enumitem}
\setlist[description]{font=\normalfont}

% Custom Commands

\newcommand*\diff{\mathop{}\!\mathrm{d}}
\newcommand*\Diff[1]{\mathop{}\!\mathrm{d^#1}}
\newcommand*\nrml{\vartriangleleft}
\newcommand*\scr[1]{\mathscr{#1}}
\newcommand*\bb[1]{\mathbb{#1}}
\newcommand*\la{\langle}
\newcommand*\ra{\rangle}
\newcommand*\gen[1]{\langle #1 \rangle}
\newcommand*\x{\times}
\newcommand*\st{\text{ s.t. }}
\newcommand*\ord[1]{\left\vert#1\right\vert}
\newcommand*\aut{\text{Aut}}
\newcommand*\lcm{\text{lcm}}
\newcommand*\mcal{\mathcal}
\newcommand*\es{\emptyset}
\newcommand*\im{\text{ Im }}
\newcommand*\N{\mathbb N}
\newcommand*\Z{\mathbb Z}
\newcommand*\R{\mathbb R}
\newcommand*\Q{\mathbb Q}
\newcommand*\C{\mathbb C}
\newcommand*\te[1]{\text{#1}}
\newcommand*\en[1]{\begin{enumerate}#1\end{enumerate}}
\newcommand*\e{\varepsilon}
\newcommand*\p[1]{\left(#1\right)}
\newcommand*\ps[1]{\left[#1\right]}
\newcommand*\pc[1]{\left\{#1\right\}}
\newcommand*\f[2]{\frac{#1}{#2}}
\newcommand*\mat[2]{\left(\begin{array}{#1}#2\end{array}\right)}
\newcommand*\ocross{\otimes}
\newcommand*\I{\te{i}}
\newcommand*\pd[3]{\frac{\partial^{#3} #1}{\partial {#2}^{#3}}}
\newcommand*\td[3]{\frac{d^{#3}#1}{d #2^{#3}}}
\newcommand*\m{\te{Mat}}
\newcommand*\End{\te{End}}
\newcommand*\irr{\te{Irr}}
\newcommand*\sgn{\te{sgn}}
\newcommand*\pn[2]{\left\|#1\right\|_{#2}}
\newcommand*\esssup{\te{ess sup}}
\newcommand*\essinf{\te{ess inf}}

% Miscellaneous

\newtheorem{theorem}{Theorem}
\usetikzlibrary{matrix,arrows,decorations.pathmorphing}

% Title
\title{Macro Impacts}
\date{\today}

\begin{document}
\maketitle
\section{Scalar Modes}
In the following, lower case denotes a quantity in position space while capital letters denote their components in Fourier space.
\\\\
Denote the displacement field $u(\vec x,t)=\nabla\phi(\vec x,t)+\nabla\times a(\vec x,t)$\,. The linearized (acoustic) wave equation for $\phi$ is then
\begin{align}
\alpha^2\nabla^2\phi&=\partial_t^2\phi\,,
\end{align}
where $\alpha^2=\f{\lambda+2\mu}{\rho}$. We may write
\begin{align}
\phi(\vec x,t)=\int\f{\diff^3 k}{(2\pi)^3}\Phi(\vec k,\omega)e^{\I(\vec k\cdot\vec x-\omega t)}\,,
\end{align}
from which we obtain the dispersion relation $\alpha k=\omega$, and hence the solution
\begin{align}
\phi(\vec x,t)=\int\f{\diff^3 k}{(2\pi)^3}\Phi(\vec k,\omega)e^{\I(\vec k\cdot\vec x-\alpha k t)}\,.
\end{align}
We now impose the constraint equation
\begin{align}
\sigma_{ij}&=\delta_{ij}\lambda\nabla\cdot u+\mu(u_{i,j}+u_{j,i})
\end{align}
which, for the scalar modes becomes
\begin{align}
\sigma_{ij}&=\delta_{ij}\lambda\nabla^2\phi+2\mu\partial_i\partial_j\phi
\end{align}
and whose trace is
\begin{align}
-p=K\nabla^2\phi=K\nabla\cdot u\,,
\end{align}
where $K=\lambda+\f23\mu$ is the bulk modulus and $p=-\f13\tr\sigma_{ij}$. We take this constraint as an initial condition at $t=0$. In Fourier space
\begin{align}
P(\vec k)=Kk^2\Phi(\vec K)\,,
\end{align}
from which we obtain the displacement field components
\begin{align}
U(\vec k)&=\I\f{P(\vec k)}{K}\f{\vec k}{k^2}\,,
\end{align}
and hence
\begin{align}
u(\vec x,t)=\int\f{\diff^3 k}{(2\pi)^3}\I\f{P(\vec k)}{K}\f{\vec k}{k^2}e^{\I(\vec k\cdot\vec x-\alpha k t)}\,,
\end{align}
The equation for the energy of the compressional modes is
\begin{align}
E&=\f12\int_V\diff^3x\p{\rho\abs{\partial_t u}^2+(\lambda+2\mu)\abs{\nabla\cdot u}^2}\,,\\
&=\f{\lambda+2\mu}{K^2}\int\f{\diff^3 k}{(2\pi)^3}\abs{P(\vec k)}^2
\end{align}
In the case that $P$ depends only on the frequency
\begin{align}
E&=4\pi\f{\lambda+2\mu}{K^2}\int\f{\diff\tilde\lambda}{\tilde\lambda^4}\abs{P(\tilde\lambda)}^2\,.
\end{align}
\\\\
Observe, the total energy can be calculated directly from $p$ as it is always true that half the energy is spring potential and half is kinetic.
\begin{align}
E&=\f{\lambda+2\mu}{K^2}\int_V\diff^3 x (p^2)
\end{align}
Now consider the case of a step function type pressure source - a cylinder of height $h$ and radius $r_X$
\begin{align}
p(\vec x,0)&=p_0\theta(r_X-r)\ps{\theta(z+h/2)-\theta(z-h/2)}\,,
\end{align}
whose Fourier components are
\begin{align}
P(\vec k)&=\f{4\pi r_X p_0}{\sqrt{k_x^2+k_y^2}k_z}J_1\p{\sqrt{k_x^2+k_y^2}r_X}\sin\p{\f{h}{2}k_z}\,,
\end{align}
which, in polar $k$-space is
\begin{align}
P(k,\varphi,\theta)&=\f{4\pi r_X p_0}{k^2\sin\theta\cos\theta}J_1\p{r_Xk\sin\theta}\sin\p{\f{h}{2}k\cos\theta}\,,
\end{align}
The total energy is clearly
\begin{align}
E_{\te{total}}&=\f{\lambda+2\mu}{K^2}p_0^2\sigma_Xh
\end{align}
where $\sigma_X=\pi r_X^2$. To calculate the energy deposition into the low frequency spectrum, we integrate $k$ from $0$ to $k_0$. Observe, we can make the following approximation
\begin{align}
P(\vec k)&\approx_{k\ll r_X}\f{2\pi r_Xp_0}{k^2\cos\theta\sin\theta}r_X k\sin\theta\sin\p{\f{h}{2}k\cos\theta}\,,\\
&=\f{2\pi r_X^2 p_0}{k\cos\theta}\sin\p{\f{h}{2}k\cos\theta}\,.
\end{align}
From this we obtain the portion of the energy relegated to the long wavelength spectrum
\begin{align}
E_{\te{propagated}}&=\f{\lambda+2\mu}{K^2}\int\f{\diff^3 k}{(2\pi)^3}\abs{\f{2\pi r_X^2 p_0}{k\cos\theta}\sin\p{\f{h}{2}k\cos\theta}}^2\,,\\
&=\ps{\f{1}{2h}(r_X^2p_0)^2\f{\lambda+2\mu}{K^2}}\ps{hk_0\cos(hk_0)+\sin(hk_0)+hk_0\p{-2+hk_0\te{Si}(hk_0)}}\,,\\
&\approx_{h\gg\lambda_0\gg1}\ps{(r_X^2p_0)^2\f{\lambda+2\mu}{K^2}}\ps{\f{\pi^3 h}{\lambda_0^2}}\,,\\
&=\ps{(\sigma_Xp_0)^2\f{\lambda+2\mu}{K^2}}\ps{\f{\pi h}{\lambda_0^2}}\,.
\end{align}
From this we obtain the fractional energy deposition into the unattenuated wavelengths
\begin{align}
\Xi&=\f{\sigma_X}{\lambda_0^2}\,.
\end{align}
This approximation holds for $\lambda_0^2\gg\sigma_X$, which is appropriate for the case that $\lambda_0$ is on the order of kilometers and $\sigma_X$ is on the order of centimeters squared.
\pagebreak
\\
It is wrong, however, to assume that the wave evolves linearly close to the source. Modes will be coupled to one another, and our expression (24) will only hold in a small neighborhood of the event. Effects such as heating and rock fracturing will cause the high frequency modes in the shock to rapidly attenuate. They are, however, difficult to quantify. It is not controversial to say that the energy of the shock after its non-linear evolution (when the overpressure exceeds the elastic limit of the Earth) will be less than its initial energy. It is also well known that the behavior of shockwaves for long time tends towards a sharp wave-front with a linearly decreasing tail. To produce an over-estimate of the detectible energy, we hypothesize an approximate pressure waveform and endow it with energy equivalent to that of the initial blast. Then, we calculate the energy per mode, and sum over only those modes whose frequency is in the regime that will not rapidly attenuate.
\\\\
We describe the long-time waveform of the shock by
\begin{align}
p=\bar p\f{r-r_0}{\Delta r}\ps{\theta(r_0+\Delta r-r)-\theta(r+\Delta r)}\ps{\theta(z+h/2)-\theta(z-h/2)}\,.
\end{align}
$\Delta r$ is the length of the tail, and $r_0$ is its base. $\bar p$ is the peak pressure of the shock, and we can take it to be the stress corresponding to the elastic limit of rock. The total energy of this pressure wave is
\begin{align}
E_{\te{total}}=\f{\lambda+2\mu}{K^2}\f16 \bar p^2 \pi\Delta r h(4 r_0+3\Delta r)\,.
\end{align}
We require that this be equal to the initial total energy. From this, we can obtain an expression for $r_0$ in terms of $\Delta r$.
\begin{align}
r_0&=\f{3(2p_0^2r_X^2-\bar p^2\Delta r^2)}{4\bar p^2\Delta r}\,.
\end{align}
A lower bound for $r_0$ is clearly 0, which sets an upper bound for $\Delta r$, i.e.
\begin{align}
\Delta r&\leq \f{\sqrt{2}p_0r_X}{\bar p}\,,\\
&=\f{1}{\bar p}\sqrt{\f{2}{\pi}\f{K^2}{\lambda+2\mu}\abs{\td{E}{x}{}}}\,.
\end{align}
Plugging in some numbers (from below) yields
\begin{align}
\Delta r\leq 1.2\te{m}\,.
\end{align}
The energy in the longest modes is, for $h\gg\lambda_0\gg r_0+\Delta r$
\begin{align}
E_{\te{propagated}}&\approx\f{\lambda+2\mu}{K^2}\f{\bar p^2 \pi^3 h \Delta r^2(3r_0+2\Delta r)^2}{9\lambda_0^2}\,.
\end{align}
Then we have that the fraction of energy that will reach seismometers is
\begin{align}
\Xi&=\f{2\pi^2\Delta r(3r_0+2\Delta r)^2}{3(4 r_0+3\Delta r)\lambda_0^2}\,,\\
&=\pi^2\p{\f{18 p_0^2 r_X^2-\bar p^2\Delta r^2}{12 p_0\bar p r_X\lambda_0}}^2
\end{align}
Since we know that the total energy is (assuming that the velocity of the macro remains approximately constant throughout its journey through Earth) we can calculate the initial overpressure.
\begin{align}
\f{\lambda+2\mu}{K^2}p_0^2\sigma_Xh=E_{\te{total}}=h\abs{\td{E}{x}{}}=h\rho_{\oplus}\sigma_Xv_X^2\,.
\end{align}
Thus we obtain the following expression for $p_0$
\begin{align}
p_0&=\sqrt{\f{\rho}{\lambda+2\mu}}Kv_X\,.
\end{align}
Thus, the suppression factor is
\begin{align}
\Xi&=\pi \f{(\bar p^2\pi \Delta r^2(\lambda+2\mu)-18K^2\sigma_Xv_X^2\rho_{\oplus})^2}{144K^2(\lambda+2\mu)\rho_{\oplus}\bar p^2 \sigma_X v_X^2\lambda_0^2}\,.
\end{align}
Reviewing a USGS document (Rock Failure and Earthquakes) suggests that $\bar p$ be set at $10^8$. To calculate $\Delta r$, notice that the Earth is initially displaced by $r_X$. The Earth will then collapse back into the hole due to ambient pressure at the speed of sound $\alpha$, yielding a time of collapse $r_X/\alpha$. The speed of the shockwave initially produced by the object is approximately the speed of the object itself. Hence $\Delta r/v_X=r_X/\alpha$. Thus $\Delta r\approx\f{v_X}{\alpha}r_X$. Thus
\begin{align}
\Xi&=\pi \f{(\bar p^2\pi \f{v_X^2}{\alpha^2}r_X^2(\lambda+2\mu)-18K^2\sigma_Xv_X^2\rho_{\oplus})^2}{144K^2(\lambda+2\mu)\rho_{\oplus}\bar p^2 \sigma_X v_X^2\lambda_0^2}\,,\\
&=\p{\pi r_X v_X\f{\bar p^2 -18K^2}{12K\alpha\bar p \lambda_0}}^2\,.
\end{align}
Given the relative magnitudes of $\bar p$ and $K^2$,
\begin{align}
\Xi&=\p{\f{3\pi r_X v_XK}{2\alpha\bar p \lambda_0}}^2\,,\\
&=\p{\f{K^2}{\lambda+2\mu}}\p{\f{3\pi}{2\bar p\lambda_0}}^2\abs{\td{E}{x}{}}\,.
\end{align}
The following are the approximate quantities used
\begin{align}
h&=1.2\times 10^7\te{m}\,,\\
v_X&=2.5\times 10^6\te{m s}^{-1}\,,\\
\rho_{\oplus}&=6\times 10^3\te{kg m}^{-3}\,,\\
\sigma_X&=10^{-11}\te{m}^2\,,\\
\abs{\td{E}{x}{}}&=\rho_{\oplus}\sigma_X v_X^2=3.75\times 10^5\te{J m}^{-1}\,,\\
E_{\te{total}}&=4.5\times 10^{12}\te{J}\,,\\
r_X&=1.8\times 10^{-6}\te{m}\,,\\
\alpha&=6\times 10^3\te{m s}^{-1}\,,\\
\beta&=3.5\times 10^{3}\te{m s}^{-1}\,,\\
\mu&=7.2\times 10^{10}\te{Pa}\,,\\
\lambda&=7.2\times 10^{10}\te{Pa}\,,\\
K&=1.2\times 10^{11}\te{Pa}\,,\\
\bar p&=10^8\te{Pa}\,.
\end{align}
These lead to the results
\begin{align}
\Delta r&=7.4\times 10^{-4}\te{m}\,,\\
r_0&=1.6\times 10^3\te{m}\,,\\
E_{\te{propagated}}&=\f{8\times 10^{13}\te{J m}^2}{\lambda_0^2}\,,\\
\Xi&=\f{17.6\te{m}^2}{\lambda_0^2}\,.
\end{align}
Conservatively we set $\lambda_0$ at about $6$ kilometers. This yields a total suppression factor of $\Xi=4.9\times 10^{-7}$. Keep in mind that this estimate is incredibly conservative, and does not take into account any non-conservative processes in the nonlinear regime $r<r_0$. Note, even though both $\lambda_0$ and $r_0$ are on the same order of magnitude, the approximations used are still good since they are overestimates of the Bessel $J$ and Struve $H$ functions.
\\\\
Our result shows that the suppression factor only depends on the stopping factor, which in turn only depends on the velocity and cross-section of the macro. This is sensible since the velocity and cross-section of the macro are the defining characteristics of the impact. The only factor missing is the density of the macro. However, we are assuming that the density is large enough that the average velocity of the macro on its journey through the earth is close to its initial velocity.
\\\\
It is significant that $\Xi$ is proportional to $\sigma_X$. For the very largest macros, our very conservative suppression factor is near unity. However, it may still be enough to rule out detection.
\\\\
Note, it is incorrect to consider the expression for $\Xi$ in the small and large parameter limits: we required (in some approximations) that $r_0\ll \lambda_0$, which places a lower bound on $\bar p$. Moreover $\bar p$ is supposed to be less than $p_0$.
\pagebreak
\section{Vector Modes}
The following is the equation of motion for the vector potential $a$ without body force:
\begin{align}
\mu\nabla^2a-\rho\ddot a=\mu\nabla(\nabla\cdot a)-\mu\nabla\times(\nabla\times a)-\rho\ddot a=0\,,
\end{align}
where the leftmost term is the vector laplacian. This gives us three wave equations for the three components of $a$. The constraint equation takes the form
\begin{align}
\sigma_{ij}&=\lambda\delta_{ij}\nabla\cdot \nabla\times a+\mu(\partial_j (\nabla\times a)_i+\partial_i (\nabla\times a)_j)\,,\\
&=\mu(\partial_j (\nabla\times a)_i+\partial_i (\nabla\times a)_j)\,,\\
&=\mu\mat{ccc}{2\partial_x(\partial_ya_z-\partial_za_y)&(\partial_y^2-\partial_x^2)a_z+\partial_z(\partial_x a_x-\partial_y a_y)&(\partial_x^2-\partial_z^2)a_y+\partial_y(\partial_z a_z-\partial_x a_x)\\
&2\partial_y(\partial_z a_x-\partial_x a_z)&(\partial_z^2-\partial_y^2)a_x+\partial_x(\partial_ya_y-\partial_za_z)\\
&&2\partial_z(\partial_xa_y-\partial_ya_x)}\,.
\end{align}
We now write $a_i$ in terms of its Fourier components
\begin{align}
a_i(\vec x,t)&=\int\f{\diff^3k}{(2\pi)^3}A_i(\vec k,\omega)e^{\I(\vec k\cdot\vec x-\omega t)}\,.
\end{align}
Observe the dispersion relation $\beta k=\omega$, which holds for each component $a_i$, and that $\beta^2=\f\mu\rho$. Thus, the solutions to the wave equation are
\begin{align}
a_i(\vec x,t)&=\int\f{\diff^3k}{(2\pi)^3}A_i(\vec k,\omega)e^{\I(\vec k\cdot\vec x-\beta k t)}\,.
\end{align}
We now use the constraint equation to write down the initial vector potential in terms of the initial stress. Note, we will obtain three equations constraining the possible initial stresses (i.e., we will be able to eliminate three components of $\sigma$ from the equations), and three equations writing $A_i$ in terms of $\Sigma_{ij}$.
\\\\
In Fourier space we have
\begin{align}
-\Sigma_{xx}&=\mu2k_x(k_y A_z-k_z A_y)\,,\\
-\Sigma_{yy}&=\mu2k_y(k_z A_x-k_x A_z)\,,\\
-\Sigma_{zz}&=\mu2k_z(k_x A_y-k_y A_x)\,,\\
-\Sigma_{xy}&=\mu\ps{(k_y^2-k_x^2) A_z+k_z(k_x A_x-k_y A_y)}\,,\\
-\Sigma_{yz}&=\mu\ps{(k_z^2-k_y^2) A_x+k_x(k_y A_y-k_z A_z)}\,,\\
-\Sigma_{zx}&=\mu\ps{(k_x^2-k_z^2) A_y+k_y(k_z A_z-k_x A_x)}\,.
\end{align}
First observe that any three of these equations together forms a singular linear system in $A_i$. Thus we cannot invert to solve for $A_i$. However, we can get some information about the $\Sigma$'s.
\\\\
Taking the sum of 31-33 yields
\begin{align}
0&=\Sigma_{xx}+\Sigma_{yy}+\Sigma_{zz}\,.
\end{align}
Observe that
\begin{align}
\f{-\Sigma_{xx}k_y}{2\mu k_xk_z}&=\f{k_y^2}{k_z}A_z-k_yA_y\,,\\
\f{-\Sigma_{yy}k_x}{2\mu k_yk_z}&=k_xA_x-\f{k_x^2}{k_z}A_z\,,\\
\f{-\Sigma_{xy}}{\mu k_z}&=\ps{\f{k_y^2-k_z^2}{k_z}}A_z+k_xA_x-k_yA_y\,.
\end{align}
Taking the sum of the first two reveals that`
\begin{align}
2k_xk_y\Sigma_{xy}&=k_y^2\Sigma_{xx}+k_x^2\Sigma_{yy}\,.
\end{align}
If you want to be silly, we find
\begin{align}
0&=(\Sigma_{x}k_y-\Sigma_{y}k_x)^2\,.
\end{align}
Anyway, we then have the system of three equations
\begin{align}
2k_xk_y\Sigma_{xy}&=k_y^2\Sigma_{xx}+k_x^2\Sigma_{yy}\,,\\
2k_yk_z\Sigma_{yz}&=k_z^2\Sigma_{yy}+k_y^2\Sigma_{zz}\,,\\
2k_zk_x\Sigma_{zx}&=k_x^2\Sigma_{zz}+k_z^2\Sigma_{xx}\,.
\end{align}
which is non-singular. They yield
\begin{align}
\Sigma_{xx}&=\f{k_x}{k_yk_z}\p{k_z\Sigma_{xy}-k_x\Sigma_{yz}+k_y\Sigma_{zx}}\,,\\
\Sigma_{yy}&=\f{k_y}{k_zk_x}\p{k_z\Sigma_{xy}+k_x\Sigma_{yz}-k_y\Sigma_{zx}}\,,\\
\Sigma_{zz}&=\f{k_z}{k_xk_y}\p{-k_z\Sigma_{xy}+k_x\Sigma_{yz}+k_y\Sigma_{zx}}\,.
\end{align}
This, along with being traceless, shows that the shear only has two independent degrees of freedom: the polarizations.
\\\\
It is impossible to solve for $A_i$ in terms of the stress as is, hence we must fix a gauge. To do this, we fix $\nabla\cdot a=0$, whence
\begin{align}
0=k_xA_x+k_yA_y+k_zA_z\,.
\end{align}
From earlier
\begin{align}
-\Sigma_{yz}&=\mu\ps{(k_z^2-k_y^2) A_x+k_x(k_y A_y-k_z A_z)}\,,\\
-\Sigma_{zx}&=\mu\ps{(k_x^2-k_z^2) A_y+k_y(k_z A_z-k_x A_x)}\,.
\end{align}
Imposing the Gauge condition,
\begin{align}
-\Sigma_{yz}&=\mu\ps{(k_x^2-k_y^2+k_z^2) A_x+2k_xk_y A_y}\,,\\
-\Sigma_{zx}&=\mu\ps{(k_x^2-k_y^2-k_z^2) A_y-2k_x k_y A_x}\,.
\end{align}
Inverting:
\begin{align}
A_z=\f{k_x\Sigma_{yz}-k_y\Sigma_{zx}}{\mu k_z k^2}
\end{align}
Hence, by symmetry
\begin{align}
A_x&=\f{k_y\Sigma_{zx}-k_z\Sigma_{xy}}{\mu k_x k^2}\,,\\
A_y&=\f{k_z\Sigma_{xy}-k_x\Sigma_{yz}}{\mu k_y k^2}\,,\\
A_z&=\f{k_x\Sigma_{yz}-k_y\Sigma_{zx}}{\mu k_z k^2}\,.
\end{align}
\\\\
Now we take the curl of $A$ to find the shear mode displacement and use the relations (73) through (75) to obtain
\begin{align}
U_x&=-\I\f{\Sigma_{xx}}{2k_x\mu}\,,\\
U_y&=-\I\f{\Sigma_{yy}}{2k_y\mu}\,,\\
U_z&=-\I\f{\Sigma_{zz}}{2k_z\mu}\,,
\end{align}
The energy of these shear waves is given
\begin{align}
E&=\f12\int_V\p{\rho\abs{u_t}^2+\mu\abs{\nabla\cdot u}^2}\,,\\
&=\rho\int_V\abs{u_t}^2\,,\\
&=\rho\beta^2\int_V\int\f{\diff^3 k}{(2\pi)^3}\int\f{\diff^3 k'}{(2\pi)^3}\p{k k'U(\vec k)\bar U(\vec k')}e^{-\I(\vec k\cdot\vec x-i\beta k t)}e^{\I(\vec k'\cdot\vec x-i\beta k' t)}\,,\\
&=\mu\int\f{\diff^3 k}{(2\pi)^3}\p{k^2\abs{U(\vec k)}^2}\,,\\
&=\mu\int\f{\diff^3 k}{(2\pi)^3}\p{\f{\Sigma_{zz}^2\sec^2\theta+\csc^2\theta(\Sigma_{yy}^2\csc^2\phi+\Sigma_{xx}^2\sec^2\phi)}{4\mu^2}}\,.
\end{align}
Assuming axial symmetry, we have $\Sigma_{xx}=\Sigma\cos^2\phi$, $\Sigma_{yy}=\Sigma\sin^2\phi$, and $\Sigma_{zz}=-\Sigma$. Note that this choice is justified since the full stress tensor is invariant under rotations about the $z$-axis. Hence, 
\begin{align}
E&=\mu\int\f{\diff^3 k}{(2\pi)^3}\Sigma^2\p{\f{\sec^2\theta+\csc^2\theta}{4\mu^2}}\,,\\
&=\f{1}{\mu}\int\f{\diff^3 k}{(2\pi)^3}\Sigma^2\csc^2(2\theta)\,,\\
&=\f{1}{\mu}\int\f{\diff k\diff\theta}{(2\pi)^2}\Sigma^2k^2\csc^2(2\theta)\sin\theta\,,\\
&=\f{1}{4\mu}\int\f{\diff k\diff\theta}{(2\pi)^2}\Sigma^2k^2\csc(\theta)\sec^2(\theta)\,.
\end{align}
In cartesian coordinates:
\begin{align}
E&=\f{1}{4\mu}\int\f{\diff^3 k}{(2\pi)^3}\Sigma^2\f{k^4}{(k_x^2+k_y^2)k_z^2}\,.
\end{align}
Something makes me think that $\csc^2(2\theta)$ should be absorbed into the definition of $\Sigma$, but I don't know how to show this.
\section{}
For a given macro of given cross-section and velocity, how much energy gets deposited into seismic waves that propagated unattenuated? A better question: as a function of the sensitivity of a seismometer, will the seismometers have measured the macro? Look at the lunar limits.
\end{document}
